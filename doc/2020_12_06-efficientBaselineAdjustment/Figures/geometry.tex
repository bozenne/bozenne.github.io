% Created 2021-06-30 on 14:38
% Intended LaTeX compiler: pdflatex
\documentclass{article}

%%%% settings when exporting code %%%% 

\usepackage{listings}
\lstdefinestyle{code-small}{
backgroundcolor=\color{white}, % background color for the code block
basicstyle=\ttfamily\small, % font used to display the code
commentstyle=\color[rgb]{0.5,0,0.5}, % color used to display comments in the code
keywordstyle=\color{black}, % color used to highlight certain words in the code
numberstyle=\ttfamily\tiny\color{gray}, % color used to display the line numbers
rulecolor=\color{black}, % color of the frame
stringstyle=\color[rgb]{0,.5,0},  % color used to display strings in the code
breakatwhitespace=false, % sets if automatic breaks should only happen at whitespace
breaklines=true, % sets automatic line breaking
columns=fullflexible,
frame=single, % adds a frame around the code (non,leftline,topline,bottomline,lines,single,shadowbox)
keepspaces=true, % % keeps spaces in text, useful for keeping indentation of code
literate={~}{$\sim$}{1}, % symbol properly display via latex
numbers=none, % where to put the line-numbers; possible values are (none, left, right)
numbersep=10pt, % how far the line-numbers are from the code
showspaces=false,
showstringspaces=false,
stepnumber=1, % the step between two line-numbers. If it's 1, each line will be numbered
tabsize=1,
xleftmargin=0cm,
emph={anova,apply,class,coef,colnames,colNames,colSums,dim,dcast,for,ggplot,head,if,ifelse,is.na,lapply,list.files,library,logLik,melt,plot,require,rowSums,sapply,setcolorder,setkey,str,summary,tapply},
aboveskip = \medskipamount, % define the space above displayed listings.
belowskip = \medskipamount, % define the space above displayed listings.
lineskip = 0pt} % specifies additional space between lines in listings
\lstset{style=code-small}
%%%% packages %%%%%

\usepackage[utf8]{inputenc}
\usepackage[T1]{fontenc}
\usepackage{lmodern}
\usepackage{textcomp}
\usepackage{color}
\usepackage{graphicx}
\usepackage{grffile}
\usepackage{wrapfig}
\usepackage{rotating}
\usepackage{longtable}
\usepackage{multirow}
\usepackage{multicol}
\usepackage{changes}
\usepackage{pdflscape}
\usepackage{geometry}
\usepackage[normalem]{ulem}
\usepackage{amssymb}
\usepackage{amsmath}
\usepackage{amsfonts}
\usepackage{dsfont}
\usepackage{array}
\usepackage{ifthen}
\usepackage{hyperref}
\usepackage{natbib}
%
%%%% additional packages %%%%
%
\usepackage{authblk}
\usepackage[pdftex,active,tightpage]{preview} % to get standalone diagram
\RequirePackage{fancyvrb}
\DefineVerbatimEnvironment{verbatim}{Verbatim}{fontsize=\small,formatcom = {\color[rgb]{0.5,0,0}}}
\RequirePackage{epstopdf} % to be able to convert .eps to .pdf image files
\RequirePackage{bm}
\newcommand{\Model}{\mathcal{M}}
\newcommand{\VModel}{\bm{\mathcal{M}}}  % require bm package
%
%%%% additional latex commands %%%%
%
\newcommand{\lightblue}{blue!50!white}
\newcommand{\darkblue}{blue!80!black}
\newcommand{\darkgreen}{green!50!black}
\newcommand{\darkred}{red!50!black}
\definecolor{gray}{gray}{0.5}
\RequirePackage{tikz}
\RequirePackage{tikz-3dplot}
\tikzset{font=\LARGE}
\tdplotsetmaincoords{60}{125}
\tdplotsetrotatedcoords{0}{0}{0} %<- rotate
\author{Brice Ozenne}
\date{\today}
\title{}
\hypersetup{
 colorlinks=true,
 pdfauthor={Brice Ozenne},
 pdftitle={},
 pdfkeywords={},
 pdfsubject={},
 pdfcreator={Emacs 27.1 (Org mode 9.4.5)},
 pdflang={English}
 }
\begin{document}

\begin{preview}
  \begin{tikzpicture}[scale=1.6,tdplot_rotated_coords,
                    cube/.style={very thick,black},
                    grid/.style={very thin,gray},
                    axis/.style={->,ultra thick},
                    rotated axis/.style={->,purple,ultra thick}]

     %draw the tangent space
     \draw [\darkblue,fill=blue!20] (0,0,0) -- (3,0,0) -- (3,3,0) -- (0,3,0) -- cycle;
     \node[\darkblue] at (2.6,2.75,0) (Label) {\(\mathcal{T}\)}; 

     %draw the set of influence functions
     \draw [\darkgreen,fill=green!20] (0,0,0) -- (3,0,0) -- (3,0,3) -- (0,0,3) -- cycle;
     \node[\darkgreen] at (2.5,0,2.6) (Label) {\(\mathcal{T}_{\eta}^{\perp}\)}; 

     %draw the main coordinate frame axes
     \draw[axis,tdplot_main_coords,\darkblue] (0,0,0) -- (4,0,0) node[anchor=north east]{$\mathcal{T}_{\psi}$};
     \draw[axis,tdplot_main_coords,\darkblue] (0,0,0) -- (0,4,0) node[anchor=north west]{$\mathcal{T}_{\eta}$};
     \draw[axis,tdplot_main_coords,\darkgreen] (0,0,0) -- (0,0,3) node[anchor=west]{$\mathcal{T}^{\perp}$};

     % draw score function
     \draw[dashed,very thick] (2,0,0)--(2,1,0);
     \draw[->,ultra thick] (0,0,0)--(2,1,0) node[anchor=west]{$\mathcal{S}_{\psi}$};
     \draw[->,ultra thick] (0,0,0)--(2,0,0) node[anchor=north]{$\mathcal{S}_{eff}$};

     % draw influence function
     \draw[dashed,very thick] (1.5,0,0)--(1.5,0,2);
     \draw[->,ultra thick] (0,0,0)--(1.5,0,2) node[anchor=south west]{$\mathcal{IF}$};
     \draw[->,ultra thick,purple!80] (0,0,0)--(1.5,0,0) node[anchor=south east]{$\mathcal{IF}_{eff}=\Pi\left(\mathcal{IF}|\mathcal{T}\right)$};

     % origin
     \shade[ball color = black] (0,0,0) circle (0.06cm) node[anchor=south west]{O};


 \end{tikzpicture}

\end{preview}
\end{document}