% Created 2024-10-22 ti 11:24
% Intended LaTeX compiler: pdflatex
\documentclass[12pt]{article}

%%%% settings when exporting code %%%% 

\usepackage{listings}
\lstdefinestyle{code-small}{
backgroundcolor=\color{white}, % background color for the code block
basicstyle=\ttfamily\small, % font used to display the code
commentstyle=\color[rgb]{0.5,0,0.5}, % color used to display comments in the code
keywordstyle=\color{black}, % color used to highlight certain words in the code
numberstyle=\ttfamily\tiny\color{gray}, % color used to display the line numbers
rulecolor=\color{black}, % color of the frame
stringstyle=\color[rgb]{0,.5,0},  % color used to display strings in the code
breakatwhitespace=false, % sets if automatic breaks should only happen at whitespace
breaklines=true, % sets automatic line breaking
columns=fullflexible,
frame=single, % adds a frame around the code (non,leftline,topline,bottomline,lines,single,shadowbox)
keepspaces=true, % % keeps spaces in text, useful for keeping indentation of code
literate={~}{$\sim$}{1}, % symbol properly display via latex
numbers=none, % where to put the line-numbers; possible values are (none, left, right)
numbersep=10pt, % how far the line-numbers are from the code
showspaces=false,
showstringspaces=false,
stepnumber=1, % the step between two line-numbers. If it's 1, each line will be numbered
tabsize=1,
xleftmargin=0cm,
emph={anova,apply,class,coef,colnames,colNames,colSums,dim,dcast,for,ggplot,head,if,ifelse,is.na,lapply,list.files,library,logLik,melt,plot,require,rowSums,sapply,setcolorder,setkey,str,summary,tapply},
aboveskip = \medskipamount, % define the space above displayed listings.
belowskip = \medskipamount, % define the space above displayed listings.
lineskip = 0pt} % specifies additional space between lines in listings
\lstset{style=code-small}
%%%% packages %%%%%

\usepackage[utf8]{inputenc}
\usepackage[T1]{fontenc}
\usepackage{lmodern}
\usepackage{textcomp}
\usepackage{color}
\usepackage{graphicx}
\usepackage{grffile}
\usepackage{wrapfig}
\usepackage{rotating}
\usepackage{longtable}
\usepackage{multirow}
\usepackage{multicol}
\usepackage{changes}
\usepackage{pdflscape}
\usepackage{geometry}
\usepackage[normalem]{ulem}
\usepackage{amssymb}
\usepackage{amsmath}
\usepackage{amsfonts}
\usepackage{dsfont}
\usepackage{array}
\usepackage{ifthen}
\usepackage{hyperref}
\usepackage{natbib}
%
%%%% specifications %%%%
%
\usepackage{ifthen}
\usepackage{xifthen}
\usepackage{xargs}
\usepackage{xspace}
\RequirePackage{fancyvrb}
\DefineVerbatimEnvironment{verbatim}{Verbatim}{fontsize=\small,formatcom = {\color[rgb]{0.5,0,0}}}
\RequirePackage{colortbl} % arrayrulecolor to mix colors
\RequirePackage{setspace} % to modify the space between lines - incompatible with footnote in beamer
\renewcommand{\baselinestretch}{1.1}
\geometry{top=1cm}
\RequirePackage{colortbl} % arrayrulecolor to mix colors
\RequirePackage{pifont}
\RequirePackage{relsize}
\newcommand{\Cross}{{\raisebox{-0.5ex}%
{\relsize{1.5}\ding{56}}}\hspace{1pt} }
\newcommand{\Valid}{{\raisebox{-0.5ex}%
{\relsize{1.5}\ding{52}}}\hspace{1pt} }
\newcommand{\CrossR}{ \textcolor{red}{\Cross} }
\newcommand{\ValidV}{ \textcolor{green}{\Valid} }
\usepackage{stackengine}
\usepackage{scalerel}
\newcommand\Warning[1][3ex]{%
\renewcommand\stacktype{L}%
\scaleto{\stackon[1.3pt]{\color{red}$\triangle$}{\tiny\bfseries !}}{#1}%
\xspace
}
\hypersetup{
citecolor=[rgb]{0,0.5,0},
urlcolor=[rgb]{0,0,0.5},
linkcolor=[rgb]{0,0,0.5},
}
\RequirePackage{epstopdf} % to be able to convert .eps to .pdf image files
\RequirePackage{capt-of} %
\RequirePackage{caption} % newlines in graphics
\RequirePackage{tikz}
\definecolor{grayR}{HTML}{8A8990}
\definecolor{grayL}{HTML}{C4C7C9}
\definecolor{blueM}{HTML}{1F63B5}
\newcommand{\Rlogo}[1][0.07]{
\begin{tikzpicture}[scale=#1]
\shade [right color=grayR,left color=grayL,shading angle=60]
(-3.55,0.3) .. controls (-3.55,1.75)
and (-1.9,2.7) .. (0,2.7) .. controls (2.05,2.7)
and (3.5,1.6) .. (3.5,0.3) .. controls (3.5,-1.2)
and (1.55,-2) .. (0,-2) .. controls (-2.3,-2)
and (-3.55,-0.75) .. cycle;

\fill[white]
(-2.15,0.2) .. controls (-2.15,1.2)
and (-0.7,1.8) .. (0.5,1.8) .. controls (2.2,1.8)
and (3.1,1.2) .. (3.1,0.2) .. controls (3.1,-0.75)
and (2.4,-1.45) .. (0.5,-1.45) .. controls (-1.1,-1.45)
and (-2.15,-0.7) .. cycle;

\fill[blueM]
(1.75,1.25) -- (-0.65,1.25) -- (-0.65,-2.75) -- (0.55,-2.75) -- (0.55,-1.15) --
(0.95,-1.15)  .. controls (1.15,-1.15)
and (1.5,-1.9) .. (1.9,-2.75) -- (3.25,-2.75)  .. controls (2.2,-1)
and (2.5,-1.2) .. (1.8,-0.95) .. controls (2.6,-0.9)
and (2.85,-0.35) .. (2.85,0.2) .. controls (2.85,0.7)
and (2.5,1.2) .. cycle;

\fill[white]  (1.4,0.4) -- (0.55,0.4) -- (0.55,-0.3) -- (1.4,-0.3).. controls (1.75,-0.3)
and (1.75,0.4) .. cycle;

\end{tikzpicture}
}
\RequirePackage{enumitem} % to be able to convert .eps to .pdf image files
\definecolor{light}{rgb}{1, 1, 0.9}
\definecolor{lightred}{rgb}{1.0, 0.7, 0.7}
\definecolor{lightblue}{rgb}{0.0, 0.8, 0.8}
\newcommand{\darkblue}{blue!80!black}
\newcommand{\darkgreen}{green!50!black}
\newcommand{\darkred}{red!50!black}
\usepackage{mdframed}
\newcommand{\first}{1\textsuperscript{st} }
\newcommand{\second}{2\textsuperscript{nd} }
\newcommand{\third}{3\textsuperscript{rd} }
\RequirePackage{amsmath}
\RequirePackage{algorithm}
\RequirePackage[noend]{algpseudocode}
\allowdisplaybreaks
\RequirePackage{dsfont}
\RequirePackage{amsmath,stmaryrd,graphicx}
\RequirePackage{prodint} % product integral symbol (\PRODI)
\newcommand\defOperator[7]{%
\ifthenelse{\isempty{#2}}{
\ifthenelse{\isempty{#1}}{#7{#3}#4}{#7{#3}#4 \left#5 #1 \right#6}
}{
\ifthenelse{\isempty{#1}}{#7{#3}#4_{#2}}{#7{#3}#4_{#1}\left#5 #2 \right#6}
}
}
\newcommand\defUOperator[5]{%
\ifthenelse{\isempty{#1}}{
#5\left#3 #2 \right#4
}{
\ifthenelse{\isempty{#2}}{\underset{#1}{\operatornamewithlimits{#5}}}{
\underset{#1}{\operatornamewithlimits{#5}}\left#3 #2 \right#4}
}
}
\newcommand{\defBoldVar}[2]{
\ifthenelse{\equal{#2}{T}}{\boldsymbol{#1}}{\mathbf{#1}}
}
\newcommandx\Cor[2][1=,2=]{\defOperator{#1}{#2}{C}{or}{\lbrack}{\rbrack}{\mathbb}}
\newcommandx\Cov[2][1=,2=]{\defOperator{#1}{#2}{C}{ov}{\lbrack}{\rbrack}{\mathbb}}
\newcommandx\Esp[2][1=,2=]{\defOperator{#1}{#2}{E}{}{\lbrack}{\rbrack}{\mathbb}}
\newcommandx\Prob[2][1=,2=]{\defOperator{#1}{#2}{P}{}{\lbrack}{\rbrack}{\mathbb}}
\newcommandx\Qrob[2][1=,2=]{\defOperator{#1}{#2}{Q}{}{\lbrack}{\rbrack}{\mathbb}}
\newcommandx\Var[2][1=,2=]{\defOperator{#1}{#2}{V}{ar}{\lbrack}{\rbrack}{\mathbb}}
\newcommandx\Binom[2][1=,2=]{\defOperator{#1}{#2}{B}{}{(}{)}{\mathcal}}
\newcommandx\Gaus[2][1=,2=]{\defOperator{#1}{#2}{N}{}{(}{)}{\mathcal}}
\newcommandx\Wishart[2][1=,2=]{\defOperator{#1}{#2}{W}{ishart}{(}{)}{\mathcal}}
\newcommandx\Likelihood[2][1=,2=]{\defOperator{#1}{#2}{L}{}{(}{)}{\mathcal}}
\newcommandx\Information[2][1=,2=]{\defOperator{#1}{#2}{I}{}{(}{)}{\mathcal}}
\newcommandx\Score[2][1=,2=]{\defOperator{#1}{#2}{S}{}{(}{)}{\mathcal}}
\newcommandx\Vois[2][1=,2=]{\defOperator{#1}{#2}{V}{}{(}{)}{\mathcal}}
\newcommandx\IF[2][1=,2=]{\defOperator{#1}{#2}{IF}{}{(}{)}{\mathcal}}
\newcommandx\Ind[1][1=]{\defOperator{}{#1}{1}{}{(}{)}{\mathds}}
\newcommandx\Max[2][1=,2=]{\defUOperator{#1}{#2}{(}{)}{min}}
\newcommandx\Min[2][1=,2=]{\defUOperator{#1}{#2}{(}{)}{max}}
\newcommandx\argMax[2][1=,2=]{\defUOperator{#1}{#2}{(}{)}{argmax}}
\newcommandx\argMin[2][1=,2=]{\defUOperator{#1}{#2}{(}{)}{argmin}}
\newcommandx\cvD[2][1=D,2=n \rightarrow \infty]{\xrightarrow[#2]{#1}}
\newcommandx\Hypothesis[2][1=,2=]{
\ifthenelse{\isempty{#1}}{
\mathcal{H}
}{
\ifthenelse{\isempty{#2}}{
\mathcal{H}_{#1}
}{
\mathcal{H}^{(#2)}_{#1}
}
}
}
\newcommandx\dpartial[4][1=,2=,3=,4=\partial]{
\ifthenelse{\isempty{#3}}{
\frac{#4 #1}{#4 #2}
}{
\left.\frac{#4 #1}{#4 #2}\right\rvert_{#3}
}
}
\newcommandx\dTpartial[3][1=,2=,3=]{\dpartial[#1][#2][#3][d]}
\newcommandx\ddpartial[3][1=,2=,3=]{
\ifthenelse{\isempty{#3}}{
\frac{\partial^{2} #1}{\partial #2^2}
}{
\frac{\partial^2 #1}{\partial #2\partial #3}
}
}
\newcommand\Ve{\mathbf{e}}
\newcommand\VT{\mathbf{T}}
\newcommand\VY{\mathbf{Y}}
\newcommand\VZ{\mathbf{Z}}
\newcommand\Vvarepsilon{\boldsymbol{\varepsilon}}
\newcommand\Vmu{\boldsymbol{\mu}}
\newcommand\Real{\mathbb{R}}
\newcommand\Rational{\mathbb{Q}}
\newcommand\Natural{\mathbb{N}}
\newcommand\trans[1]{{#1}^\intercal}%\newcommand\trans[1]{{\vphantom{#1}}^\top{#1}}
\newcommand{\independent}{\mathrel{\text{\scalebox{1.5}{$\perp\mkern-10mu\perp$}}}}
\newcommand\half{\frac{1}{2}}
\newcommand\normMax[1]{\left|\left|#1\right|\right|_{max}}
\newcommand\normTwo[1]{\left|\left|#1\right|\right|_{2}}
\author{Brice Ozenne}
\date{\today}
\title{Random intercept model in a balanced design}
\hypersetup{
 colorlinks=true,
 pdfauthor={Brice Ozenne},
 pdftitle={Random intercept model in a balanced design},
 pdfkeywords={},
 pdfsubject={},
 pdfcreator={Emacs 27.2 (Org mode 9.5.2)},
 pdflang={English}
 }
\begin{document}

\maketitle


\section{Notations}
\label{sec:orga2b237a}

Consider an outcome variable \(Y\) measured in \(n\) subjects at \(p\)
occasions. We will index the subjects by \(i \in \{1,\ldots,n\}\) and
the occasions by \(j \in \{1,\ldots,p\}\). During their follow-up each
subject is subject to an active (\(T=1\)) and a control treatment
(\(T=0\)) respectively \(p_1\) and \(p_0\) times. We will use the bold
notation to denote vector of random variables, e.g.
\(\VT_i=\{T_{i,1},\ldots,T_{i,p}\}\).

\bigskip

\noindent As a working model we will consider the following random intercept
model:
\begin{align*}
Y_{i,j} = \alpha + \beta T_{i,j} + u_i + \Vvarepsilon_{i,j}
\end{align*}
where \(u_i \sim \Gaus[0,\tau]\) and \(\varepsilon_{i,j} \sim
\Gaus[0,\delta]\). Introducing \(\rho = \frac{\tau}{\tau+\delta}\) and
\(\sigma^2=\tau+\delta\), we can then express the residual
variance-covariance matrix as:
\begin{align*}
\Var[\VY_{i}|\VT_i] = \Var[u_{i} + \Vvarepsilon_{i}|T_i] = \Omega = \sigma^2 R = \sigma^2 ((1-\rho) I + \rho \Ve\trans{\Ve})
\end{align*}
where \(I\) denotes the \(p \times p\) identity matrix and \(\Ve\) a
corresponding of size \(p\) containing only 1. \(\Theta =
(\alpha,\beta,\delta,\tau)\) or equivalently
\((\alpha,\beta,\rho,\sigma)\) will denote the vector of model
parameters and \(\Vmu_{i}=\left(\alpha+\beta
T_{i1},\ldots,\alpha+\beta T_{ip}\right)\) the vector of fitted
values. Note that since we assume a balanced design and since
\(\Omega\) is unchanged by re-ordering, we can re-order the data such
that \(\VT_i=\VT_{i^{\prime}}=\VT\) for all \((i,i^{\prime})\in\{1,\ldots,n\}^2\).

\clearpage

\section{Estimates in a random intercept model}
\label{sec:orga5a6822}

\subsection{Theory}
\label{sec:orgda95cf0}

Appendix \ref{sm:rhoML} shows that the Maximum Likelihood estimate of \(\Theta\) are:
\begin{itemize}
\item \textbf{mean parameters}: \(\widehat{\alpha}= \frac{1}{np} \sum_{i=1}^n
  \sum_{j=1}^p (1-T_{i,j}) Y_{i,j}\) \newline
\hphantom{\textbf{mean parameters:} } \(\widehat{\beta}=
  \frac{1}{np} \sum_{i=1}^n \sum_{t=1}^p (2 T_{i,j}-1) Y_{i,j}\) \newline
\hphantom{\textbf{mean parameters:} } \(\widehat{\mu}_{i,j} = \widehat{\alpha} - T_{i,j}\widehat{\beta}\)
\item \textbf{variance parameter}: \(\widehat{\sigma}^2 =
  \frac{1}{np}\sum_{i=1}^n\sum_{j=1}^p (Y_{i,j}-\widehat{\mu}_{i,j})^2\)
\item \textbf{correlation parameter}: \(\widehat{\rho} =
  \frac{1}{p(p-1)/2}\sum_{j=1}^p \sum_{j^{\prime} \in
  \{1,\ldots,j-1\}}\widehat{\rho}_{j,j^{\prime}}\) \newline where for
\(j \in \{1,\ldots,p\}\), \(j^{\prime} \in \{1,\ldots,j-1\}\),
\(\widehat{\rho}_{j,j^{\prime}} = \frac{1}{n}\sum_{i=1}^n
  \frac{(Y_{i,j}-\widehat{\mu}_{i,j})}{\widehat{\sigma}}\frac{(Y_{i,j^{\prime}}-\widehat{\mu}_{i,j^{\prime}})}{\widehat{\sigma}}\)
\end{itemize}
i.e. the empirical mean of the outcome, the empirical residual
variance, and the average empirical residual correlation.

\subsection{Numerical example}
\label{sec:org159d14d}

We will illustrate the previous result on an example. First we
simulate some data in the long format:
\lstset{language=r,label= ,caption= ,captionpos=b,numbers=none}
\begin{lstlisting}
library(LMMstar)
library(lava)
library(Matrix)

set.seed(1)
n.obs <- 1e3
n.times <- 4
dfL <- sampleRem(n.obs, n.times = n.times, format = "long",
                 mu = c(1,0,0,0), sigma = 1:4, lambda = c(0.5,0.25,2,1))
dfL <- dfL[order(dfL$id),c("id","visit","Y")]
dfL$treatment <- as.numeric(dfL$visit) %% 2
head(dfL)
\end{lstlisting}

\begin{verbatim}
  id visit          Y treatment
1  1     1  0.2551614         1
2  1     2  0.7913185         0
3  1     3 -2.1031314         1
4  1     4 -0.4489691         0
5  2     1  1.5637433         1
6  2     2 -0.1637081         0
\end{verbatim}



\clearpage

Converting to the wide format facilitate the calculation of the time
specific mean, variance, and correlation:
\lstset{language=r,label= ,caption= ,captionpos=b,numbers=none}
\begin{lstlisting}
dfW <- reshape(dfL[,c("id","visit","Y")],
               direction = "wide", idvar = "id", timevar = "visit")
rbind(mean = colMeans(dfW[,-1]),
      var = apply(dfW[,-1],2,var))
cor(dfW[,-1])
\end{lstlisting}

\begin{verbatim}
          Y.1       Y.2       Y.3       Y.4
mean 1.534321 0.2534847  2.101116  1.040294
var  3.770008 1.6149499 47.740082 12.611689
          Y.1       Y.2       Y.3       Y.4
Y.1 1.0000000 0.5515201 0.8579057 0.8330143
Y.2 0.5515201 1.0000000 0.6468049 0.6131780
Y.3 0.8579057 0.6468049 1.0000000 0.9503735
Y.4 0.8330143 0.6131780 0.9503735 1.0000000
\end{verbatim}

\subsubsection{Maximum likelihood}
\label{sec:org8f52d41}

A random intercept model estimated by Maximum Likelihood (ML) leads to
the following results:
\lstset{language=r,label= ,caption= ,captionpos=b,numbers=none}
\begin{lstlisting}
eML.RI <- lmm(Y~treatment+(1|id), data = dfL, method.fit = "ML")
coef(eML.RI, effects = "all")
\end{lstlisting}

\begin{verbatim}
(Intercept)   treatment       sigma     rho(id) 
  0.6468893   1.1708292   4.0663607   0.5049300
\end{verbatim}


We retrive the empirical means for the intercept and treatment effects:
\lstset{language=r,label= ,caption= ,captionpos=b,numbers=none}
\begin{lstlisting}
alphaHat <- mean(dfL$Y[dfL$treatment == 0])
betaHat <- mean(dfL$Y[dfL$treatment == 1]) - alphaHat
c(alphaHat, betaHat)
\end{lstlisting}

\begin{verbatim}
[1] 0.6468893 1.1708292
\end{verbatim}


the empirical squared residuals for the variance:
\lstset{language=r,label= ,caption= ,captionpos=b,numbers=none}
\begin{lstlisting}
dfL$res <- dfL$Y - alphaHat - dfL$treatment*+betaHat
sqrt(mean(dfL$res^2))
\end{lstlisting}

\begin{verbatim}
[1] 4.066361
\end{verbatim}


\clearpage

and the empirical residual correlation:
\lstset{language=r,label= ,caption= ,captionpos=b,numbers=none}
\begin{lstlisting}
dfL$res.normML <- dfL$res/sqrt(mean(dfL$res^2))
dfWres.normML <- reshape(dfL[,c("id","visit","res.normML")],
                         direction = "wide", idvar = "id", timevar = "visit")
M.MLcor <- crossprod(as.matrix(dfWres.normML[,-1]))/n.obs
mean(M.MLcor[lower.tri(M.MLcor)])
\end{lstlisting}

\begin{verbatim}
[1] 0.50493
\end{verbatim}

\subsubsection{Restricted maximum likelihood}
\label{sec:org011ac45}

When fitting a random intercept model estimated by Maximum Likelihood
(REML):
\lstset{language=r,label= ,caption= ,captionpos=b,numbers=none}
\begin{lstlisting}
eREML.RI <- lmm(Y~treatment+(1|id), data = dfL, method.fit = "REML")
coef(eREML.RI, effects = "all")
\end{lstlisting}

\begin{verbatim}
(Intercept)   treatment       sigma     rho(id) 
  0.6468893   1.1708292   4.0678916   0.5051376
\end{verbatim}


We retrive the empirical means for the intercept and treatment
effects.  However we do not 'exactly' retrieve the REML estimate of the residual
standard deviation using:
\lstset{language=r,label= ,caption= ,captionpos=b,numbers=none}
\begin{lstlisting}
sd(dfL$res)
\end{lstlisting}

\begin{verbatim}
[1] 4.066869
\end{verbatim}


To closer we can get would be using 3 degrees of freedom:
\lstset{language=r,label= ,caption= ,captionpos=b,numbers=none}
\begin{lstlisting}
NROW(dfL)-sum(tapply(dfL$res^2, dfL$visit, sum))/(coef(eREML.RI, effects = "variance"))^2
\end{lstlisting}

\begin{verbatim}
   sigma 
3.010256
\end{verbatim}



We do not 'exactly' retrieve the REML estimate of the residual
correlation using the Pearson correlation:
\lstset{language=r,label= ,caption= ,captionpos=b,numbers=none}
\begin{lstlisting}
dfL$res.normREML <- dfL$res/coef(eREML.RI, effects = "variance")
dfWres.normREML <- reshape(dfL[,c("id","visit","res.normREML")],
                           direction = "wide", idvar = "id", timevar = "visit")
M.REMLcor <- crossprod(as.matrix(dfWres.normREML[,-1]))/(NROW(dfWres.normREML)-1)
mean(M.REMLcor[lower.tri(M.REMLcor)])
\end{lstlisting}

\begin{verbatim}
[1] 0.505055
\end{verbatim}


\clearpage

\section{Standard error in a random intercept model}
\label{sec:org2c1f733}

\subsection{Theory}
\label{sec:org5dc2765}

Appendix \ref{sm:seRI} shows that the standard error of the treatment
effect estimator can be expressed as:
\begin{align*}
\sigma_{\widehat{\beta}} =\sqrt{\frac{\delta}{n} \frac{p}{p_1(p-p_1)}}
\end{align*}

It also shows that in the special case of Maximum Likelihood
estimation with as many observations under treatment as under control
(\(p_1=p/2\)) it simplifies to:
\begin{align}
\sigma_{\widehat{\beta}} = \sqrt{\frac{2\left(1-\frac{1}{p(p-1)/2}\sum_{j < j^{\prime}} \rho_{j,j^{\prime}}\right) \sigma^2}{n p_1}} \label{eq:seML}
\end{align}
where for \(j \in \{1,\ldots,p\}\), \(j^{\prime} \in \{1,\ldots,j-1\}\),
  \(\widehat{\rho}_{j,j^{\prime}} = \frac{1}{n}\sum_{i=1}^n
  \frac{(Y_{i,j}-\widehat{\mu}_{i,j})}{\sigma}\frac{(Y_{i,j^{\prime}}-\mu_{i,j^{\prime}})}{\widehat{\sigma}}\)

\subsection{Comparison to a t-test on the first change}
\label{sec:org3d50852}

When using a t-test on the change based \textbf{only on the first observation
under each treatment}, the variance is:
\begin{align*}
\sigma_{\widehat{\Delta}(1)} = \sqrt{\frac{2(1-\rho_{1,p_1+1}) \sigma^2}{n}}
\end{align*}

where for convenience the first \(p_1\) observations are under one
treatment condition and the last \(p_1\) observations under the other
treatment condition. This strategy controls the type 1 error and is
optimal if observations from the same treatment are (nearly) perfectly
correlated i.e. \(\rho_{j,j^{\prime}} \approx 1\) if
\((j,j^{\prime})\in \{1,\ldots,p_1\}^2\) or if \((j,j^{\prime})\in
\{p_1+1,\ldots,p_1^2\}\). In such a case the cross-correlation must be
(nearly) constant for the correlation matrix to be positive
definite. We thus have:
\begin{align*}
\frac{1}{p(p-1)/2}\sum_{j < j^{\prime}} \rho_{j,j^{\prime}} &= \frac{2(p/2)(p/2-1)/2 \rho_{1,1} +(p/2)^2 \rho_{1,p_1+1}}{p(p-1)/2} \\
&= \frac{(p/2-1)\rho_{1,1}+p/2 \rho_{1,p_1+1}}{p-1} = \frac{p\frac{\rho_{1,p_1+1}+\rho_{1,1}}{2} - \rho_{1,1}}{p-1} \\
&= \rho_{1,1} - \frac{p}{2(p-1)}(\rho_{1,1}-\rho_{1,p_1+1}) \approx 1 - \frac{p}{2(p-1)}(1-\rho_{1,p_1+1})
\end{align*}


\noindent The random intercept model will not control the type 1 error
whenever \(\sigma_{\widehat{\Delta}(1)}>\sigma_{\widehat{\beta}}\):
\begin{align*}
1-\rho_{1,p_1+1} &> \frac{1-\frac{1}{p(p-1)/2}\sum_{j < j^{\prime}} \rho_{j,j^{\prime}}}{p_1} = \frac{1}{p-1}(1-\rho_{1,p_1+1})
\end{align*}
which is always true unless \(p=1\) (no repetition) or \(\rho_{1,p_1+1}\approx 1\) (compound symmetry structure).

\subsection{Comparison to a t-test on changes in a paired design}
\label{sec:orge96a8dd}

Consider now computing \(p_1\) changes per patient, e.g. \(p_1+1\)
vs. \(1\), \(p_1+2\) vs. \(2\), \ldots, using only distinct
observations (no observation is used twice when computing changes),
and stacking all changes into a t-test. This strategy will generally
not control the type 1 error (as it disregard within-individual
correlation). The variance of the corresponding estimator can be
expressed as:
\begin{align*}
\sigma_{\widehat{\Delta}(p_1)} = \sqrt{\frac{2(1-\rho_{1,p_1+1}) \sigma^2}{np_1}}
\end{align*}

Assuming constant within (no necessarily close to one) and constant
cross-correlation we can compare this variance with the mixed model
variance:
\begin{align*}
\sigma_{\widehat{\beta}} - \sigma_{\widehat{\Delta}(p)} = \sqrt{\frac{4\left(1-\rho_{1,1} + \frac{p}{2(p-1)}(\rho_{1,1}-\rho_{1,p_1+1})\right) \sigma^2}{n p}} -  \sqrt{\frac{4(1-\rho_{1,p_1+1}) \sigma^2}{np}}
\end{align*}
which has the same sign as:
\begin{align*}
\left(1-\rho_{1,1} + \frac{p}{2(p-1)}(\rho_{1,1}-\rho_{1,p_1+1})\right) - (1-\rho_{1,p_1+1})
& = (\rho_{1,1}-\rho_{1,p_1+1}) \left(\frac{p}{2(p-1)} - 1 \right) \\
& = (\rho_{1,1}-\rho_{1,p_1+1}) \frac{2-p}{2(p-1)} 
\end{align*}

Therefore if \(\rho_{1,1}>\rho_{1,p_1+1}\) the mixed model will be
less liberal otherwise it will be more liberal.

\subsection{Numerical example}
\label{sec:org20afe35}

We can retrieve the standard error estimated by the linear mixed model:
\lstset{language=r,label= ,caption= ,captionpos=b,numbers=none}
\begin{lstlisting}
model.tables(eML.RI)["treatment",]
\end{lstlisting}

\begin{verbatim}
          estimate         se       df     lower    upper p.value
treatment 1.170829 0.09047721 2999.845 0.9934256 1.348233       0
\end{verbatim}


pluging in formula \autoref{eq:seML} the variance and correlation
estimates based on the residuals:
\lstset{language=r,label= ,caption= ,captionpos=b,numbers=none}
\begin{lstlisting}
sqrt(2*(1-mean(M.MLcor[lower.tri(M.MLcor)]))*mean(dfL$res^2)/(n.obs*n.times/2))
\end{lstlisting}

\begin{verbatim}
[1] 0.09047721
\end{verbatim}




\clearpage

\appendix

\section{Inverse of a compound symmetry matrix}
\label{sm:invCS}
Consider the compound symmetry matrix:
\begin{align*}
R= (1-\rho) I + \rho \Ve\trans{\Ve}= \rho\left(\frac{1-\rho}{\rho} I + \Ve\trans{\Ve}\right) 
\end{align*}
The Sherman-Morrison formula indicates that:
\begin{align*}
R^{-1} &= \rho^{-1} \left(\frac{\rho}{1-\rho} I - \frac{\rho^2}{(1-\rho)^2}\frac{\Ve\trans{\Ve}}{1+\frac{\rho}{1-\rho}\trans{\Ve}\Ve}\right) = \frac{1}{1-\rho} I - \frac{\rho}{(1-\rho)^2}\frac{\Ve\trans{\Ve}}{1+\frac{\rho}{1-\rho}p} \\
&=  \frac{1}{1-\rho} I - \frac{\rho \Ve\trans{\Ve}}{(1-\rho)^2+\rho(1-\rho)p} =  \frac{1}{1-\rho} \left(I - \frac{\rho \Ve\trans{\Ve}}{1+\rho(p-1)}\right)
\end{align*}

\section{Estimates in a random intercept model}
\label{sm:rhoML}
The log-likelihood of a random intercept model can be written:
\begin{align*}
\Likelihood(\Theta|\VY,\VT) =& \sum_{i=1}^{n} \left(-\frac{m}{2} \log(2\pi) - \frac{1}{2} \log\left|\Omega\right| - \frac{1}{2} \trans{(\VY_i-\Vmu_i)} \Omega^{-1} (\VY_i-\Vmu_i) \right)
\end{align*}
and the corresponding restricted likelihood:
\begin{align*}
\Likelihood^R(\Theta|\VY,\VT) = \Likelihood(\Theta|\VY,\VT) + \frac{p}{2} \log(2\pi)-\frac{1}{2} \log\left(\left|\sum_{i=1}^n \trans{\VZ}_i \Omega^{-1} \VZ_i \right|\right)
\end{align*}
where \(\VZ_i = (1,\VT_i)\) is the design matrix w.r.t. subject \(i\).


\subsection{Mean parameters}
\label{sec:orgbad81f4}

The score equation w.r.t. the mean parameters is identical when
considering the log-likelihood or the restricted log-likelihood. Using
the expression of \(R^{-1}\) found in appendix \ref{sm:rhoML} we get:
\begin{align*}
\begin{bmatrix}
0 \\ 0
\end{bmatrix}
=
\begin{bmatrix}
\sum_{i=1}^n \trans{e}\Omega^{-1} (\VY_i-\Vmu_i)) \\
\sum_{i=1}^n \trans{\VT}\Omega^{-1} (\VY_i-\Vmu_i)
\end{bmatrix}
=
\begin{bmatrix}
\frac{1}{\sigma^2(1-\rho)}\sum_{i=1}^n \trans{e}\left(I- \frac{\rho \Ve \trans{\Ve}}{1+\rho(p-1)}\right) (\VY_i-\Vmu_i) \\
\frac{1}{\sigma^2(1-\rho)}\sum_{i=1}^n \trans{\VT}\left(I- \frac{\rho \Ve \trans{\Ve}}{1+\rho(p-1)}\right) (\VY_i-\Vmu_i)
\end{bmatrix}
\end{align*}

which is equivalent to:
\begin{align*}
\begin{bmatrix}
0 \\ 0
\end{bmatrix}
&=
\begin{bmatrix}
\sum_{i=1}^n \left(\trans{e}(\VY_i-\Vmu_i)- \frac{\rho p \trans{\Ve}(\VY_i-\Vmu_i)}{1+\rho(p-1)}\right) \\
\sum_{i=1}^n \left(\trans{\VT}(\VY_i-\Vmu_i)- \frac{\rho p_1 \trans{\Ve}(\VY_i-\Vmu_i)}{1+\rho(p-1)}\right) 
\end{bmatrix} \\ 
& =
\begin{bmatrix}
\left(1 - \frac{\rho p}{1+\rho(p-1)}\right) \sum_{i=1}^n \trans{e}(\VY_i-\Vmu_i) \\
\sum_{i=1}^n \trans{\VT}(\VY_i-\Vmu_i)- \frac{\rho p_1}{1+\rho(p-1)} \sum_{i=1}^n \trans{\Ve}(\VY_i-\Vmu_i)
\end{bmatrix}
\end{align*}
Using that \(1 - \frac{\rho p}{1+\rho(p-1)} = 1 + \rho(p-1) - \rho p =
1 - \rho > 0\) and substracting \(p_1/p\) times equation 1 from equation 2 we get:
\begin{align*}
\begin{bmatrix}
0 \\ 0
\end{bmatrix}
& =
\begin{bmatrix}
\sum_{i=1}^n \trans{e}(\VY_i-\Vmu_i) \\
\sum_{i=1}^n \trans{\VT}(\VY_i-\Vmu_i) - \frac{p_1}{p}\sum_{i=1}^n \trans{\Ve}(\VY_i-\Vmu_i)
\end{bmatrix}
\end{align*}
Denoting the by \(\widehat{\alpha}= \frac{1}{np} \sum_{i=1}^n
\sum_{t=1}^p (1-T_{it}) Y_{it}\) and \(\widehat{\beta}= \frac{1}{np}
\sum_{i=1}^n \sum_{t=1}^p T_{it} Y_{it} - \widehat{\alpha}\) the
empirical mean over timepoints and patients under control and under
treatment. The former equations are equivalent to:
\begin{align*}
\begin{bmatrix}
0 \\ 0
\end{bmatrix}
& =
\begin{bmatrix}
\widehat{\alpha} - \alpha + p_1 (\widehat{\beta} - \beta) \\
p_1 (\widehat{\alpha} + \widehat{\beta} - \alpha - \beta) - \frac{p_1}{p} (\widehat{\alpha} - \alpha + p_1 (\widehat{\beta} - \beta))
\end{bmatrix} \\
\begin{bmatrix}
0 \\ 0
\end{bmatrix} 
& =
\begin{bmatrix}
\widehat{\alpha} - \alpha + (\widehat{\beta} - \beta) \\
(\widehat{\alpha} - \alpha + \widehat{\beta} - \beta ) - \frac{1}{p} (\widehat{\alpha} - \alpha + p_1 (\widehat{\beta} - \beta))
\end{bmatrix} 
\end{align*}
So \(\widehat{\beta} - \beta = -\frac{1}{p_1}(\widehat{\alpha} - \alpha)\) and:
\begin{align*}
0 = (\widehat{\alpha} - \alpha)\left(1-\frac{1}{p_1}-\frac{1}{p}+1) \right)
\end{align*}
Since design \(p_0 \geq 1\) and \(p \geq 2\) so \(2-\frac{1}{p_1}-\frac{1}{p} \geq 0.5\). It
follows that \(\alpha = \widehat{\alpha}\) and therefore
\(\beta=\widehat{\beta}\): the maximum likelihood (ML) and restricted
maximum likelihood (REML) estimates of the mean parameters are the
empirical means in the appropriate sub-groups.

\subsection{Correlation parameter (ML)}
\label{sec:orga8023fd}

The ML score equation w.r.t the correlation parameter is:
\begin{align*}
0 =& -\frac{n}{2} tr\left(\Omega^{-1} \frac{\partial \Omega}{\partial\rho}\right) + \frac{1}{2} \sum_{i=1}^n \trans{(\VY_i-\widehat{\Vmu}_i)} \Omega^{-1} \frac{\partial \Omega}{\partial \rho} \Omega^{-1} (\VY_i-\widehat{\Vmu}_i) \\
  =& -\frac{n}{2} tr\left(R^{-1} \frac{\partial R}{\partial\rho}\right) + \frac{1}{2\sigma^2} tr\left(R^{-1} \frac{\partial R}{\partial \rho} R^{-1} \sum_{i=1}^n (\VY_i-\widehat{\Vmu}_i)\trans{(\VY_i-\widehat{\Vmu}_i)}\right) 
\end{align*}

We first explicit the first term:
\begin{align*}
R^{-1} \frac{\partial R}{\partial\rho} &= \frac{1}{1-\rho} \left(I - \frac{\rho \Ve\trans{\Ve}}{1+\rho(p-1)}\right)\left(-I + \Ve\trans{\Ve}\right) \\
&= \frac{1}{1-\rho} \left(-I + \Ve\trans{\Ve} + \frac{\rho \Ve\trans{\Ve}}{1+\rho(p-1)} - \frac{\rho p \Ve\trans{\Ve}}{1+\rho(p-1)}\right)\\
&= \frac{1}{1-\rho} \left(-I + \Ve\trans{\Ve} \frac{1+\rho(p-1)+\rho-\rho p}{1+\rho(p-1)}\right)\\
&= \frac{1}{1-\rho} \left(-I +  \frac{\Ve\trans{\Ve}}{1+\rho(p-1)}\right)
\end{align*}

Thus:
\begin{align*}
tr \left( R^{-1} \frac{\partial R}{\partial\rho} \right) &= \frac{p}{1-\rho}\left(-1+\frac{1}{1+\rho(p-1)}\right) = -\frac{p\rho(p-1)}{(1-\rho)(1+\rho(p-1))}
\end{align*}

We now consider:
\begin{align*}
R^{-1} \frac{\partial R}{\partial\rho} R^{-1} &= \frac{1}{(1-\rho)^2} \left(-I +  \frac{\Ve\trans{\Ve}}{1+\rho(p-1)}\right)\left(I - \frac{\rho \Ve\trans{\Ve}}{1+\rho(p-1)}\right) \\
&= \frac{1}{(1-\rho)^2} \left(-I + \frac{\rho \Ve\trans{\Ve}}{1+\rho(p-1)} + \frac{\Ve\trans{\Ve}}{1+\rho(p-1)} - \frac{\rho p \Ve\trans{\Ve}}{(1+\rho(p-1))^2}\right) \\
&= \frac{1}{(1-\rho)^2} \left(-I + \Ve\trans{\Ve} \frac{\rho+\rho^2(p-1) + 1+ \rho(p-1) - \rho p}{(1+\rho(p-1))^2}\right) \\
&= \frac{1}{(1-\rho)^2} \left(-I + \Ve\trans{\Ve} \frac{\rho^2(p-1) + 1}{(1+\rho(p-1))^2}\right) 
\end{align*}

We now consider the matrix \(\frac{1}{n}\sum_{i=1}^n (\VY_i-\widehat{\Vmu}_i)\trans{(\VY_i-\widehat{\Vmu}_i)}\) and denote by:
\begin{itemize}
\item \(\left(\widetilde{\sigma}^2_1,\ldots,\widetilde{\sigma}^2_p\right)=\left(\frac{1}{n}\sum_{i=1}^n
  (Y_{i,1}-\widehat{\mu}_{i,1})^2,\ldots,\frac{1}{n}\sum_{i=1}^n
  (Y_{i,p}-\widehat{\mu}_{i,p})^2\right)\) its diagonal elements. The tilde
notation is used instead of the hat notation to stress that they
generally differ from the time-specific empirical variance estimator
(which would center the residuals at each timepoint). Note that
their average equal the empirical residual variance:
\(\widehat{\sigma}^2 = \frac{1}{p} \sum_{j=1}^p
  \widetilde{\sigma}^2_j\).
\item \(\forall (j,j^{\prime})\in \{1,\ldots,p\}\) such that \(j \neq
  j^{\prime}\), we denote the off diagonal elements by
\(\widehat{\sigma}^2\widehat{\rho}_{j,j^{\prime}}\) where
\(\widehat{\rho}_{j,j^{\prime}} = \widehat{\rho}_{j^{\prime},j} =
  \frac{1}{n}\sum_{i=1}^n \frac{Y_{i,j}-\widehat{\mu}_{i,j}}{\widehat{\sigma}}
  \frac{Y_{i,j^{\prime}}-\widehat{\mu}_{i,j^{\prime}}}{\widehat{\sigma}}\) its
off diagonal elements.
\end{itemize}
Then:  
\begin{align*}
& tr \left( R^{-1} \frac{\partial R}{\partial\rho} R^{-1} \sum_{i=1}^n  (\VY_i-\widehat{\Vmu}_i)\trans{(\VY_i-\widehat{\Vmu}_i)} \right) \\
& = \frac{n\widehat{\sigma}^2}{(1-\rho)^2}\left(p\left(-1+\frac{\rho^2(p-1) + 1}{(1+\rho(p-1))^2}\right) + \frac{2\rho^2(p-1) + 2}{(1+\rho(p-1))^2} \sum_{j < j^{\prime}}\widehat{\rho}_{j,j^{\prime}}\right) \\
&= \frac{n\widehat{\sigma}^2}{(1-\rho)^2}\left(p\left(\frac{-2\rho(p-1)-\rho^2(p-1)^2+\rho^2(p-1)}{(1+\rho(p-1))^2}\right) + \frac{2\rho^2(p-1) + 2}{(1+\rho(p-1))^2} \sum_{j < j^{\prime}}\widehat{\rho}_{j,j^{\prime}}\right) \\
&= \frac{n\widehat{\sigma}^2}{(1-\rho)^2(1+\rho(p-1))^2}\left(p\rho(p-1)\left(-2-\rho (p-2)\right) + \left(2\rho^2(p-1) + 2\right) \sum_{j < j^{\prime}}\widehat{\rho}_{j,j^{\prime}}\right)
\end{align*}

The score equation becomes:
\begin{align*}
0 &= \frac{n p (p-1)}{2(1-\rho)^2(1+\rho(p-1))^2}\left(\rho(1-\rho)(1+\rho(p-1)) - \frac{\widehat{\sigma}^2}{\sigma^2}  \rho\left(2+\rho (p-2)\right) + \frac{\rho^2(p-1) + 1}{p(p-1)/2} \sum_{j < j^{\prime}}\widehat{\rho}_{j,j^{\prime}}\right) \\
  &= \frac{n p (p-1)(\rho^2(p-1) + 1)}{2(1-\rho)^2(1+\rho(p-1))^2}\left(\rho\frac{(1-\rho)(1+\rho(p-1))- \frac{\widehat{\sigma}^2}{\sigma^2}  \left(2 + \rho (p-2)\right))}{\rho^2(p-1) + 1} + \frac{1}{p(p-1)/2} \sum_{j < j^{\prime}}\widehat{\rho}_{j,j^{\prime}}\right)
\end{align*}

Using that \((1-\rho)(1+\rho(p-1))=1-\rho+\rho(p-1)-\rho^2(p-1)=-\rho^2(p-1)+\rho(p-2)+1=-(\rho^2(p-1)+1)+\rho(p-2)+2\), it follows that:
\begin{align*}
0 &= \frac{n p (p-1) \rho^2(p-1) + 1}{2(1-\rho)^2(1+\rho(p-1))^2}\left(- \rho + \rho \frac{  \rho(p-2)+2 - \frac{\widehat{\sigma}^2}{\sigma^2}  \left(2 + \rho (p-2)\right))}{\rho^2(p-1) + 1} + \frac{1}{p(p-1)/2} \sum_{j < j^{\prime}}\widehat{\rho}_{j,j^{\prime}}\right) \\
  &= \frac{n p (p-1) \rho^2(p-1) + 1}{2(1-\rho)^2(1+\rho(p-1))^2}\left(- \rho - \rho \left(\frac{\widehat{\sigma}^2}{\sigma^2}-1\right) \frac{2+\rho(p-2)}{1+\rho^2(p-1)} + \frac{1}{p(p-1)/2} \sum_{j < j^{\prime}}\widehat{\rho}_{j,j^{\prime}}\right)
\end{align*}
Since the first term is strictly positive (\(0<\rho<1\) and \(p>1\)) we can simplify and get that:
\begin{align}
\frac{1}{p(p-1)/2}\sum_{j < j^{\prime}}\widehat{\rho}_{j,j^{\prime}}  &= \rho + \rho \left(\frac{\sigma^2}{\widehat{\sigma}^2}-1\right)\frac{2+\rho (p-2)}{1 + \rho^2(p-1)}   \label{eq:scoreRho:simplified2}
\end{align}

\subsection{Variance parameter (ML)}
\label{sec:orgd281494}

The ML score equation w.r.t the variance parameter is:
\begin{align*}
0=&-\frac{n}{2} tr\left(\Omega^{-1} \frac{\partial \Omega}{\partial\sigma^2}\right) + \frac{1}{2} \sum_{i=1}^n \trans{(\VY_i-\widehat{\Vmu}_i)} \Omega^{-1} \frac{\partial \Omega}{\partial \sigma^2} \Omega^{-1} (\VY_i-\widehat{\Vmu}_i) \\
 =&-\frac{n}{2} tr\left(\sigma^{-2} R^{-1} R \right) + \frac{1}{2 \sigma^4} \sum_{i=1}^n \trans{(\VY_i-\widehat{\Vmu}_i)} R^{-1} R R^{-1} (\VY_i-\widehat{\Vmu}_i) \\
 =&-\frac{np}{2 \sigma^2} + \frac{1}{2 \sigma^4} \sum_{i=1}^n \trans{(\VY_i-\widehat{\Vmu}_i)} R^{-1} (\VY_i-\widehat{\Vmu}_i) 
\end{align*}

Using the expression of \(R^{-1}\) found in appendix \ref{sm:rhoML} we get:
\begin{align*}
0 =&-\frac{np}{2 \sigma^2} + \frac{1}{2 \sigma^4(1- \rho)} \sum_{i=1}^n \trans{(\VY_i-\widehat{\Vmu}_i)} \left(I - \frac{\rho \Ve\trans{\Ve}}{(1-\rho)+\rho p} \right) (\VY_i-\widehat{\Vmu}_i) \\ 
  =&-\frac{np}{2 \sigma^2} + \frac{1}{2 \sigma^4(1- \rho)} \sum_{i=1}^n \trans{(\VY_i-\widehat{\Vmu}_i)}(\VY_i-\widehat{\Vmu}_i) \\ & - \frac{\rho}{2 \sigma^4(1- \rho)((1-\rho)+\rho p)} \sum_{i=1}^n \trans{(\VY_i-\widehat{\Vmu}_i)}\Ve\trans{\Ve}(\VY_i-\widehat{\Vmu}_i)  \\ 
  =&-\frac{np}{2 \sigma^2} + \frac{np\widehat{\sigma}^2}{2 \sigma^4(1- \rho)} - \frac{\rho n p^2}{2 \sigma^4(1- \rho)((1-\rho)+\rho p)} \frac{1}{n}\sum_{i=1}^n \left(\frac{1}{p} \sum_{i=1}^n Y_{i,j} - \widehat{\mu}_{i,j} \right)^2 
\end{align*}

Since:
\begin{align*}
\frac{1}{n} \sum_{i=1}^n \left(\frac{1}{p}\sum_{j=1}^p Y_{i,j}-\widehat{\mu}_{i,j}\right)^2=& \frac{1}{np^2} \sum_{i=1}^n \sum_{j=1}^p \sum_{j^{\prime}=1}^p \left(Y_{i,j}-\widehat{\mu}_j\right)\left(Y_{i,j^{\prime}}-\widehat{\mu}_{j^{\prime}}\right) \\
=&  \frac{\widehat{\sigma}^2}{p^2} \left(p + 2\sum_{j < j^{\prime}}\widehat{\rho}_{j,j^{\prime}}\right)  = \frac{\widehat{\sigma}^2}{p} \left(1 + (p-1)\frac{1}{p(p-1)/2}\sum_{j < j^{\prime}}\widehat{\rho}_{j,j^{\prime}}\right) 
\end{align*}

we obtain:
\begin{align*}
0 =&-\frac{np}{2 \sigma^2} + \frac{np\widehat{\sigma}^2}{2 \sigma^4(1- \rho)} - \frac{\rho n p \widehat{\sigma}^2}{2 \sigma^4(1- \rho)((1-\rho)+\rho p)} \left(1 + (p-1)\frac{1}{p(p-1)/2}\sum_{j < j^{\prime}}\widehat{\rho}_{j,j^{\prime}}\right) \\
0 =&-\frac{np \widehat{\sigma}^2}{2 \sigma^4}\left(\frac{\sigma^2}{\widehat{\sigma}^2} - \frac{1}{(1- \rho)} + \frac{\rho }{(1- \rho)((1-\rho)+\rho p)} \left(1 + (p-1)\frac{1}{p(p-1)/2}\sum_{j < j^{\prime}}\widehat{\rho}_{j,j^{\prime}}\right)\right)  
\end{align*}
Since
\begin{align*}
& - \frac{1}{(1- \rho)}  + \frac{\rho}{(1-\rho)^2+\rho(1-\rho)p} \left(1 + (p-1)\frac{1}{p(p-1)/2}\sum_{j < j^{\prime}}\widehat{\rho}_{j,j^{\prime}}\right) \\
=& - \frac{1}{(1- \rho)}\left(1  - \frac{\rho}{1+\rho(p-1)} - \frac{\rho(p-1)}{1+\rho(p-1)}\frac{1}{p(p-1)/2}\sum_{j < j^{\prime}}\widehat{\rho}_{j,j^{\prime}}\right) \\
=&  -\frac{1}{(1- \rho)(1+\rho(p-1))}\left(1 + \rho (p-2) - \rho(p-1)\frac{1}{p(p-1)/2}\sum_{j < j^{\prime}}\widehat{\rho}_{j,j^{\prime}}\right) 
\end{align*}

We get:
\begin{align*}
0 =&-\frac{np \widehat{\sigma}^2}{2 \sigma^4}\left(\frac{\sigma^2}{\widehat{\sigma}^2} -\frac{1}{(1- \rho)(1+\rho(p-1))}\left(1 + \rho (p-2) - \rho(p-1)\frac{1}{p(p-1)/2}\sum_{j < j^{\prime}}\widehat{\rho}_{j,j^{\prime}}\right)\right)   \\
  =&-\frac{np \widehat{\sigma}^2}{2 \sigma^4}\left(\frac{\sigma^2}{\widehat{\sigma}^2} -\frac{1}{1 + \rho(p-2)  - \rho^2(p-1)}\left(1 + \rho (p-2) - \rho(p-1)\frac{1}{p(p-1)/2}\sum_{j < j^{\prime}}\widehat{\rho}_{j,j^{\prime}}\right)\right) \\
  =&-\frac{np \widehat{\sigma}^2}{2 \sigma^4}\left(\frac{\sigma^2}{\widehat{\sigma}^2} - 1  - \frac{\rho(p-1)\left(\rho-\frac{1}{p(p-1)/2}\sum_{j < j^{\prime}}\widehat{\rho}_{j,j^{\prime}}\right)}{1 + \rho(p-2)  - \rho^2(p-1)}\right) 
\end{align*}

Since \(\frac{np \widehat{\sigma}^2}{2 \sigma^4}\neq 0\) and using equation \autoref{eq:scoreRho:simplified2}, we obtain:
\begin{align*}
0 &= \frac{\sigma^2}{\widehat{\sigma}^2} - 1 + \frac{\rho^2(p-1)\left(\frac{\sigma^2}{\widehat{\sigma}^2}-1\right)\frac{2+\rho (p-2)}{1 + \rho^2(p-1)}}{1 + \rho(p-2)  - \rho^2(p-1)} 
= \left(\frac{\sigma^2}{\widehat{\sigma}^2} - 1\right) \left(1 + \frac{\rho^2(p-1)\frac{2+\rho (p-2)}{1 + \rho^2(p-1)}}{(1- \rho)(1+\rho(p-1))}\right) 
\end{align*}
The second term is strictly positive: it is clear when \(p>2\) because
all terms are positive or null and one is added. When \(p=1\) then \(2+\rho
(p-2)=2-\rho>0\) because \(\rho<1\). So we must have \(\sigma^2 =
\widehat{\sigma}^2\). Plugging this value in the score equation for
the correlation parameter leads to:
\begin{align*}
\rho &= \frac{1}{p(p-1)/2}\sum_{j < j^{\prime}}\widehat{\rho}_{j,j^{\prime}} 
\end{align*}

\section{Standard error of the treatment effect \newline in a balanced random intercept model}
\label{sm:seRI}
The standard error of the treatment effect based on the expected
information is the last element (i.e. second row, second column) of
the variance-covariance matrix \(\left(\trans{X} \Omega^{-1}
X\right)^{-1}\).

\bigskip

Using the expression of the inverse of \(R\) given in appendix
\ref{sm:invCS}, we get that its matrix product with the \(p \times 2\)
matrix \(X=(1,T)\) where \(T\) is either \(0\) or \(1\) (respectively
\(p_0\) and \(p_1\) times) is:
\begin{align*}
\trans{X} R^{-1} X &= \frac{1}{1-\rho} \trans{X}X - \frac{\rho\trans{X} \Ve\trans{\Ve} X}{(1-\rho)^2+\rho(1-\rho)p}  \\
&= \frac{1}{1-\rho} \left(\trans{X}X - \frac{\rho\trans{X} \Ve\trans{\Ve} X}{1 + \rho (p-1)}\right)  \\
&= \frac{1}{1-\rho} \left(\begin{bmatrix} p & p_1 \\ p_1 & p_1 \end{bmatrix} - \frac{\rho}{1+\rho(p-1)}  \begin{bmatrix} p^2 & p p_1 \\ p p_1 & p^2_1 \end{bmatrix}\right) \\
&= \frac{1}{(1-\rho)(1+\rho(p-1))} \begin{bmatrix} p+p\rho(p-1) - \rho p^2
                  & p_1+p_1\rho(p-1)- \rho p p_1
                  \\ p_1+p_1\rho(p-1)- \rho p p_1
                  & p_1+p_1\rho(p-1)- \rho p_1^2
\end{bmatrix}   \\
&= \frac{1}{(1-\rho)(1+\rho(p-1))} \begin{bmatrix} p(1-\rho)
                  & p_1(1-\rho)
                  \\ p_1(1-\rho)
                  & p_1(1+\rho (p-p_1-1))
\end{bmatrix}   
\end{align*}

whose inverse is:
\begin{align*}
\left(\trans{X} R^{-1} X\right)^{-1} &= \frac{(1-\rho)(1+\rho(p-1))}{p_1 p (1-\rho)(1+\rho (p-p_1-1)) - p^2_1(1-\rho)^2} \begin{bmatrix} p_1(1+\rho (p-p_1-1))
                  & -p_1(1-\rho)
                  \\ -p_1(1-\rho)
                  & p(1-\rho)
\end{bmatrix} \\
&= \frac{1+\rho(p-1)}{p_1 p (1+\rho (p-p_1-1)) - p^2_1(1-\rho)} \begin{bmatrix} p_1(1+\rho (p-p_1-1))
                  & -p_1(1-\rho)
                  \\ -p_1(1-\rho)
                  & p(1-\rho)
\end{bmatrix} \\
&= \frac{1+\rho(p-1)}{(p - p_1) + \rho (p^2-p p_1-p+p_1)} \begin{bmatrix} 1+\rho (p-p_1-1)
                  & -(1-\rho)
                  \\ -(1-\rho)
                  & \frac{p}{p_1}(1-\rho)
\end{bmatrix} \\
&= \frac{1}{p-p_1} \begin{bmatrix} 1+\rho (p-p_1-1)
                  & -(1-\rho)
                  \\ -(1-\rho)
                  & \frac{p}{p_1}(1-\rho)
\end{bmatrix}   
\end{align*}

So in the random intercept model, the standard error of the treatment
estimator will be:
\begin{align*}
\sigma_{\widehat{\beta}} = \sqrt{\sigma_0^2(1-\rho) \frac{p}{n p_1(p-p_1)}}=\sqrt{\frac{\delta}{n} \frac{p}{p_1(p-p_1)}}
\end{align*}

In a design with as many observations under treatment as under control \(p_1=p/2\) and the expression simplifies into.
\begin{align*}
\sigma_{\widehat{\beta}} = \sqrt{\frac{4\delta}{np}} = \sqrt{\frac{2\delta}{np_1}}
\end{align*}

From appendix \ref{sm:rhoML} we deduce that in a balanced design the
standard error of the Maximum Likelihood estimator is:
\begin{align*}
\sigma_{\widehat{\beta}} = \sqrt{\frac{\left(1-\frac{1}{p(p-1)/2}\sum_{j < j^{\prime}} \rho_{j,j^{\prime}}\right) \sigma^2}{n}\frac{p}{p_1(p-p_1)}}
\end{align*}
which in a design with as many observations under treatment as under control simplifies to:
\begin{align*}
\sigma_{\widehat{\beta}} = \sqrt{\frac{2\left(1-\frac{1}{p(p-1)/2}\sum_{j < j^{\prime}} \rho_{j,j^{\prime}}\right) \sigma^2}{n p_1}}
\end{align*}


\clearpage
\end{document}