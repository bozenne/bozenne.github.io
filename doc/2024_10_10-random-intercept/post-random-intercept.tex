% Created 2024-10-15 ti 19:48
% Intended LaTeX compiler: pdflatex
\documentclass[12pt]{article}

%%%% settings when exporting code %%%% 

\usepackage{listings}
\lstdefinestyle{code-small}{
backgroundcolor=\color{white}, % background color for the code block
basicstyle=\ttfamily\small, % font used to display the code
commentstyle=\color[rgb]{0.5,0,0.5}, % color used to display comments in the code
keywordstyle=\color{black}, % color used to highlight certain words in the code
numberstyle=\ttfamily\tiny\color{gray}, % color used to display the line numbers
rulecolor=\color{black}, % color of the frame
stringstyle=\color[rgb]{0,.5,0},  % color used to display strings in the code
breakatwhitespace=false, % sets if automatic breaks should only happen at whitespace
breaklines=true, % sets automatic line breaking
columns=fullflexible,
frame=single, % adds a frame around the code (non,leftline,topline,bottomline,lines,single,shadowbox)
keepspaces=true, % % keeps spaces in text, useful for keeping indentation of code
literate={~}{$\sim$}{1}, % symbol properly display via latex
numbers=none, % where to put the line-numbers; possible values are (none, left, right)
numbersep=10pt, % how far the line-numbers are from the code
showspaces=false,
showstringspaces=false,
stepnumber=1, % the step between two line-numbers. If it's 1, each line will be numbered
tabsize=1,
xleftmargin=0cm,
emph={anova,apply,class,coef,colnames,colNames,colSums,dim,dcast,for,ggplot,head,if,ifelse,is.na,lapply,list.files,library,logLik,melt,plot,require,rowSums,sapply,setcolorder,setkey,str,summary,tapply},
aboveskip = \medskipamount, % define the space above displayed listings.
belowskip = \medskipamount, % define the space above displayed listings.
lineskip = 0pt} % specifies additional space between lines in listings
\lstset{style=code-small}
%%%% packages %%%%%

\usepackage[utf8]{inputenc}
\usepackage[T1]{fontenc}
\usepackage{lmodern}
\usepackage{textcomp}
\usepackage{color}
\usepackage{graphicx}
\usepackage{grffile}
\usepackage{wrapfig}
\usepackage{rotating}
\usepackage{longtable}
\usepackage{multirow}
\usepackage{multicol}
\usepackage{changes}
\usepackage{pdflscape}
\usepackage{geometry}
\usepackage[normalem]{ulem}
\usepackage{amssymb}
\usepackage{amsmath}
\usepackage{amsfonts}
\usepackage{dsfont}
\usepackage{array}
\usepackage{ifthen}
\usepackage{hyperref}
\usepackage{natbib}
%
%%%% specifications %%%%
%
\usepackage{ifthen}
\usepackage{xifthen}
\usepackage{xargs}
\usepackage{xspace}
\RequirePackage{fancyvrb}
\DefineVerbatimEnvironment{verbatim}{Verbatim}{fontsize=\small,formatcom = {\color[rgb]{0.5,0,0}}}
\RequirePackage{colortbl} % arrayrulecolor to mix colors
\RequirePackage{setspace} % to modify the space between lines - incompatible with footnote in beamer
\renewcommand{\baselinestretch}{1.1}
\geometry{top=1cm}
\RequirePackage{colortbl} % arrayrulecolor to mix colors
\RequirePackage{pifont}
\RequirePackage{relsize}
\newcommand{\Cross}{{\raisebox{-0.5ex}%
{\relsize{1.5}\ding{56}}}\hspace{1pt} }
\newcommand{\Valid}{{\raisebox{-0.5ex}%
{\relsize{1.5}\ding{52}}}\hspace{1pt} }
\newcommand{\CrossR}{ \textcolor{red}{\Cross} }
\newcommand{\ValidV}{ \textcolor{green}{\Valid} }
\usepackage{stackengine}
\usepackage{scalerel}
\newcommand\Warning[1][3ex]{%
\renewcommand\stacktype{L}%
\scaleto{\stackon[1.3pt]{\color{red}$\triangle$}{\tiny\bfseries !}}{#1}%
\xspace
}
\hypersetup{
citecolor=[rgb]{0,0.5,0},
urlcolor=[rgb]{0,0,0.5},
linkcolor=[rgb]{0,0,0.5},
}
\RequirePackage{epstopdf} % to be able to convert .eps to .pdf image files
\RequirePackage{capt-of} %
\RequirePackage{caption} % newlines in graphics
\RequirePackage{tikz}
\definecolor{grayR}{HTML}{8A8990}
\definecolor{grayL}{HTML}{C4C7C9}
\definecolor{blueM}{HTML}{1F63B5}
\newcommand{\Rlogo}[1][0.07]{
\begin{tikzpicture}[scale=#1]
\shade [right color=grayR,left color=grayL,shading angle=60]
(-3.55,0.3) .. controls (-3.55,1.75)
and (-1.9,2.7) .. (0,2.7) .. controls (2.05,2.7)
and (3.5,1.6) .. (3.5,0.3) .. controls (3.5,-1.2)
and (1.55,-2) .. (0,-2) .. controls (-2.3,-2)
and (-3.55,-0.75) .. cycle;

\fill[white]
(-2.15,0.2) .. controls (-2.15,1.2)
and (-0.7,1.8) .. (0.5,1.8) .. controls (2.2,1.8)
and (3.1,1.2) .. (3.1,0.2) .. controls (3.1,-0.75)
and (2.4,-1.45) .. (0.5,-1.45) .. controls (-1.1,-1.45)
and (-2.15,-0.7) .. cycle;

\fill[blueM]
(1.75,1.25) -- (-0.65,1.25) -- (-0.65,-2.75) -- (0.55,-2.75) -- (0.55,-1.15) --
(0.95,-1.15)  .. controls (1.15,-1.15)
and (1.5,-1.9) .. (1.9,-2.75) -- (3.25,-2.75)  .. controls (2.2,-1)
and (2.5,-1.2) .. (1.8,-0.95) .. controls (2.6,-0.9)
and (2.85,-0.35) .. (2.85,0.2) .. controls (2.85,0.7)
and (2.5,1.2) .. cycle;

\fill[white]  (1.4,0.4) -- (0.55,0.4) -- (0.55,-0.3) -- (1.4,-0.3).. controls (1.75,-0.3)
and (1.75,0.4) .. cycle;

\end{tikzpicture}
}
\RequirePackage{enumitem} % to be able to convert .eps to .pdf image files
\definecolor{light}{rgb}{1, 1, 0.9}
\definecolor{lightred}{rgb}{1.0, 0.7, 0.7}
\definecolor{lightblue}{rgb}{0.0, 0.8, 0.8}
\newcommand{\darkblue}{blue!80!black}
\newcommand{\darkgreen}{green!50!black}
\newcommand{\darkred}{red!50!black}
\usepackage{mdframed}
\newcommand{\first}{1\textsuperscript{st} }
\newcommand{\second}{2\textsuperscript{nd} }
\newcommand{\third}{3\textsuperscript{rd} }
\RequirePackage{amsmath}
\RequirePackage{algorithm}
\RequirePackage[noend]{algpseudocode}
\allowdisplaybreaks
\RequirePackage{dsfont}
\RequirePackage{amsmath,stmaryrd,graphicx}
\RequirePackage{prodint} % product integral symbol (\PRODI)
\newcommand\defOperator[7]{%
\ifthenelse{\isempty{#2}}{
\ifthenelse{\isempty{#1}}{#7{#3}#4}{#7{#3}#4 \left#5 #1 \right#6}
}{
\ifthenelse{\isempty{#1}}{#7{#3}#4_{#2}}{#7{#3}#4_{#1}\left#5 #2 \right#6}
}
}
\newcommand\defUOperator[5]{%
\ifthenelse{\isempty{#1}}{
#5\left#3 #2 \right#4
}{
\ifthenelse{\isempty{#2}}{\underset{#1}{\operatornamewithlimits{#5}}}{
\underset{#1}{\operatornamewithlimits{#5}}\left#3 #2 \right#4}
}
}
\newcommand{\defBoldVar}[2]{
\ifthenelse{\equal{#2}{T}}{\boldsymbol{#1}}{\mathbf{#1}}
}
\newcommandx\Cor[2][1=,2=]{\defOperator{#1}{#2}{C}{or}{\lbrack}{\rbrack}{\mathbb}}
\newcommandx\Cov[2][1=,2=]{\defOperator{#1}{#2}{C}{ov}{\lbrack}{\rbrack}{\mathbb}}
\newcommandx\Esp[2][1=,2=]{\defOperator{#1}{#2}{E}{}{\lbrack}{\rbrack}{\mathbb}}
\newcommandx\Prob[2][1=,2=]{\defOperator{#1}{#2}{P}{}{\lbrack}{\rbrack}{\mathbb}}
\newcommandx\Qrob[2][1=,2=]{\defOperator{#1}{#2}{Q}{}{\lbrack}{\rbrack}{\mathbb}}
\newcommandx\Var[2][1=,2=]{\defOperator{#1}{#2}{V}{ar}{\lbrack}{\rbrack}{\mathbb}}
\newcommandx\Binom[2][1=,2=]{\defOperator{#1}{#2}{B}{}{(}{)}{\mathcal}}
\newcommandx\Gaus[2][1=,2=]{\defOperator{#1}{#2}{N}{}{(}{)}{\mathcal}}
\newcommandx\Wishart[2][1=,2=]{\defOperator{#1}{#2}{W}{ishart}{(}{)}{\mathcal}}
\newcommandx\Likelihood[2][1=,2=]{\defOperator{#1}{#2}{L}{}{(}{)}{\mathcal}}
\newcommandx\Information[2][1=,2=]{\defOperator{#1}{#2}{I}{}{(}{)}{\mathcal}}
\newcommandx\Score[2][1=,2=]{\defOperator{#1}{#2}{S}{}{(}{)}{\mathcal}}
\newcommandx\Vois[2][1=,2=]{\defOperator{#1}{#2}{V}{}{(}{)}{\mathcal}}
\newcommandx\IF[2][1=,2=]{\defOperator{#1}{#2}{IF}{}{(}{)}{\mathcal}}
\newcommandx\Ind[1][1=]{\defOperator{}{#1}{1}{}{(}{)}{\mathds}}
\newcommandx\Max[2][1=,2=]{\defUOperator{#1}{#2}{(}{)}{min}}
\newcommandx\Min[2][1=,2=]{\defUOperator{#1}{#2}{(}{)}{max}}
\newcommandx\argMax[2][1=,2=]{\defUOperator{#1}{#2}{(}{)}{argmax}}
\newcommandx\argMin[2][1=,2=]{\defUOperator{#1}{#2}{(}{)}{argmin}}
\newcommandx\cvD[2][1=D,2=n \rightarrow \infty]{\xrightarrow[#2]{#1}}
\newcommandx\Hypothesis[2][1=,2=]{
\ifthenelse{\isempty{#1}}{
\mathcal{H}
}{
\ifthenelse{\isempty{#2}}{
\mathcal{H}_{#1}
}{
\mathcal{H}^{(#2)}_{#1}
}
}
}
\newcommandx\dpartial[4][1=,2=,3=,4=\partial]{
\ifthenelse{\isempty{#3}}{
\frac{#4 #1}{#4 #2}
}{
\left.\frac{#4 #1}{#4 #2}\right\rvert_{#3}
}
}
\newcommandx\dTpartial[3][1=,2=,3=]{\dpartial[#1][#2][#3][d]}
\newcommandx\ddpartial[3][1=,2=,3=]{
\ifthenelse{\isempty{#3}}{
\frac{\partial^{2} #1}{\partial #2^2}
}{
\frac{\partial^2 #1}{\partial #2\partial #3}
}
}
\newcommand\Ve{\mathbf{e}}
\newcommand\VT{\mathbf{T}}
\newcommand\VY{\mathbf{Y}}
\newcommand\VZ{\mathbf{Z}}
\newcommand\Vvarepsilon{\boldsymbol{\varepsilon}}
\newcommand\Vmu{\boldsymbol{\mu}}
\newcommand\Real{\mathbb{R}}
\newcommand\Rational{\mathbb{Q}}
\newcommand\Natural{\mathbb{N}}
\newcommand\trans[1]{{#1}^\intercal}%\newcommand\trans[1]{{\vphantom{#1}}^\top{#1}}
\newcommand{\independent}{\mathrel{\text{\scalebox{1.5}{$\perp\mkern-10mu\perp$}}}}
\newcommand\half{\frac{1}{2}}
\newcommand\normMax[1]{\left|\left|#1\right|\right|_{max}}
\newcommand\normTwo[1]{\left|\left|#1\right|\right|_{2}}
\author{Brice Ozenne}
\date{\today}
\title{Random intercept model with a balanced design}
\hypersetup{
 colorlinks=true,
 pdfauthor={Brice Ozenne},
 pdftitle={Random intercept model with a balanced design},
 pdfkeywords={},
 pdfsubject={},
 pdfcreator={Emacs 27.2 (Org mode 9.5.2)},
 pdflang={English}
 }
\begin{document}

\maketitle


\section{Notations}
\label{sec:org8cde870}

Consider an outcome variable \(Y\) measured in \(n\) subjects at \(p\)
occasions. We will index the subjects by \(i \in \{1,\ldots,n\}\) and
the occasions by \(j \in \{1,\ldots,p\}\). During their follow-up each
subject is subject to an active (\(T=1\)) and a control treatment
(\(T=0\)) respectively \(p_1\) and \(p_0\) times. We will use the bold
notation to denote vector of random variables, e.g.
\(\VT_i=\{T_{i,1},\ldots,T_{i,p}\}\).

\bigskip

The (data-generating) variance of the outcome will be denoted
\(\sigma^2\). The (data-generating) correlation between any two
measurements from a subject will be denoted \(r_{j,j^{\prime}} = \Cor(Y_{i,j},Y_{i,j^{\prime}}|T_{i,j},T_{i,j^{\prime}})\)

\bigskip

As a working model we will consider the following random intercept
model:
\begin{align*}
Y_{i,j} = \alpha + \beta T_{i,j} + u_i + \Vvarepsilon_{i,j}
\end{align*}
where \(u_i \sim \Gaus[0,\tau]\) and \(\varepsilon_{i,j} \sim
\Gaus[0,\delta]\). Introducing \(\rho = \frac{\tau}{\tau+\delta}\) and
\(\sigma^2=\tau+\delta\), we can then express the residual
variance-covariance matrix as:
\begin{align*}
\Var[\VY_{i}|\VT_i] = \Var[u_{i} + \Vvarepsilon_{i}|T_i] = \Omega = \sigma^2 R = \sigma^2 ((1-\rho) I + \rho \Ve\trans{\Ve})
\end{align*}
where \(I\) denotes the \(p \times p\) identity matrix and \(\Ve\) a
corresponding of size \(p\) containing only 1. \(\Theta =
(\alpha,\beta,\delta,\tau)\) or equivalently
\((\alpha,\beta,\rho,\sigma)\) will denote the vector of model
parameters and \(\Vmu_{i}=\left(\alpha+\beta
T_{i1},\ldots,\alpha+\beta T_{ip}\right)\) the vector of fitted
values. Note that since we assume a balanced design and since
\(\Omega\) is unchanged by re-ordering, we can re-order the data such
that \(\VT_i=\VT_{i^{\prime}}=\VT\) for all \((i,i^{\prime})\in\{1,\ldots,n\}^2\).

\clearpage

\section{Estimates in a random intercept model}
\label{sec:org2bc3643}

\subsection{Theory (single random effect)}
\label{sec:orgfdbb73e}

Appendix \ref{sm:rhoML} shows that the Maximum Likelihood estimate of \(\Theta\) are:
\begin{itemize}
\item \textbf{mean parameters}: \(\widehat{\alpha}= \frac{1}{np} \sum_{i=1}^n
  \sum_{t=1}^p (1-T_{it}) Y_{it}\) \newline
\hphantom{\textbf{mean parameters:} } \(\widehat{\beta}=
  \frac{1}{np} \sum_{i=1}^n \sum_{t=1}^p (2 T_{it}-1) Y_{it}\)
\item \textbf{variance parameter}: \(\widehat{\sigma}^2 =
  \frac{1}{np}\sum_{i=1}^n\sum_{j=1}^p (Y_{i,j}-\mu_{i,j})^2\)
\item \textbf{correlation parameter}: \(\widehat{\rho} =
  \frac{1}{p(p-1)/2}\sum_{j=1}^p \sum_{j^{\prime} \in
  \{1,\ldots,j-1\}}\widehat{\rho}_{j,j^{\prime}}\) \newline where for
\(j \in \{1,\ldots,p\}\), \(j^{\prime} \in \{1,\ldots,j-1\}\),
\(\widehat{\rho}_{j,j^{\prime}} = \frac{1}{n}\sum_{i=1}^n
  \frac{(Y_{i,j}-\mu_{i,j})}{\widehat{\sigma}}\frac{(Y_{i,j^{\prime}}-\mu_{i,j^{\prime}})}{\widehat{\sigma}}\)
\end{itemize}
i.e. the empirical mean of the outcome, the empirical residual
variance, and the average empirical residual correlation.

\subsection{Numerical example}
\label{sec:orgb208955}

We will illustrate the previous result on an example. First we
simulate some data in the long format:
\lstset{language=r,label= ,caption= ,captionpos=b,numbers=none}
\begin{lstlisting}
library(LMMstar)
library(lava)
library(Matrix)

set.seed(1)
dfL <- sampleRem(1e3, n.times = 4, format = "long",
                 mu = c(1,0,0,0), sigma = 1:4, lambda = c(0.5,0.25,2,1))
dfL <- dfL[order(dfL$id),c("id","visit","Y")]
dfL$treatment <- as.numeric(dfL$visit) %% 2
head(dfL)
\end{lstlisting}

\begin{verbatim}
  id visit          Y treatment
1  1     1  0.2551614         1
2  1     2  0.7913185         0
3  1     3 -2.1031314         1
4  1     4 -0.4489691         0
5  2     1  1.5637433         1
6  2     2 -0.1637081         0
\end{verbatim}



\clearpage

Converting to the wide format facilitate the calculation of the time
specific mean, variance, and correlation:
\lstset{language=r,label= ,caption= ,captionpos=b,numbers=none}
\begin{lstlisting}
dfW <- reshape(dfL[,c("id","visit","Y")],
               direction = "wide", idvar = "id", timevar = "visit")
rbind(mean = colMeans(dfW[,-1]),
      var = apply(dfW[,-1],2,var))
cor(dfW[,-1])
\end{lstlisting}

\begin{verbatim}
          Y.1       Y.2       Y.3       Y.4
mean 1.534321 0.2534847  2.101116  1.040294
var  3.770008 1.6149499 47.740082 12.611689
          Y.1       Y.2       Y.3       Y.4
Y.1 1.0000000 0.5515201 0.8579057 0.8330143
Y.2 0.5515201 1.0000000 0.6468049 0.6131780
Y.3 0.8579057 0.6468049 1.0000000 0.9503735
Y.4 0.8330143 0.6131780 0.9503735 1.0000000
\end{verbatim}

\subsubsection{Maximum likelihood (single random effect)}
\label{sec:org5f99ea6}

A random intercept model estimated by Maximum Likelihood (ML) leads to
the following results:
\lstset{language=r,label= ,caption= ,captionpos=b,numbers=none}
\begin{lstlisting}
eML.RI <- lmm(Y~treatment+(1|id), data = dfL, method.fit = "ML")
coef(eML.RI, effects = "all")
\end{lstlisting}

\begin{verbatim}
(Intercept)   treatment       sigma     rho(id) 
  0.6468893   1.1708292   4.0663607   0.5049300
\end{verbatim}


We retrive the empirical means for the intercept and treatment effects:
\lstset{language=r,label= ,caption= ,captionpos=b,numbers=none}
\begin{lstlisting}
alphaHat <- mean(dfL$Y[dfL$treatment == 0])
betaHat <- mean(dfL$Y[dfL$treatment == 1]) - alphaHat
c(alphaHat, betaHat)
\end{lstlisting}

\begin{verbatim}
[1] 0.6468893 1.1708292
\end{verbatim}


the empirical squared residuals for the variance:
\lstset{language=r,label= ,caption= ,captionpos=b,numbers=none}
\begin{lstlisting}
dfL$res <- dfL$Y - alphaHat - dfL$treatment*+betaHat
sqrt(mean(dfL$res^2))
\end{lstlisting}

\begin{verbatim}
[1] 4.066361
\end{verbatim}


\clearpage

and the empirical residual correlation:
\lstset{language=r,label= ,caption= ,captionpos=b,numbers=none}
\begin{lstlisting}
dfL$res.normML <- dfL$res/sqrt(mean(dfL$res^2))
dfWres.normML <- reshape(dfL[,c("id","visit","res.normML")],
                         direction = "wide", idvar = "id", timevar = "visit")
M.MLcor <- crossprod(as.matrix(dfWres.normML[,-1]))/NROW(dfWres.normML)
mean(M.MLcor[lower.tri(M.MLcor)])
\end{lstlisting}

\begin{verbatim}
[1] 0.50493
\end{verbatim}

\subsubsection{Restricted maximum likelihood (single random effect)}
\label{sec:org0bbf0fd}

When fitting a random intercept model estimated by Maximum Likelihood
(REML):
\lstset{language=r,label= ,caption= ,captionpos=b,numbers=none}
\begin{lstlisting}
eREML.RI <- lmm(Y~treatment+(1|id), data = dfL, method.fit = "REML")
coef(eREML.RI, effects = "all")
\end{lstlisting}

\begin{verbatim}
(Intercept)   treatment       sigma     rho(id) 
  0.6468893   1.1708292   4.0678916   0.5051376
\end{verbatim}


We retrive the empirical means for the intercept and treatment
effects.  However we do not 'exactly' retrieve the REML estimate of the residual
standard deviation using:
\lstset{language=r,label= ,caption= ,captionpos=b,numbers=none}
\begin{lstlisting}
sd(dfL$res)
\end{lstlisting}

\begin{verbatim}
[1] 4.066869
\end{verbatim}


To get closer, we could use 3 degrees of freedom (even though 2 would
be better match the intuition):
\lstset{language=r,label= ,caption= ,captionpos=b,numbers=none}
\begin{lstlisting}
NROW(dfL)-sum(tapply(dfL$res^2, dfL$visit, sum))/(coef(eREML.RI, effects = "variance"))^2
\end{lstlisting}

\begin{verbatim}
   sigma 
3.010256
\end{verbatim}



We do not 'exactly' retrieve the REML estimate of the residual
correlation using the Pearson correlation:
\lstset{language=r,label= ,caption= ,captionpos=b,numbers=none}
\begin{lstlisting}
dfL$res.normREML <- dfL$res/coef(eREML.RI, effects = "variance")
dfWres.normREML <- reshape(dfL[,c("id","visit","res.normREML")],
                           direction = "wide", idvar = "id", timevar = "visit")
M.REMLcor <- crossprod(as.matrix(dfWres.normREML[,-1]))/(NROW(dfWres.normREML)-1)
mean(M.REMLcor[lower.tri(M.REMLcor)])
\end{lstlisting}

\begin{verbatim}
[1] 0.505055
\end{verbatim}

\subsubsection{Maximum likelihood (nested random effects)}
\label{sec:org8c5efc6}

\lstset{language=r,label= ,caption= ,captionpos=b,numbers=none}
\begin{lstlisting}
dfL$baseline <- dfL$visit %in% 1:2
eML.RI2 <- lmm(Y~treatment+(1|id/baseline), repetition = ~visit|id, data = dfL, method.fit = "ML")
coef(eML.RI2, effects = "all")
\end{lstlisting}

\begin{verbatim}
(Intercept)        treatment            sigma rho(id/baseline) 
  0.6468893        1.1708292        4.0663607        0.7522951 
    rho(id) 
  0.3812474
\end{verbatim}


We retrive again the same estimates for the mean and variance
parameters as in the random intercept model with a single random
effect. The ML estimate of the correlation also match the average
empirical residual correlation at the appropriate pairs of timepoints:
\lstset{language=r,label= ,caption= ,captionpos=b,numbers=none}
\begin{lstlisting}
c(mean(c(M.MLcor[1,2],M.MLcor[3,4])), mean(M.MLcor[1:2,3:4]))
\end{lstlisting}

\begin{verbatim}
[1] 0.7522951 0.3812474
\end{verbatim}



\clearpage

\appendix

\section{Inverse of a compound symmetry matrix}
\label{sm:invCS}
Consider the compound symmetry matrix:
\begin{align*}
R= (1-\rho) I + \rho \Ve\trans{\Ve}= \rho\left(\frac{1-\rho}{\rho} I + \Ve\trans{\Ve}\right) 
\end{align*}
The Sherman-Morrison formula indicates that:
\begin{align*}
R^{-1} &= \rho^{-1} \left(\frac{\rho}{1-\rho} I - \frac{\rho^2}{(1-\rho)^2}\frac{\Ve\trans{\Ve}}{1+\frac{\rho}{1-\rho}\trans{\Ve}\Ve}\right) = \frac{1}{1-\rho} I - \frac{\rho}{(1-\rho)^2}\frac{\Ve\trans{\Ve}}{1+\frac{\rho}{1-\rho}p} \\
&=  \frac{1}{1-\rho} I - \frac{\rho \Ve\trans{\Ve}}{(1-\rho)^2+\rho(1-\rho)p} =  \frac{1}{1-\rho} \left(I - \frac{\rho \Ve\trans{\Ve}}{1+\rho(p-1)}\right)
\end{align*}

\section{Estimates in a random intercept model}
\label{sm:rhoML}
The log-likelihood of a random intercept model can be written:
\begin{align*}
\Likelihood(\Theta|\VY,\VT) =& \sum_{i=1}^{n} \left(-\frac{m}{2} \log(2\pi) - \frac{1}{2} \log\left|\Omega\right| - \frac{1}{2} \trans{(\VY_i-\Vmu_i)} \Omega^{-1} (\VY_i-\Vmu_i) \right)
\end{align*}
and the corresponding restricted likelihood:
\begin{align*}
\Likelihood^R(\Theta|\VY,\VT) = \Likelihood(\Theta|\VY,\VT) + \frac{p}{2} \log(2\pi)-\frac{1}{2} \log\left(\left|\sum_{i=1}^n \trans{\VZ}_i \Omega^{-1} \VZ_i \right|\right)
\end{align*}
where \(\VZ_i = (1,\VT_i)\) is the design matrix w.r.t. subject \(i\).


\subsection{Mean parameters}
\label{sec:org37a9caf}

The score equation w.r.t. the mean parameters is identical when
considering the log-likelihood or the restricted log-likelihood. Using
the expression of \(R^{-1}\) found in appendix \ref{sm:rhoML} we get:
\begin{align*}
\begin{bmatrix}
0 \\ 0
\end{bmatrix}
=
\begin{bmatrix}
\sum_{i=1}^n \trans{e}\Omega^{-1} (\VY_i-\Vmu_i)) \\
\sum_{i=1}^n \trans{\VT}\Omega^{-1} (\VY_i-\Vmu_i)
\end{bmatrix}
=
\begin{bmatrix}
\frac{1}{\sigma^2(1-\rho)}\sum_{i=1}^n \trans{e}\left(I- \frac{\rho \Ve \trans{\Ve}}{1+\rho(p-1)}\right) (\VY_i-\Vmu_i) \\
\frac{1}{\sigma^2(1-\rho)}\sum_{i=1}^n \trans{\VT}\left(I- \frac{\rho \Ve \trans{\Ve}}{1+\rho(p-1)}\right) (\VY_i-\Vmu_i)
\end{bmatrix}
\end{align*}

which is equivalent to:
\begin{align*}
\begin{bmatrix}
0 \\ 0
\end{bmatrix}
&=
\begin{bmatrix}
\sum_{i=1}^n \left(\trans{e}(\VY_i-\Vmu_i)- \frac{\rho p \trans{\Ve}(\VY_i-\Vmu_i)}{1+\rho(p-1)}\right) \\
\sum_{i=1}^n \left(\trans{\VT}(\VY_i-\Vmu_i)- \frac{\rho p_1 \trans{\Ve}(\VY_i-\Vmu_i)}{1+\rho(p-1)}\right) 
\end{bmatrix} \\ 
& =
\begin{bmatrix}
\left(1 - \frac{\rho p}{1+\rho(p-1)}\right) \sum_{i=1}^n \trans{e}(\VY_i-\Vmu_i) \\
\sum_{i=1}^n \trans{\VT}(\VY_i-\Vmu_i)- \frac{\rho p_1}{1+\rho(p-1)} \sum_{i=1}^n \trans{\Ve}(\VY_i-\Vmu_i)
\end{bmatrix}
\end{align*}
Using that \(1 - \frac{\rho p}{1+\rho(p-1)} = 1 + \rho(p-1) - \rho p =
1 - \rho > 0\) and substracting \(p_1/p\) times equation 1 from equation 2 we get:
\begin{align*}
\begin{bmatrix}
0 \\ 0
\end{bmatrix}
& =
\begin{bmatrix}
\sum_{i=1}^n \trans{e}(\VY_i-\Vmu_i) \\
\sum_{i=1}^n \trans{\VT}(\VY_i-\Vmu_i) - \frac{p_1}{p}\sum_{i=1}^n \trans{\Ve}(\VY_i-\Vmu_i)
\end{bmatrix}
\end{align*}
Denoting the by \(\widehat{\alpha}= \frac{1}{np} \sum_{i=1}^n
\sum_{t=1}^p (1-T_{it}) Y_{it}\) and \(\widehat{\beta}= \frac{1}{np}
\sum_{i=1}^n \sum_{t=1}^p T_{it} Y_{it} - \widehat{\alpha}\) the
empirical mean over timepoints and patients under control and under
treatment. The former equations are equivalent to:
\begin{align*}
\begin{bmatrix}
0 \\ 0
\end{bmatrix}
& =
\begin{bmatrix}
\widehat{\alpha} - \alpha + p_1 (\widehat{\beta} - \beta) \\
p_1 (\widehat{\alpha} + \widehat{\beta} - \alpha - \beta) - \frac{p_1}{p} (\widehat{\alpha} - \alpha + p_1 (\widehat{\beta} - \beta))
\end{bmatrix} \\
\begin{bmatrix}
0 \\ 0
\end{bmatrix} 
& =
\begin{bmatrix}
\widehat{\alpha} - \alpha + (\widehat{\beta} - \beta) \\
(\widehat{\alpha} - \alpha + \widehat{\beta} - \beta ) - \frac{1}{p} (\widehat{\alpha} - \alpha + p_1 (\widehat{\beta} - \beta))
\end{bmatrix} 
\end{align*}
So \(\widehat{\beta} - \beta = -\frac{1}{p_1}(\widehat{\alpha} - \alpha)\) and:
\begin{align*}
0 = (\widehat{\alpha} - \alpha)\left(1-\frac{1}{p_1}-\frac{1}{p}+1) \right)
\end{align*}
Since design \(p_0 \geq 1\) and \(p \geq 2\) so \(2-\frac{1}{p_1}-\frac{1}{p} \geq 0.5\). It
follows that \(\alpha = \widehat{\alpha}\) and therefore
\(\beta=\widehat{\beta}\): the maximum likelihood (ML) and restricted
maximum likelihood (REML) estimates of the mean parameters are the
empirical means in the appropriate sub-groups.

\subsection{Correlation parameter (ML)}
\label{sec:org9ae704b}

The ML score equation w.r.t the correlation parameter is:
\begin{align*}
0 =& -\frac{n}{2} tr\left(\Omega^{-1} \frac{\partial \Omega}{\partial\rho}\right) + \frac{1}{2} \sum_{i=1}^n \trans{(\VY_i-\Vmu_i)} \Omega^{-1} \frac{\partial \Omega}{\partial \rho} \Omega^{-1} (\VY_i-\Vmu_i) \\
  =& -\frac{n}{2} tr\left(R^{-1} \frac{\partial R}{\partial\rho}\right) + \frac{1}{2\sigma^2} tr\left(R^{-1} \frac{\partial R}{\partial \rho} R^{-1} \sum_{i=1}^n \trans{(\VY_i-\Vmu_i)}  (\VY_i-\Vmu_i)\right) \\
  =& tr\left(R^{-1} \frac{\partial R}{\partial\rho}\right) - tr\left(R^{-1} \frac{\partial R}{\partial \rho} R^{-1} \frac{1}{n \sigma^2}\sum_{i=1}^n \trans{(\VY_i-\Vmu_i)} (\VY_i-\Vmu_i) \right) 
\end{align*}


We first explicit the first term:
\begin{align*}
R^{-1} \frac{\partial R}{\partial\rho} &= \frac{1}{1-\rho} \left(I - \frac{\rho \Ve\trans{\Ve}}{1+\rho(p-1)}\right)\left(-I + \Ve\trans{\Ve}\right) \\
&= \frac{1}{1-\rho} \left(-I + \Ve\trans{\Ve} + \frac{\rho \Ve\trans{\Ve}}{1+\rho(p-1)} - \frac{\rho p \Ve\trans{\Ve}}{1+\rho(p-1)}\right)\\
&= \frac{1}{1-\rho} \left(-I + \Ve\trans{\Ve} \frac{1+\rho(p-1)+\rho-\rho p}{1+\rho(p-1)}\right)\\
&= \frac{1}{1-\rho} \left(-I +  \frac{\Ve\trans{\Ve}}{1+\rho(p-1)}\right)
\end{align*}

Thus:
\begin{align*}
tr \left( R^{-1} \frac{\partial R}{\partial\rho} \right) &= \frac{p}{1-\rho}\left(-1+\frac{1}{1+\rho(p-1)}\right) = -\frac{p\rho(p-1)}{(1-\rho)(1+\rho(p-1))}
\end{align*}

We now consider:
\begin{align*}
R^{-1} \frac{\partial R}{\partial\rho} R^{-1} &= \frac{1}{(1-\rho)^2} \left(-I +  \frac{\Ve\trans{\Ve}}{1+\rho(p-1)}\right)\left(I - \frac{\rho \Ve\trans{\Ve}}{1+\rho(p-1)}\right) \\
&= \frac{1}{(1-\rho)^2} \left(-I + \frac{\rho \Ve\trans{\Ve}}{1+\rho(p-1)} + \frac{\Ve\trans{\Ve}}{1+\rho(p-1)} - \frac{\rho p \Ve\trans{\Ve}}{(1+\rho(p-1))^2}\right) \\
&= \frac{1}{(1-\rho)^2} \left(-I + \Ve\trans{\Ve} \frac{\rho+\rho^2(p-1) + 1+ \rho(p-1) - \rho p}{(1+\rho(p-1))^2}\right) \\
&= \frac{1}{(1-\rho)^2} \left(-I + \Ve\trans{\Ve} \frac{\rho^2(p-1) + 1}{(1+\rho(p-1))^2}\right) 
\end{align*}

We now consider the matrix \(\frac{1}{n}\sum_{i=1}^n \trans{(\VY_i-\Vmu_i)} (\VY_i-\Vmu_i)\) and denote by
\(\left(\widehat{\sigma}^2_1,\ldots,\widehat{\sigma}^2_p\right)\) its
diagonal elements and by
\(\widehat{\sigma}^2_{j,j^{\prime}}=\widehat{\sigma}_j
\widehat{\sigma}_{j^{\prime}}\widehat{\rho}_{j,j^{\prime}}\) its off
diagonal elements.
\begin{align*}
& tr \left( R^{-1} \frac{\partial R}{\partial\rho} R^{-1} \frac{1}{n\sigma^2}\sum_{i=1}^n \trans{(\VY_i-\Vmu_i)} (\VY_i-\Vmu_i) \right) \\
& = \frac{1}{\sigma^2(1-\rho)^2}\left(\sum_{j=1}^p \widehat{\sigma}^2_j\left(-1+\frac{\rho^2(p-1) + 1}{(1+\rho(p-1))^2}\right) + \frac{2\rho^2(p-1) + 2}{(1+\rho(p-1))^2} \sum_{j < j^{\prime}}\widehat{\sigma}_j
\widehat{\sigma}_{j^{\prime}}\widehat{\rho}_{j,j^{\prime}}\right) \\
&= \frac{1}{\sigma^2(1-\rho)^2}\left(\sum_{j=1}^p \widehat{\sigma}^2_j\left(\frac{-2\rho(p-1)-\rho^2(p-1)^2+\rho^2(p-1)}{(1+\rho(p-1))^2}\right) + \frac{2\rho^2(p-1) + 2}{(1+\rho(p-1))^2} \sum_{j < j^{\prime}}\widehat{\sigma}_j
\widehat{\sigma}_{j^{\prime}}\widehat{\rho}_{j,j^{\prime}}\right) \\
&= \frac{1}{\sigma^2(1-\rho)^2(1+\rho(p-1))^2}\left(\sum_{j=1}^p \widehat{\sigma}^2_j \rho(p-1)\left(-2-\rho (p-2)\right) + \left(2\rho^2(p-1) + 2\right) \sum_{j < j^{\prime}}\widehat{\sigma}_j
\widehat{\sigma}_{j^{\prime}}\widehat{\rho}_{j,j^{\prime}}\right)
\end{align*}

Then \(0 = tr\left(R^{-1} \frac{\partial R}{\partial\rho}\right) - tr\left(R^{-1} \frac{\partial R}{\partial \rho} R^{-1} \frac{1}{n}\sum_{i=1}^n \zeta_i  \trans{\zeta}_i\right)\) involves that:
\begin{align*}
\sigma^2(\rho-1)(1+\rho(p-1)) p\rho(p-1) &= \sum_{j=1}^p \widehat{\sigma}^2_j \rho(p-1)\left(-2-\rho (p-2)\right) + \left(2\rho^2(p-1) + 2\right) \sum_{j < j^{\prime}}\widehat{\sigma}_j
\widehat{\sigma}_{j^{\prime}}\widehat{\rho}_{j,j^{\prime}}  \\
\frac{1}{p(p-1)/2}\sum_{j < j^{\prime}}\widehat{\sigma}_j \widehat{\sigma}_{j^{\prime}}\widehat{\rho}_{j,j^{\prime}} &= \rho
\frac{\sigma^2(\rho-1)(1+\rho(p-1)) + \frac{1}{p}\sum_{j=1}^p \widehat{\sigma}^2_j \left(2+\rho (p-2)\right)}{\rho^2(p-1) + 1}
\end{align*}
Using that \((\rho-1)(1+\rho(p-1))=\rho-1+\rho^2(p-1)-\rho(p-1)=\rho^2(p-1)-\rho(p-2)-1\):
\begin{align*}
\sigma^2(\rho-1)(1+\rho(p-1)) p\rho(p-1) &= \sum_{j=1}^p \widehat{\sigma}^2_j \rho(p-1)\left(-2-\rho (p-2)\right) + \left(2\rho^2(p-1) + 2\right) \sum_{j < j^{\prime}}\widehat{\sigma}_j
\widehat{\sigma}_{j^{\prime}}\widehat{\rho}_{j,j^{\prime}}  \\
\frac{1}{p(p-1)/2}\sum_{j < j^{\prime}}\widehat{\sigma}_j \widehat{\sigma}_{j^{\prime}}\widehat{\rho}_{j,j^{\prime}} &= \rho
\frac{\sigma^2 \rho^2(p-1) + \rho(p-2) (\frac{1}{p}\sum_{j=1}^p \widehat{\sigma}^2_j - \sigma^2) + 2 \frac{1}{p}\sum_{j=1}^p \widehat{\sigma}^2_j - \sigma^2}{\rho^2(p-1) + 1} \\
\frac{1}{p(p-1)/2}\sum_{j < j^{\prime}}\frac{\widehat{\sigma}_j \widehat{\sigma}_{j^{\prime}}}{\sigma^2}\widehat{\rho}_{j,j^{\prime}} &= \rho
\frac{\rho^2(p-1) + \rho(p-2) (\frac{1}{p}\sum_{j=1}^p \frac{\widehat{\sigma}^2_j}{\sigma^2} - 1) + 2 \frac{1}{p}\sum_{j=1}^p \frac{\widehat{\sigma}^2_j}{\sigma^2} - 1}{\rho^2(p-1) + 1} 
\end{align*}

Dividing by \(\sigma^2\) (which is assumed strictly positive), the
score equation for the correlation parameter can be simplified into:
\begin{align*}
\frac{1}{p(p-1)/2}\sum_{j < j^{\prime}}\frac{\widehat{\sigma}_j \widehat{\sigma}_{j^{\prime}}}{\sigma^2}\widehat{\rho}_{j,j^{\prime}} &= \rho + \rho \left(\frac{1}{p}\sum_{j=1}^p \frac{\widehat{\sigma}^2_j}{\sigma^2} - 1\right) \frac{\rho(p-2)  + 2}{\rho^2(p-1) + 1}
\end{align*}

\subsection{Variance parameter (ML)}
\label{sec:orgfc83723}

The ML score equation w.r.t the variance parameter is:
\begin{align*}
0=&-\frac{n}{2} tr\left(\Omega^{-1} \frac{\partial \Omega}{\partial\sigma^2}\right) + \frac{1}{2} \sum_{i=1}^n \trans{(\VY_i-\Vmu_i)} \Omega^{-1} \frac{\partial \Omega}{\partial \sigma^2} \Omega^{-1} (\VY_i-\Vmu_i) \\
 =&-\frac{n}{2} tr\left(\sigma^{-2} R^{-1} R \right) + \frac{1}{2 \sigma^4} \sum_{i=1}^n \trans{(\VY_i-\Vmu_i)} R^{-1} R R^{-1} (\VY_i-\Vmu_i) \\
 =&-\frac{pn}{2 \sigma^2} + \frac{1}{2 \sigma^4} \sum_{i=1}^n \trans{(\VY_i-\Vmu_i)} R^{-1} (\VY_i-\Vmu_i) \\ 
\sigma^2 =& \frac{1}{n p} \sum_{i=1}^n \trans{(\VY_i-\Vmu_i)} R^{-1} (\VY_i-\Vmu_i) 
\end{align*}

Using the expression of \(R^{-1}\) found in appendix \ref{sm:rhoML} we get:
\begin{align*}
\sigma^2 =& \frac{1}{n p (1- \rho)} \sum_{i=1}^n \trans{(\VY_i-\Vmu_i)} \left(I - \frac{\rho \Ve\trans{\Ve}}{(1-\rho)+\rho p} \right) (\VY_i-\Vmu_i)  \\
 =& \frac{1}{n p (1- \rho)} \sum_{i=1}^n \trans{(\VY_i-\Vmu_i)}(\VY_i-\Vmu_i)  - \frac{\rho}{(1-\rho)^2+\rho(1-\rho) p} \frac{1}{np} \sum_{i=1}^n \trans{(\VY_i-\Vmu_i)} \Ve\trans{\Ve} (\VY_i-\Vmu_i)  \\
 =& \frac{\widehat{\sigma}^2}{1- \rho}  - \frac{\rho p}{(1-\rho)^2+\rho(1-\rho) p} \frac{1}{n} \sum_{i=1}^n \left(\frac{1}{p}\sum_{j=1}^p Y_{i,j}-\mu_{i,j}\right)^2
\end{align*}

Since:
\begin{align*}
\frac{1}{n} \sum_{i=1}^n \left(\frac{1}{p}\sum_{j=1}^p Y_{i,j}-\mu_{i,j}\right)^2=& \frac{1}{np^2} \sum_{i=1}^n \sum_{j=1}^p \sum_{j^{\prime}=1}^p \left(Y_{i,j}-\mu_j\right)\left(Y_{i,j^{\prime}}-\mu_{j^{\prime}}\right) \\
=&  \frac{1}{p^2} \left(\sum_{j=1}^p \widehat{\sigma}^2_j + 2\sum_{j < j^{\prime}}\widehat{\sigma}_j \widehat{\sigma}_{j^{\prime}}\widehat{\rho}_{j,j^{\prime}}\right) 
\end{align*}

We have that:
\begin{align*}
\sigma^2 =& \frac{\widehat{\sigma}^2}{(1- \rho)}  - \frac{1}{p}\frac{\rho}{(1-\rho)^2+\rho(1-\rho)p} \left(\sum_{j=1}^p \widehat{\sigma}^2_j + 2\sum_{j < j^{\prime}}\widehat{\sigma}_j \widehat{\sigma}_{j^{\prime}}\widehat{\rho}_{j,j^{\prime}}\right)
\end{align*}
Dividing by \(\sigma^2\) (which is assumed strictly positive) and
using the score equation w.r.t to the correlation parameter:
\begin{align*}
1- \rho =& \frac{\widehat{\sigma}^2}{\sigma^2}  - \frac{1}{p}\frac{\rho}{1-\rho+\rho p} \left(\sum_{j=1}^p \frac{\widehat{\sigma}^2_j}{\sigma^2} + \rho p (p-1) + \rho p (p-1) \left(\frac{1}{p}\sum_{j=1}^p \frac{\widehat{\sigma}^2_j}{\sigma^2} - 1\right) \frac{\rho(p-2)  + 2}{\rho^2(p-1) + 1}\right) \\
 =& \frac{\widehat{\sigma}^2}{\sigma^2}  - \frac{\rho^2(p-1)}{\rho(p-1)+1} \left(1 - \frac{\rho(p-2)  + 2}{\rho^2(p-1) + 1} \right) - \frac{\rho}{1-\rho+\rho p} \left(1 + \rho (p-1) \frac{\rho(p-2)  + 2}{\rho^2(p-1) + 1} \right) \frac{1}{p} \sum_{j=1}^p \frac{\widehat{\sigma}^2_j}{\sigma^2} 
\end{align*}


We first simplify the second term, adding substracting \(\rho\) in the first numerator:
\begin{align*}
& \frac{\rho^2(p-1)}{\rho(p-1)+1} \left(1 - \frac{\rho(p-2)  + 2}{\rho^2(p-1) + 1} \right)
=  \left(\rho - \frac{\rho}{\rho(p-1)+1}\right)\left(1 - \frac{\rho(p-2)  + 2}{\rho^2(p-1) + 1} \right) \\
&=  \rho - \frac{\rho}{\rho(p-1)+1} -  \frac{\rho^2(p-2)  + 2\rho}{\rho^2(p-1) + 1} +  \frac{\rho^2(p-2)  + 2\rho}{(\rho^2(p-1) + 1)(\rho(p-1)+1)} \\
&=  \rho - 1 - \frac{\rho}{\rho(p-1)+1} -  \frac{-\rho^2 + 2\rho - 1}{\rho^2(p-1) + 1} +  \frac{\rho^2(p-2)  + 2\rho}{(\rho^2(p-1) + 1)(\rho(p-1)+1)}  \\
&=  \rho - 1 + \frac{- \rho^3 (p-1) - \rho + \rho^3(p-1) - 2\rho^2(p-1)+\rho(p-1)+\rho^2-2\rho-1 + \rho^2(p-2)  + 2\rho}{(\rho^2(p-1) + 1)(\rho(p-1)+1)}  \\
&=  \rho - 1 + \frac{-2\rho^2(p-1)+\rho^2 + \rho^2(p-2)- \rho +\rho(p-1)-2\rho  + 2\rho +1 }{(\rho^2(p-1) + 1)(\rho(p-1)+1)}  \\
&=  \rho - 1 + \frac{\rho^2(-p+1) + \rho(p-2) +1 }{(\rho^2(p-1) + 1)(\rho(p-1)+1)} =  \rho - 1 + \frac{-\rho(\rho (p-1) +1) + \rho(p-1) +1 }{(\rho^2(p-1) + 1)(\rho(p-1)+1)} \\
&=  \rho - 1 + \frac{1-\rho}{\rho^2(p-1) + 1} 
\end{align*}

We then simplify the third term, adding substracting \(\rho\) in the first numerator:
\begin{align*}
\frac{\rho}{1-\rho+\rho p} \left(1 + \rho (p-1) \frac{\rho(p-2)  + 2}{\rho^2(p-1) + 1} \right) =& \frac{\rho^3(p-1) + \rho + \rho^2 (p-1)(\rho(p-2)+2) }{(\rho^2(p-1) + 1)(\rho(p-1)+1)}  \\
=& \frac{\rho^3(p-1)^2 + 2\rho^2(p-1) + \rho}{\rho^3(p-1)^2 + \rho^2(p-1) + \rho(p-1) + 1}  \\
=& 1 + \frac{\rho^2(p-1) - \rho (p-2) - 1}{(\rho^2(p-1) + 1)(\rho(p-1)+1)}  \\
=& 1 - \frac{1-\rho}{\rho^2(p-1) + 1} 
\end{align*}

Collecting the terms we get:
\begin{align*}
1- \rho =& \frac{\widehat{\sigma}^2}{\sigma^2}  - \left(\rho - 1 + \frac{1-\rho}{\rho^2(p-1) + 1}\right)  - \left(1 - \frac{1-\rho}{\rho^2(p-1) + 1} \right)\frac{1}{p} \sum_{j=1}^p \frac{\widehat{\sigma}^2_j}{\sigma^2}  \\
0 =& \frac{\widehat{\sigma}^2}{\sigma^2} -\frac{1}{p} \sum_{j=1}^p \frac{\widehat{\sigma}^2_j}{\sigma^2} + \frac{1-\rho}{\rho^2(p-1) + 1} \left(\frac{1}{p} \sum_{j=1}^p \frac{\widehat{\sigma}^2_j}{\sigma^2} - 1\right)
\end{align*}
Using that \(\widehat{\sigma}^2 = \frac{1}{np}\sum_{i=1}^n\sum_{j=1}^p (Y_{i,j}-\mu_{i,j})^2 = \frac{1}{p}\sum_{j=1}^p \widehat{\sigma}^2_j\), we finally obtain:
\begin{align*}
0 =& \frac{1-\rho}{\rho^2(p-1) + 1} \left(\frac{\widehat{\sigma}^2}{\sigma^2}-1\right) 
\end{align*}
Since \(\frac{1-\rho}{\rho^2(p-1) + 1}\neq 0\) for acceptable \(\rho\)
(i.e. \(\rho \in ]-1,1[\)) then we must have \(\sigma^2 =
\widehat{\sigma}^2\). Plugging this value in the score equation for
the correlation parameter leads to:
\begin{align*}
\frac{1}{p(p-1)/2}\sum_{j < j^{\prime}}\frac{\widehat{\sigma}_j \widehat{\sigma}_{j^{\prime}}}{\widehat{\sigma}^2}\widehat{\rho}_{j,j^{\prime}} &= \rho
\frac{\rho^2(p-1) + 1}{\rho^2(p-1) + 1} = \rho
\end{align*}


\subsection{Variance parameter (REML)}
\label{sec:orgd581d8c}

The REML score equation w.r.t the variance parameter is the same as the ML score equation with the additional term:
\begin{align*}
&\frac{1}{2} tr\left(\left(\trans{X}\Omega^{-1}X\right)^{-1} \left(\trans{X}\Omega^{-1}\frac{\partial \Omega}{\partial\sigma^2}\Omega^{-1}X \right) \right) \\
&= \frac{1}{2\sigma^2} tr\left(\left(\trans{X}\Omega^{-1}X\right)^{-1} \left(\trans{X}\Omega^{-1}X \right) \right) = \frac{2}{2\sigma^2}
\end{align*}
leading to
\begin{align*}
\sigma^2 =& \frac{1}{n p - 2} \sum_{i=1}^n \trans{(\VY_i-\Vmu_i)} R^{-1} (\VY_i-\Vmu_i) 
\end{align*}


\subsection{Correlation parameter (REML)}
\label{sec:org9d1ee10}

The REML score equation w.r.t the correlation parameter is the same as the ML score equation with the additional term:
\begin{align*}
&\frac{1}{2} tr\left(\left(\trans{X}\Omega^{-1}X\right)^{-1} \left(\trans{X}\Omega^{-1}\frac{\partial \Omega}{\partial\rho}\Omega^{-1}X \right) \right) \\
=&\frac{1}{2 \sigma^4} tr\left(\left(\trans{X}R^{-1}X\right)^{-1} \left(\trans{X}R^{-1}\frac{\partial R}{\partial\rho}R^{-1}X \right) \right) 
\end{align*}

Using from appendix \ref{sm:seRI} that:
\begin{align*}
\left(\trans{X}R^{-1}X\right)^{-1} = \frac{1}{p-p_1} \begin{bmatrix} 1+\rho (p-p_1-1)
                  & -(1-\rho)
                  \\ -(1-\rho)
                  & \frac{p}{p_1}(1-\rho)
                  \end{bmatrix}
\end{align*}
and that:
We now consider:
\begin{align*}
R^{-1} \frac{\partial R}{\partial\rho} R^{-1} &= \frac{1}{(1-\rho)^2} \left(-I + \Ve\trans{\Ve} \frac{\rho^2(p-1) + 1}{(1+\rho(p-1))^2}\right)  \\
\trans{X}R^{-1} \frac{\partial R}{\partial\rho} R^{-1} X &= \frac{1}{(1-\rho)^2} \left(-\trans{X}X + \trans{X}\Ve\trans{\Ve}X \frac{\rho^2(p-1) + 1}{(1+\rho(p-1))^2}\right)  \\
&= \frac{1}{(1-\rho)^2} \left(-\begin{bmatrix} p
                  & p_1
                  \\ p_1
                  & p
                  \end{bmatrix}
                  + \begin{bmatrix} p^2
                  & p p_1
                  \\ p p_1
                  & p_1^2
                  \end{bmatrix} \frac{\rho^2(p-1) + 1}{(1+\rho(p-1))^2}\right) 
\end{align*}
which does not seems to simplify, i.e. the trace has a complicated expression:

\begin{align*}
&\frac{1}{2 \sigma^4} tr\left(\left(\trans{X}R^{-1}X\right)^{-1} \left(\trans{X}R^{-1}\frac{\partial R}{\partial\rho}R^{-1}X \right) \right) = \ldots \\
=&\frac{p p_1^2 ((p-1)\rho^2+1)}{2 \sigma^4 p_1(p-p_1)(1-\rho)((p-1)\rho+1)^2} - \frac{p^2}{2 \sigma^4 p_1(p-p_1)(1-\rho)} + \frac{(p-1)p(\rho(p-p_1-1)+1)}{2\sigma^4(p-p_1)((p-1)\rho+1)^2} \\
=& \frac{p}{2\sigma^4(p-p_1)} \left(\frac{p_1^2 ((p-1)\rho^2+1)}{p_1(1-\rho)((p-1)\rho+1)^2} - \frac{p}{p_1(1-\rho)} + \frac{(p-1)(\rho(p-p_1-1)+1)}{((p-1)\rho+1)^2}\right) = \ldots \\
=& \frac{p}{2\sigma^4(p-p_1)} \left(\frac{p^2-p_1^2}{p p_1 (\rho-1)} + \frac{p^2 - p p_1 - p - p_1}{p (p\rho - \rho + 1)} + \frac{2p_1}{p \rho - \rho +1 }\right)
\end{align*}


\clearpage

\section{Standard error of the treatment effect \newline in a balanced random intercept model}
\label{sm:seRI}
Consider a random intercept model including single binary covariate
(called treatment):
\begin{align*}
Y_{it} = \mu + \beta T_{it} + \alpha_i + \varepsilon_{it}
\end{align*}
where \(\alpha_i \sim \Gaus[0,\tau]\) and \(\varepsilon_{it} \sim
\Gaus[0,\delta]\). Denote \(\rho = \frac{\tau}{\tau+\delta}\) and
\(\sigma^2=\tau+\delta\) such that:
\begin{align*}
\Var[Y_{it}] = \Omega = \sigma^2 R = \sigma^2 ((1-\rho) I + \rho e\trans{e})
\end{align*}
where \(I\) and \(e\) were defined in section \ref{sm:invCS}. The inverse
of \(R\) was also explicit in section \ref{sm:invCS} and when multiplied
the \(p \times 2\) matrix \(X=(1,T)\) where \(T\) is either \(0\) or
\(1\), respectively \(p_0\) and \(p_1\) times, we get:
\begin{align*}
\trans{X} R^{-1} X &= \frac{1}{1-\rho} \trans{X}X - \frac{\rho\trans{X} e\trans{e} X}{(1-\rho)^2+\rho(1-\rho)p}  \\
&= \frac{1}{1-\rho} \left(\trans{X}X - \frac{\rho\trans{X} e\trans{e} X}{1 + \rho (p-1)}\right)  \\
&= \frac{1}{1-\rho} \left(\begin{bmatrix} p & p_1 \\ p_1 & p_1 \end{bmatrix} - \frac{\rho}{1+\rho(p-1)}  \begin{bmatrix} p^2 & p p_1 \\ p p_1 & p^2_1 \end{bmatrix}\right) \\
&= \frac{1}{(1-\rho)(1+\rho(p-1))} \begin{bmatrix} p+p\rho(p-1) - \rho p^2
                  & p_1+p_1\rho(p-1)- \rho p p_1
                  \\ p_1+p_1\rho(p-1)- \rho p p_1
                  & p_1+p_1\rho(p-1)- \rho p_1^2
\end{bmatrix}   \\
&= \frac{1}{(1-\rho)(1+\rho(p-1))} \begin{bmatrix} p(1-\rho)
                  & p_1(1-\rho)
                  \\ p_1(1-\rho)
                  & p_1(1+\rho (p-p_1-1))
\end{bmatrix}   
\end{align*}

whose inverse is:
\begin{align*}
\left(\trans{X} R^{-1} X\right)^{-1} &= \frac{(1-\rho)(1+\rho(p-1))}{p_1 p (1-\rho)(1+\rho (p-p_1-1)) - p^2_1(1-\rho)^2} \begin{bmatrix} p_1(1+\rho (p-p_1-1))
                  & -p_1(1-\rho)
                  \\ -p_1(1-\rho)
                  & p(1-\rho)
\end{bmatrix} \\
&= \frac{1+\rho(p-1)}{p_1 p (1+\rho (p-p_1-1)) - p^2_1(1-\rho)} \begin{bmatrix} p_1(1+\rho (p-p_1-1))
                  & -p_1(1-\rho)
                  \\ -p_1(1-\rho)
                  & p(1-\rho)
\end{bmatrix} \\
&= \frac{1+\rho(p-1)}{(p - p_1) + \rho (p^2-p p_1-p+p_1)} \begin{bmatrix} 1+\rho (p-p_1-1)
                  & -(1-\rho)
                  \\ -(1-\rho)
                  & \frac{p}{p_1}(1-\rho)
\end{bmatrix} \\
&= \frac{1}{p-p_1} \begin{bmatrix} 1+\rho (p-p_1-1)
                  & -(1-\rho)
                  \\ -(1-\rho)
                  & \frac{p}{p_1}(1-\rho)
\end{bmatrix}   
\end{align*}

\clearpage

So in the random intercept model, the standard error of the treatment
estimator will be:
\begin{align*}
\sigma_{\widehat{\beta}} = \sqrt{\sigma_0^2(1-\rho) \frac{p}{n p_1(p-p_1)}}=\sqrt{\frac{\delta}{n} \frac{p}{p_1(p-p_1)}}
\end{align*}

In a design with as many observations under treatment as under control \(p_1=p/2\) and the expression simplifies into.
\begin{align*}
\sigma_{\widehat{\beta}} = \sqrt{\frac{4\delta}{np}} = \sqrt{\frac{2\delta}{np_1}}
\end{align*}

From section \ref{sm:rhoML} we deduce that:
\begin{align*}
\sigma_{\widehat{\beta}} = \sqrt{\frac{\left(1-\frac{1}{p(p-1)/2}\sum_{t \neq t^{\prime}} \rho_{t,t^{\prime}}\right) \sigma^2}{n}\frac{p}{p_1(p-p_1)}}
\end{align*}
which in a design with as many observations under treatment as under control simplifies to:
\begin{align*}
\sigma_{\widehat{\beta}} = \sqrt{\frac{2\left(1-\frac{1}{p(p-1)/2}\sum_{t \neq t^{\prime}} \rho_{t,t^{\prime}}\right) \sigma^2}{n p_1}}
\end{align*}

Note: when using a t-test on the change based only on the first
observation under each treatment the variance is:
\begin{align*}
\sigma_{\widehat{\beta}} = \sqrt{\frac{2(1-\rho_{1,p+1}) \sigma^2}{n}}
\end{align*}

\clearpage
\end{document}