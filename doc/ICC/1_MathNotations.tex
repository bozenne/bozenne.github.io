%%%% NOTE
% requires package: 

\usepackage{ifthen} % equality test
\usepackage{xifthen} % isempty function
\usepackage{xargs} % \newcommandx




%  Functions for shortcuts
\newcommand\defOperator[7]{%
	\ifthenelse{\isempty{#2}}{
		\ifthenelse{\isempty{#1}}{#7{#3}#4}{#7{#3}#4 \left#5 #1 \right#6}
	}{
	\ifthenelse{\isempty{#1}}{#7{#3}#4_{#2}}{#7{#3}#4_{#1}\left#5 #2 \right#6}
}
}

\newcommand\defUOperator[5]{%
	\ifthenelse{\isempty{#1}}{
		#5\left#3 #2 \right#4
	}{
	\ifthenelse{\isempty{#2}}{\underset{#1}{\operatornamewithlimits{#5}}}{
		\underset{#1}{\operatornamewithlimits{#5}}\left#3 #2 \right#4}
}
}

\newcommand{\defBoldVar}[2]{	
	\ifthenelse{\equal{#2}{T}}{\boldsymbol{#1}}{\mathbf{#1}}
}


%  Operators
\newcommandx\dpartiel[4][1=,2=,3=,4=\partial]{
	\ifthenelse{\isempty{#3}}{
		\frac{#4 #1}{#4 #2}
	}{
	\left.\frac{#4 #1}{#4 #2}\right|_{#3}
}
}
\newcommandx\dTpartiel[3][1=,2=,3=]{\dpartiel[#1][#2][#3][d]}

\newcommandx\ddpartiel[3][1=,2=,3=]{
	\ifthenelse{\isempty{#3}}{
		\frac{\partial^{2} #1}{\left( \partial #2\right)^2}
	}{
	\frac{\partial^2 #1}{\partial #2\partial #3}
}
}
\newcommandx\stack[1][1=]{
	\ifthenelse{\isempty{#1}}{
		vec \,
	}{
	vec\left\{ #1 \right\}
}
}


\newcommandx\Cov[2][1=,2=]{\defOperator{#1}{#2}{C}{ov}{[}{]}{\mathbb}}
\newcommandx\Esp[2][1=,2=]{\defOperator{#1}{#2}{E}{}{[}{]}{\mathbb}}
\newcommandx\Prob[2][1=,2=]{\defOperator{#1}{#2}{P}{}{[}{]}{\mathbb}}
\newcommandx\Qrob[2][1=,2=]{\defOperator{#1}{#2}{Q}{}{[}{]}{\mathbb}}
\newcommandx\Var[2][1=,2=]{\defOperator{#1}{#2}{V}{ar}{[}{]}{\mathbb}}

\newcommandx\Binom[2][1=,2=]{\defOperator{#1}{#2}{B}{}{(}{)}{\mathcal}}
\newcommandx\Gaus[2][1=,2=]{\defOperator{#1}{#2}{N}{}{(}{)}{\mathcal}}
\newcommandx\Wishart[2][1=,2=]{\defOperator{#1}{#2}{W}{ishart}{(}{)}{\mathcal}}
\newcommandx\Information[2][1=,2=]{\defOperator{#1}{#2}{I}{}{(}{)}{\mathcal}}
\newcommandx\Score[2][1=,2=]{\defOperator{#1}{#2}{S}{}{(}{)}{\mathcal}}

\newcommandx\Vois[2][1=,2=]{\defOperator{#1}{#2}{V}{}{(}{)}{\mathcal}}
\newcommandx\Ind[1][1=]{\defOperator{}{#1}{1}{}{(}{)}{\mathds}}

\newcommandx\Max[2][1=,2=]{\defUOperator{#1}{#2}{(}{)}{min}}
\newcommandx\Min[2][1=,2=]{\defUOperator{#1}{#2}{(}{)}{max}}
\newcommandx\argMax[2][1=,2=]{\defUOperator{#1}{#2}{(}{)}{argmax}}
\newcommandx\argMin[2][1=,2=]{\defUOperator{#1}{#2}{(}{)}{argmin}}
\newcommandx\cvD[2][1=D,2=n \rightarrow \infty]{\xrightarrow[#2]{#1}}

\newcommandx\dnorm[4][1=X,2=,3=\sigma^2,4=]{
	\ifthenelse{\isempty{#2}}{
		\ifthenelse{\isempty{#4}}{
			\frac{1}{\sqrt{ 2 \pi #3 }} \e{-\frac{#1^{2}}{2#3}} 
		}{
		\frac{1}{\sqrt{ (2 \pi)^#4 |#3| }} \e{-\frac{\trans{#1}#1}{2#3}} 
	}
}{
\ifthenelse{\isempty{#3}}{
	\frac{1}{\sqrt{ 2 \pi #3 }} \e{-\frac{\left(#1-#2\right)^{2}}{2#3}}
}{
\frac{1}{\sqrt{ (2 \pi)^#4 |#3| }} \e{-\frac{\trans{\left(#1-#2\right)}\left(#1-#2\right)}{2#3}} 
}
}
}

\newcommand\ABold{\defBoldVar{A}{F}}
\newcommand\BBold{\defBoldVar{B}{F}}
\newcommand\GBold{\defBoldVar{G}{F}}
\newcommand\KBold{\defBoldVar{K}{F}}
\newcommand\PBold{\defBoldVar{P}{F}}
\newcommand\TBold{\defBoldVar{T}{F}}
\newcommand\UBold{\defBoldVar{U}{F}}
\newcommand\sBold{\defBoldVar{s}{F}}
\newcommand\vBold{\defBoldVar{v}{F}}
\newcommand\XBold{\defBoldVar{X}{F}}
\newcommand\yBold{\defBoldVar{y}{F}}
\newcommand\YBold{\defBoldVar{Y}{F}}
\newcommand\ZBold{\defBoldVar{Z}{F}}
\newcommand\alphaBold{\defBoldVar{\alpha}{T}}
\newcommand\etaBold{\defBoldVar{\eta}{T}}
\newcommand\GammaBold{\defBoldVar{\Gamma}{T}}
\newcommand\LambdaBold{\defBoldVar{\Lambda}{T}}
\newcommand\muBold{\defBoldVar{\mu}{T}}
\newcommand\nuBold{\defBoldVar{\nu}{T}}
\newcommand\OmegaBold{\defBoldVar{\Omega}{T}}
\newcommand\SigmaBold{\defBoldVar{\Sigma}{T}}
\newcommand\thetaBold{\defBoldVar{\theta}{T}}
\newcommand\varepsilonBold{\defBoldVar{\varepsilon}{T}}
\newcommand\xiBold{\defBoldVar{\xi}{T}}
\newcommand\zetaBold{\defBoldVar{\zeta}{T}}
\newcommand\reditem[1]{\setbeamercolor{item}{fg=red}\item #1} 
\newcommand\greenitem[1]{\setbeamercolor{item}{fg=green}\item #1} 
\newcommand\grayitem[1]{\setbeamercolor{item}{fg=gray}\item #1} 

\newcommand\Real{\mathbb{R}}
\newcommand\Rational{\mathbb{Q}}
\newcommand\Natural{\mathbb{N}}
\newcommand\trans[1]{{#1}^\intercal}%\newcommand\trans[1]{{\vphantom{#1}}^\top{#1}}
\newcommand\pmax[2]{\underset{#1}{\operatornamewithlimits{max}}\left( #2 \right)}
\newcommand{\independant}{\mathrel{\text{\scalebox{1.5}{$\perp\mkern-10mu\perp$}}}}
\newcommand\e[1]{e^{#1}}

