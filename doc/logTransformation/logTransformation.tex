% Created 2020-12-17 tor 20:18
% Intended LaTeX compiler: pdflatex
\documentclass[12pt]{article}

%%%% settings when exporting code %%%% 

\usepackage{listings}
\lstset{
backgroundcolor=\color{white},
basewidth={0.5em,0.4em},
basicstyle=\ttfamily\small,
breakatwhitespace=false,
breaklines=true,
columns=fullflexible,
commentstyle=\color[rgb]{0.5,0,0.5},
frame=single,
keepspaces=true,
keywordstyle=\color{black},
literate={~}{$\sim$}{1},
numbers=left,
numbersep=10pt,
numberstyle=\ttfamily\tiny\color{gray},
showspaces=false,
showstringspaces=false,
stepnumber=1,
stringstyle=\color[rgb]{0,.5,0},
tabsize=4,
xleftmargin=.23in,
emph={anova,apply,class,coef,colnames,colNames,colSums,dim,dcast,for,ggplot,head,if,ifelse,is.na,lapply,list.files,library,logLik,melt,plot,require,rowSums,sapply,setcolorder,setkey,str,summary,tapply},
emphstyle=\color{blue}
}

%%%% packages %%%%%

\usepackage[utf8]{inputenc}
\usepackage[T1]{fontenc}
\usepackage{lmodern}
\usepackage{textcomp}
\usepackage{color}
\usepackage{graphicx}
\usepackage{grffile}
\usepackage{wrapfig}
\usepackage{rotating}
\usepackage{longtable}
\usepackage{multirow}
\usepackage{multicol}
\usepackage{changes}
\usepackage{pdflscape}
\usepackage{geometry}
\usepackage[normalem]{ulem}
\usepackage{amssymb}
\usepackage{amsmath}
\usepackage{amsfonts}
\usepackage{dsfont}
\usepackage{array}
\usepackage{ifthen}
\usepackage{hyperref}
\usepackage{natbib}
\RequirePackage{ifthen}
\RequirePackage{xifthen}
\RequirePackage{xargs}
\RequirePackage{xspace}
\newcommand\Rlogo{\textbf{\textsf{R}}\xspace} %
\RequirePackage{fancyvrb}
\DefineVerbatimEnvironment{verbatim}{Verbatim}{fontsize=\small,formatcom = {\color[rgb]{0.5,0,0}}}
\RequirePackage{colortbl} % arrayrulecolor to mix colors
\RequirePackage{setspace} % to modify the space between lines - incompatible with footnote in beamer
\usepackage{authblk} % enable several affiliations (clash with beamer)
\renewcommand{\baselinestretch}{1.1}
\geometry{top=1cm}
\RequirePackage{epstopdf} % to be able to convert .eps to .pdf image files
\RequirePackage{capt-of} %
\RequirePackage{caption} % newlines in graphics
\usepackage{stackengine}
\usepackage{scalerel}
\newcommand\Warning[1][3ex]{%
\renewcommand\stacktype{L}%
\scaleto{\stackon[1.3pt]{\color{red}$\triangle$}{\tiny\bfseries !}}{#1}%
\xspace
}
\RequirePackage{amsmath}
\RequirePackage{algorithm}
\RequirePackage[noend]{algpseudocode}
\RequirePackage{dsfont}
\RequirePackage{amsmath,stmaryrd,graphicx}
\RequirePackage{prodint} % product integral symbol (\PRODI)
\newcommand\defOperator[7]{%
\ifthenelse{\isempty{#2}}{
\ifthenelse{\isempty{#1}}{#7{#3}#4}{#7{#3}#4 \left#5 #1 \right#6}
}{
\ifthenelse{\isempty{#1}}{#7{#3}#4_{#2}}{#7{#3}#4_{#1}\left#5 #2 \right#6}
}
}
\newcommand\defUOperator[5]{%
\ifthenelse{\isempty{#1}}{
#5\left#3 #2 \right#4
}{
\ifthenelse{\isempty{#2}}{\underset{#1}{\operatornamewithlimits{#5}}}{
\underset{#1}{\operatornamewithlimits{#5}}\left#3 #2 \right#4}
}
}
\newcommand{\defBoldVar}[2]{
\ifthenelse{\equal{#2}{T}}{\boldsymbol{#1}}{\mathbf{#1}}
}
\newcommandx\Cor[2][1=,2=]{\defOperator{#1}{#2}{C}{or}{\lbrack}{\rbrack}{\mathbb}}
\newcommandx\Cov[2][1=,2=]{\defOperator{#1}{#2}{C}{ov}{\lbrack}{\rbrack}{\mathbb}}
\newcommandx\Esp[2][1=,2=]{\defOperator{#1}{#2}{E}{}{\lbrack}{\rbrack}{\mathbb}}
\newcommandx\Prob[2][1=,2=]{\defOperator{#1}{#2}{P}{}{\lbrack}{\rbrack}{\mathbb}}
\newcommandx\Qrob[2][1=,2=]{\defOperator{#1}{#2}{Q}{}{\lbrack}{\rbrack}{\mathbb}}
\newcommandx\Var[2][1=,2=]{\defOperator{#1}{#2}{V}{ar}{\lbrack}{\rbrack}{\mathbb}}
\newcommandx\Binom[2][1=,2=]{\defOperator{#1}{#2}{B}{}{(}{)}{\mathcal}}
\newcommandx\Gaus[2][1=,2=]{\defOperator{#1}{#2}{N}{}{(}{)}{\mathcal}}
\newcommandx\Wishart[2][1=,2=]{\defOperator{#1}{#2}{W}{ishart}{(}{)}{\mathcal}}
\newcommandx\Likelihood[2][1=,2=]{\defOperator{#1}{#2}{L}{}{(}{)}{\mathcal}}
\newcommandx\Information[2][1=,2=]{\defOperator{#1}{#2}{I}{}{(}{)}{\mathcal}}
\newcommandx\Score[2][1=,2=]{\defOperator{#1}{#2}{S}{}{(}{)}{\mathcal}}
\newcommandx\Vois[2][1=,2=]{\defOperator{#1}{#2}{V}{}{(}{)}{\mathcal}}
\newcommandx\IF[2][1=,2=]{\defOperator{#1}{#2}{IF}{}{(}{)}{\mathcal}}
\newcommandx\Ind[1][1=]{\defOperator{}{#1}{1}{}{(}{)}{\mathds}}
\newcommandx\Max[2][1=,2=]{\defUOperator{#1}{#2}{(}{)}{min}}
\newcommandx\Min[2][1=,2=]{\defUOperator{#1}{#2}{(}{)}{max}}
\newcommandx\argMax[2][1=,2=]{\defUOperator{#1}{#2}{(}{)}{argmax}}
\newcommandx\argMin[2][1=,2=]{\defUOperator{#1}{#2}{(}{)}{argmin}}
\newcommandx\cvD[2][1=D,2=n \rightarrow \infty]{\xrightarrow[#2]{#1}}
\newcommandx\Hypothesis[2][1=,2=]{
\ifthenelse{\isempty{#1}}{
\mathcal{H}
}{
\ifthenelse{\isempty{#2}}{
\mathcal{H}_{#1}
}{
\mathcal{H}^{(#2)}_{#1}
}
}
}
\newcommandx\dpartial[4][1=,2=,3=,4=\partial]{
\ifthenelse{\isempty{#3}}{
\frac{#4 #1}{#4 #2}
}{
\left.\frac{#4 #1}{#4 #2}\right\rvert_{#3}
}
}
\newcommandx\dTpartial[3][1=,2=,3=]{\dpartial[#1][#2][#3][d]}
\newcommandx\ddpartial[3][1=,2=,3=]{
\ifthenelse{\isempty{#3}}{
\frac{\partial^{2} #1}{\left( \partial #2\right)^2}
}{
\frac{\partial^2 #1}{\partial #2\partial #3}
}
}
\newcommand\Real{\mathbb{R}}
\newcommand\Rational{\mathbb{Q}}
\newcommand\Natural{\mathbb{N}}
\newcommand\trans[1]{{#1}^\intercal}%\newcommand\trans[1]{{\vphantom{#1}}^\top{#1}}
\newcommand{\independent}{\mathrel{\text{\scalebox{1.5}{$\perp\mkern-10mu\perp$}}}}
\newcommand\half{\frac{1}{2}}
\newcommand\normMax[1]{\left|\left|#1\right|\right|_{max}}
\newcommand\normTwo[1]{\left|\left|#1\right|\right|_{2}}
\author{Brice Ozenne}
\date{\today}
\title{Estimating a relative change using a log-transformation of the outcome}
\hypersetup{
 colorlinks=true,
 citecolor=[rgb]{0,0.5,0},
 urlcolor=[rgb]{0,0,0.5},
 linkcolor=[rgb]{0,0,0.5},
 pdfauthor={Brice Ozenne},
 pdftitle={Estimating a relative change using a log-transformation of the outcome},
 pdfkeywords={},
 pdfsubject={},
 pdfcreator={Emacs 27.0.50 (Org mode 9.0.4)},
 pdflang={English}
 }
\begin{document}

\maketitle

\section{Interpretation of the regression coefficient after log-transformation}
\label{sec:org039d702}
Let's denote by \(Y\) the outcome and by \(G\) a binary group
variable. We are interested in the relative change in \(Y\) between
the groups. We decide to model the group effect on the log scale:
\begin{align*}
\log(Y) = Z = \alpha + \beta G + \varepsilon \text{ where } \Esp[\varepsilon]=0 \text{ and } \Esp[\varepsilon]=\sigma^2
\end{align*}
We claim that:
\begin{align*}
\frac{\Esp[Y|G=1]-\Esp[Y|G=0]}{\Esp[Y|G=0]} = e^{\beta} - 1
\end{align*}

\subsection{Proof: re-writting the model as a multiplicative model}
\label{sec:org337839d}
We can re-write the model as:
\begin{align*}
Y = e^{\alpha + \beta G}e^{\varepsilon} \text{ where }
\end{align*}
So for \(g\in\{1,2\}\):
\begin{align*}
\Esp[Y|G=g] = e^{\alpha + \beta g} \Esp[e^{\varepsilon}]
\end{align*}
Then:
\begin{align*}
\frac{\Esp[Y|G=1]-\Esp[Y|G=0]}{\Esp[Y|G=0]}
& = \frac{e^{\alpha + \beta} \Esp[e^{\varepsilon}]-e^{\alpha} \Esp[e^{\varepsilon}]}{e^{\alpha} \Esp[e^{\varepsilon}]} \\
& = \frac{e^{\alpha + \beta} -e^{\alpha}}{e^{\alpha}}  = e^{\beta} - 1 \\
\end{align*}

\subsection{Proof: using a Taylor expansion}
\label{sec:org0c11632}

Using a second order Taylor expansion of \(\exp(Z)\) around
\(\mu(G)=\alpha + \beta G\) and assuming that the first moments of
\(Z\) are finite and the remaining moments are neglectable regarding
the factorial of the moment order (i.e. \(\forall i \geq 1\),
\(\frac{1}{i!}\Esp[\varepsilon^i ]< +\infty\) and \(\sum_{i=1}^{\infty} \frac{1}{i!}\Esp[\varepsilon^i ]< +\infty\)), we get:
\begin{align*}
Y &= e^{Z} = e^{\mu} + \sum_{i=1}^{\infty} \frac{1}{i!} (Z - \mu)^i \frac{\partial^i e^{\mu}}{(\partial \mu)^i} \\
&= e^{\alpha + \beta G} + \sum_{i=1}^{\infty} \frac{1}{i!} (Z - \alpha - \beta G)^i e^{\alpha + \beta G} \\
\Esp[Y|G=g] &= e^{\alpha + \beta G} + \sum_{i=1}^{\infty} \frac{1}{i!} \Esp[(Z - \alpha - \beta g)^i] e^{\alpha + \beta G} \\
&= e^{\alpha + \beta G} \left(1 + \sum_{i=1}^{\infty} \frac{1}{i!} \Esp[\varepsilon^i] \right)
\end{align*}
where we used that the distribution of \(\varepsilon\) is independent
of \(g\). We can now express our parameter of interest:
\begin{align*}
\Delta_G &= \frac{\Esp[Y|G=1]-\Esp[Y|G=0]}{\Esp[Y|G=0]} = \frac{\Esp[Y|G=1]}{\Esp[Y|G=0]} - 1 \\
&= \frac{e^{\alpha + \beta} \left(1 + \sum_{i=1}^{\infty} \frac{1}{i!} \Esp[\varepsilon^{i}] \right)}{e^{\alpha} \left(1 + \sum_{i=1}^{\infty} \frac{1}{i!} \Esp[\varepsilon^{i}] \right)} - 1 \\
&= e^{\beta} - 1
\end{align*}


\clearpage

\section{Power calculation: comparison between two groups}
\label{sec:org5c289c9}

Consider two groups \(G=0\) and \(G=1\) for which we want to compare
the percentage difference in outcome \(Y\). We are willing to assume
that on the log-scale \(Y\) is normally distributed. Our parameter of
interest is:
\begin{align*}
\frac{\Esp[Y|G=1]-\Esp[Y|G=0]}{\Esp[Y|G=0]} = \gamma
\end{align*}
We further fix \(\alpha = \Esp[Y|G=0]\) and \(\sigma^2 = \Var[Y|G=0]\)
and we assume that on the log-scale:
\begin{align*}
\Var[\log(Y)|G=1]  = \Var[\log(Y)|G=0] = s^2
\end{align*}
To evaluate the power for a given \((\alpha,\sigma^2,\gamma)\), we need to identify the joint distribution: 
\begin{align*}
\begin{bmatrix}
Z_0 = \log(Y)|G=0 \\ Z_1  = \log(Y)|G=1
\end{bmatrix}
\sim \Gaus \left(
\begin{bmatrix}
m_0 \\ m_1
\end{bmatrix}
,
\begin{bmatrix}
s^2 & \rho s^2 \\ \rho s^2 & s^2
\end{bmatrix}
\right)
\end{align*}
The standardized effect size is then: \(\frac{m_1-m_0}{s\sqrt{2(1-\rho)}}\). 

\bigskip

Note: in the case of two independent samples \(\rho=0\)

\subsection{Method 1: Taylor expansion}
\label{sec:orgec6be2e}

We will use the fact that \(Z_0,Z_1\) are jointly normally distributed
to identify \(m_0,m_1,s^2,\rho\). First we start by identifying
\(m_0\) and \(s^2\) based on \(\alpha\) and \(\sigma^2\) (reference
group). A Taylor expansion gives (see appendix \ref{SM:exptrans}):
\begin{align*}
\alpha &\approx \exp(m_0)\left(1 + \frac{s^2}{2}+\frac{s^4}{8}+\frac{s^6}{48}\right) \\
\sigma^2 &\approx \exp(2 m_0)\left(s^2 + \frac{3}{2} s^4 + \frac{7}{6} s^6 + \frac{11}{24} s^8 + \frac{21}{320} s^{10}\right)
\end{align*}
So:
\begin{align*}
\frac{\alpha^2}{\sigma^2} - \frac{\left(1 + \frac{s^2}{2}+\frac{s^4}{8}+\frac{s^6}{48}\right)^2}{s^2 + \frac{3}{2} s^4 + \frac{7}{6} s^6 + \frac{11}{24} s^8 + \frac{21}{320} s^{10}} \approx 0
\end{align*}
We get \(s^2\) by solving this equation using that \(s^2 \in
[0;\sigma^2]\) (upper bound follow from Jensen's inequality applied to
\((X-\mu)^2\), \(\log\) being concave). We can then deduce \(m_0\):
\begin{align*}
m_0  &\approx \log\left(\frac{\alpha}{1 + \frac{s^2}{2}+\frac{s^4}{8}+\frac{s^6}{48}}\right) = \log(\alpha) - \log\left(1 + \frac{s^2}{2}+\frac{s^4}{8}+\frac{s^6}{48}\right)
\end{align*}
Then we can identify \(m_1\) using once more a Taylor expansion:
\begin{align*}
\alpha(1+\gamma) &\approx \exp(m_1)\left(1 + \frac{s^2}{2}+\frac{s^4}{8}+\frac{s^6}{48}\right) \\
m_1 &\approx \log\left(\frac{\alpha(1+\gamma)}{1 + \frac{s^2}{2}+\frac{s^4}{8}+\frac{s^6}{48}}\right) = m_0 + \log(1+\gamma)
\end{align*}
Now

\subsection{Method 2: Log-normal distribution}
\label{sec:org173cc71}
We will use the fact that \(Z_0\) follows a log-normal distribution,
meaning that:
\begin{align*}
\alpha &= \exp(m_0 + \frac{1}{2} s^2) \\
\sigma^2 &= \exp(2*m_0 + s^2)*(\exp(s^2)-1) \\
\end{align*}
So
\begin{align*}
s^2 &= \log\left(1+\frac{\sigma^2}{\alpha^2}\right)\\
m_0 &= \log(\alpha)-\frac{s^2}{2}\\
\end{align*}
Then we can identify \(m_1\) using that \(Z_1\) follows a log-normal distribution, i.e.:
\begin{align*}
\alpha(1+\gamma) &= \exp(m_1 + \frac{1}{2} s^2) \\
m_1 &= m_0+\log(1+\gamma)\\
\end{align*}
\subsection{Illustration 1: two sample t-test}
\label{sec:org6a59369}

\textbf{Illustration:} We consider two groups having a 10\% difference
in their baseline value (\(\alpha=1.15\)) and a variance of \(\sigma^2
= 0.15\). What are the parameters of the corresponding normal
distribution on the log-scale and the standardized effect size?
\lstset{language=r,label= ,caption= ,captionpos=b,numbers=none}
\begin{lstlisting}
alpha <- 1.15
sigma2 <- 0.15
gamma <- 0.1
\end{lstlisting}

\textbf{Taylor expansion:} we first identify \(s^2\), \(m_0\), and
\(m_1\):
\lstset{language=r,label= ,caption= ,captionpos=b,numbers=none}
\begin{lstlisting}
s2.taylor <- uniroot(function(x){
    alpha^2/sigma2 - (1+x/2+x^2/8+x^3/48)^2/(x+(3/2)*x^2+(7/6)*x^3+(11/24)*x^4+(21/320)*x^5)},
    interval = c(1e-12,sigma2))$root
m0.taylor <- log(alpha/(1+s2.taylor/2+s2.taylor^2/8+s2.taylor^3/48))
m1.taylor <- m0.taylor + log(1+gamma)
\end{lstlisting}

\textbf{lognormal distribution:} we first identify \(s^2\), \(m_0\),
and \(m_1\):
\lstset{language=r,label= ,caption= ,captionpos=b,numbers=none}
\begin{lstlisting}
s2.logdist <- log(1+sigma2/alpha^2)
m0.logdist <- log(alpha) - s2.logdist/2
m1.logdist <- m0.logdist + log(1+gamma)
\end{lstlisting}

We can compare the moments of am exp-transformed normal distribution
based on these values to the input:
\lstset{language=r,label= ,caption= ,captionpos=b,numbers=none}
\begin{lstlisting}
x <- exp(rnorm(1e5, mean = m0.taylor, sd = sqrt(s2.taylor)))
y <- exp(rnorm(1e5, mean = m1.taylor, sd = sqrt(s2.taylor)))
yx.x <- mean(y)/mean(x)-1
X <- exp(rnorm(1e5, mean = m0.logdist, sd = sqrt(s2.logdist)))
Y <- exp(rnorm(1e5, mean = m1.logdist, sd = sqrt(s2.logdist)))
YX.X <- mean(Y)/mean(X)-1

rbind(
    data.frame(method = "taylor", m0=m0.taylor, m1=m1.taylor, s2=s2.taylor), 
    data.frame(method = "logdist", m0=m0.logdist, m1=m1.logdist, s2=s2.logdist)
)
rbind(
    data.frame(method = "true", 
	       alpha=alpha, gamma=gamma, sigma2=sigma2), 
    data.frame(method = "error.taylor", 
	       alpha=mean(x)-alpha, gamma=yx.x-gamma, sigma2=var(x)-sigma2),
    data.frame(method = "error.logdist", 
	       alpha=mean(X)-alpha, gamma=YX.X-gamma, sigma2=var(X)-sigma2)
)
\end{lstlisting}

\begin{verbatim}
   method         m0        m1        s2
1  taylor 0.08603197 0.1813421 0.1074606
2 logdist 0.08604307 0.1813532 0.1074378
         method         alpha         gamma        sigma2
1          true  1.1500000000  0.1000000000  0.1500000000
2  error.taylor  0.0012850559 -0.0010820104 -0.0002242144
3 error.logdist -0.0005174973 -0.0009134562 -0.0012306318
\end{verbatim}
Similar performance. Maybe a bit better for log-dist.

\subsection{Illustration 2: paired t-test}
\label{sec:org9c71eac}

\textbf{Illustration:} We consider one group having a 10\% difference
between its baseline value (\(\alpha=1.15\)) and its follow-up
value. We assume a variance of \(\sigma^2 = 0.15\) for the baseline
value and a correlation of \(\rho=0.5\) between the baseline and
follow-up value. What are the parameters of the corresponding normal
distribution on the log-scale and the standardized effect size?
\lstset{language=r,label= ,caption= ,captionpos=b,numbers=none}
\begin{lstlisting}
alpha <- 1.15
sigma2 <- 0.15
gamma <- 0.1
rho <- 0.5
\end{lstlisting}

We previously obtained the values for\(s^2\). We can now search for
the right correlation value on the log-scale
\lstset{language=r,label= ,caption= ,captionpos=b,numbers=none}
\begin{lstlisting}
rho.taylor <- uniroot(function(x){
    rho - (x+0.5*x^2*s2.taylor)/(1+(3/2)*s2.taylor+(7/6)*s2.taylor^2+(11/24)*s2.taylor^3+(21/320)*s2.taylor^4)
},interval = c(0,0.9999))$root
\end{lstlisting}

\lstset{language=r,label= ,caption= ,captionpos=b,numbers=none}
\begin{lstlisting}
library(mvtnorm)
Sigma <- diag(s2.taylor*(1 - rho.taylor),2,2)+s2.taylor*rho.taylor
z <- exp(rmvnorm(1e5, mean = c(m0.taylor, m1.taylor), sigma = Sigma))
cor(z[,1],z[,2])

cov(z[,1],z[,2])
exp(m0.taylor+m1.taylor)*(rho.taylor*s2.taylor+0.5*rho.taylor^2*s2.taylor^2)

(rho.taylor+0.5*rho.taylor^2*s2.taylor)/(1+(3/2)*s2.taylor+(7/6)*s2.taylor^2+(11/24)*s2.taylor^3+(21/320)*s2.taylor^4)
\end{lstlisting}

\begin{verbatim}
[1] 0.5551681
[1] 0.09181254
[1] 0.08250216
[1] 0.5
\end{verbatim}

\subsection{Application: two independent groups}
\label{sec:orgd703435}

We consider two groups having a 10\% difference in their baseline value
(\(\alpha=1.15\)) and a variance of \(\sigma^2 = 0.15\). What are the
parameters of the corresponding normal distribution on the log-scale
and the standardized effect size?
\lstset{language=r,label= ,caption= ,captionpos=b,numbers=none}
\begin{lstlisting}
alpha <- 1.15
sigma2 <- 0.15
gamma <- 0.1
\end{lstlisting}

Solve the equations:
\begin{verbatim}
        a0         s0         a1         s1 
0.08802784 0.10608948 0.19175319 0.08851048
\end{verbatim}

We can check that \texttt{uniroot} converged correctly:
\lstset{language=r,label= ,caption= ,captionpos=b,numbers=none}
\begin{lstlisting}
c(exp(a0)*(1+s0/2) - alpha, 
  exp(2*a0)*(s0+s0^2*7/4) - sigma2, 
  exp(a1)*(1+s1/2) - alpha*(1+gamma), 
  exp(2*a1)*(s1+s1^2*7/4) - sigma2)
\end{lstlisting}

\begin{verbatim}
[1] -5.563198e-05  0.000000e+00 -1.895835e-05  0.000000e+00
\end{verbatim}

and the variables have the appropriate distribution:
\lstset{language=r,label= ,caption= ,captionpos=b,numbers=none}
\begin{lstlisting}
Z0 <- exp(rnorm(1e4, mean=a0, sd = sqrt(s0)))
Z1 <- exp(rnorm(1e4, mean=a1, sd = sqrt(s1)))
c(alpha = mean(Z0), 
  gamma = (mean(Z1)-mean(Z0))/mean(Z0), 
  sigma2 = var(Z0), 
  sigma2 = var(Z1))
\end{lstlisting}

\begin{verbatim}
    alpha     gamma    sigma2    sigma2 
1.1435272 0.1090391 0.1473705 0.1507638
\end{verbatim}

For a power calculation we would use:
\lstset{language=r,label= ,caption= ,captionpos=b,numbers=none}
\begin{lstlisting}
pwr.t.test(d = (a1-a0)/sqrt(s0/2+s1/2), sig.level = 0.05, power = 0.8)
## dvmisc::power_2t_unequal(n = 143, d = a1-a0, sigsq1 = s0, sigsq2 = s1, alpha = 0.05)
\end{lstlisting}

\begin{verbatim}
     Two-sample t test power calculation 

              n = 142.9312
              d = 0.3325282
      sig.level = 0.05
          power = 0.8
    alternative = two.sided

NOTE: n is number in *each* group
\end{verbatim}


\clearpage
\appendix
% \titleformat{\section}
% {\normalfont\Large\bfseries}{Appendix~\thesection}{1em}{}

\section{Moments of the normal distribution}
\label{SM:moments}
Denote \(X\) and \(Y\) two normally distributed variables, with mean
\(\mu_X\),\(\mu_Y\) and variance \(\sigma^2_X\),\(\sigma^2_Y\). Then:
\begin{itemize}
\item \(\Esp[X^2] = \sigma^2_X + \mu_X^2\)
\item \(\Esp[X^3] = 3 \mu_X \sigma^2_X + \mu_X^3\)
\item \(\Esp[X^4]=3\left(\sigma^2_X\right)^2 + 6 \sigma^2_X \mu_X^2 + \mu_X^4\)
\item \(\Esp[X^5]=15 \left(\sigma^2_X\right)^2 \mu + 10 \sigma^2_X \mu^3 + \mu^5\)
\item \(\Esp[(X-\mu_X)^6]= 15\left(\sigma_X^2\right)^3\)
\item \(\Esp[(X-\mu_X)^8]= 105\left(\sigma_X^2\right)^4\)
\end{itemize}
\bigskip

\begin{itemize}
\item \(\Cov[X^2,X]=2 \mu_X \sigma^2_X\)
\item \(\Cov[X^2,Y]=2 \mu_X \rho \sigma_X \sigma_Y\)
\item \(\Esp[X^2*Y^2] = (\sigma^2_X+\mu_X^2)(\sigma^2_Y+\mu_Y^2) + 2 \rho^2 \sigma^2_X \sigma^2_Y + 4 \rho \sigma_Y \sigma_X \mu_X \mu_Y\)
\item \(\Cov[\left(X-\mu_X\right)^2,\left(Y-\mu_Y\right)^2] = 2 \rho^2 \sigma^2_X \sigma^2_Y\)
\item \(\Cov[\left(X-\mu_X\right),\left(Y-\mu_Y\right)^3] = 3 \rho \sigma_X \sigma^3_Y\)
\item \(\Cov[\left(X-\mu_X\right)^3,\left(Y-\mu_Y\right)^3] = (6 \rho^3 + 9\rho) \sigma^3_X \sigma^3_Y\)
\end{itemize}
\clearpage

\section{Moments after transformation}
\label{sec:orgcfd4115}
\subsection{Recall: Taylor expansion for normally distributed variables}
\label{sec:org376b37f}

Taylor expansion for a smooth function \(f\) around the mean value \(\mu_Y=\Esp[Y]\):
\begin{align*}
f(Y) = f(\mu_Y) + f'(\mu_Y) (Y-\mu_Y) + \frac{1}{2} f''(\mu_Y) (Y-\mu_Y)^2  + \frac{1}{6} f'''(\mu_Y) (Y-\mu_Y)^3 + R_4(Y-\mu_Y)
\end{align*}
where \(R_4\) is a residual term. Introducing \(\bar{Y}=Y-\mu_Y\),
\(\sigma_Y^2 = \Var[Y]\) and using results for the moments of a normal
distribution (appendix \ref{SM:moments}), we have:
\begin{align*}
\Esp[f(Y)] \approx& f(\mu_Y) + f(\mu_Y) \Esp[\bar{Y}]  + \frac{1}{2} f''(\mu_Y) \Esp[\bar{Y}^2] + \frac{1}{6} f'''(\mu_Y) \Esp[\bar{Y}^3] 
= f(\mu_Y) + \frac{\sigma_Y^2}{2} f''(\mu_Y) \\
\Var[f(Y)] \approx& \left(f'(\mu_Y)\right)^2 \Var\left[\bar{Y}\right] + \frac{\left(f''(\mu_Y)\right)^2}{4} \Var\left[\bar{Y}^2\right]  + \frac{\left(f'''(\mu_Y)\right)^2}{36} \Var\left[\bar{Y}^3\right] \\
& +f'(\mu_Y) f''(\mu_Y)\Cov\left[\bar{Y},\bar{Y}^2\right] + \frac{f'(\mu_Y) f'''(\mu_Y)}{3} \Cov\left[\bar{Y},\bar{Y}^3\right] +\frac{f''(\mu_Y) f'''(\mu_Y)}{6}\Cov\left[\bar{Y}^2,\bar{Y}^3\right]\\
\approx& \left(f'(\mu_Y)\right)^2 \sigma_Y^2 + \frac{\left(f''(\mu_Y)\right)^2}{4} \left(3 \sigma_Y^4 - \sigma_Y^4\right)  + \frac{\left(f'''(\mu_Y)\right)^2}{36}  15 \sigma_Y^6
 + \frac{f'(\mu_Y) f'''(\mu_Y)}{3} 3 \sigma_Y^4 \\
\approx& \left(f'(\mu_Y)\right)^2 \sigma_Y^2 + \left(\frac{\left(f''(\mu_Y)\right)^2}{2} + f'(\mu_Y)f'''(\mu_Y)\right) \sigma_Y^4 + \frac{\left(f'''(\mu_Y)\right)^2}{36}  15 \sigma_Y^6 
\end{align*}
and introducing \(X\) with mean \(\mu_X\), variance \(\sigma_X^2\), and correlation \(\rho\) with \(Y\):
\begin{align*}
\Cov[f(X),f(Y)] \approx& f'(\mu_X) f'(\mu_Y) \Cov[X-\mu_X,Y-\mu_Y] \\ &+ \frac{1}{4} f''(\mu_X) f''(\mu_Y) \Cov[(X-\mu_X)^2,(Y-\mu_Y)^2]
\end{align*}

\Warning these approximations are precise when the higher order
moments are small (i.e. mean and variance are small). More precise
approximations can be obtained considering higher-order terms:
\begin{align*}
&\Esp[f(Y)] \approx f(\mu_Y) + \frac{\sigma_Y^2}{2} f^{(2)}(\mu_Y)  + \frac{\sigma_Y^4}{8} f^{(4)}(\mu_Y) + \frac{\sigma_Y^6}{48} f^{(6)}(\mu_Y) \\
&\Var[f(Y)] \approx \left(f^{(1)}(\mu_Y)\right)^2 \sigma_Y^2 + \left(\frac{\left(f^{(2)}(\mu_Y)\right)^2}{2} + f^{(1)}(\mu_Y)f^{(3)}(\mu_Y)\right) \sigma_Y^4 \\
                  &+ \left(\frac{5\left(f^{(3)}(\mu_Y)\right)^2}{12} + \frac{f^{(2)}(\mu_Y)f^{(4)}(\mu_Y)}{2}  + \frac{f^{(1)}(\mu_Y)f^{(5)}(\mu_Y)}{4} \right) \sigma_Y^6  \\
                  &+ \left(\frac{\left(f^{(4)}(\mu_Y)\right)^2}{6} + \frac{7 f^{(3)}(\mu_Y)f^{(5)}(\mu_Y)}{24} \right) \sigma_Y^8 + \frac{21\left(f^{(5)}(\mu_Y)\right)^2}{320} \sigma_Y^{10}
\end{align*}

\clearpage

\subsection{Application:  exponential transformation (\(f = \exp\))}
\label{SM:exptrans}
Using that \(\Cov[(X-\mu_X)^2,(Y-\mu_Y)^2]\approx2 \rho^2 \sigma_X^2 \sigma_Y^2\):
\begin{align*}
\Esp[\exp(Y)] &\approx \exp(\mu_Y)\left(1 + \frac{\sigma_Y^2}{2}\right) \\
\Var[\exp(Y)] &\approx \exp(2\mu_Y)\left(\sigma_Y^2 + \frac{3}{2} \sigma_Y^4 + \frac{15}{36} \sigma_Y^6 \right)\\
\Cov[\exp(X),\exp(Y)] &\approx \exp(\mu_X+\mu_Y)\left(\rho \sigma_X \sigma_Y + \frac{1}{2} \rho^2 \sigma_X^2 \sigma_Y^2\right) 
\end{align*}

Note: one can always go one order further to get a better approximation:
\begin{align*}
\Esp[\exp(Y)] &\approx \exp(\mu_Y)\left(1 + \frac{\sigma_Y^2}{2}+\frac{\sigma_Y^4}{8}+\frac{\sigma_Y^6}{48}\right) \\
\Var[\exp(Y)] &\approx \exp(2\mu_Y)\left(\sigma_Y^2 + \frac{3}{2} \sigma_Y^4 + \frac{7}{6} \sigma_Y^6 + \frac{11}{24} \sigma_Y^8 + \frac{21}{320} \sigma_Y^{10}\right) \\
\Cov[\exp(X),\exp(Y)] &\approx \exp(\mu_X+\mu_Y)\left(\rho \sigma_X \sigma_Y + \frac{1}{2} \rho^2 \sigma_X^2 \sigma_Y^2 \right.\\
& \left. + \frac{1}{2} \rho \left(\sigma_X \sigma_Y^3 + \sigma_Y \sigma_X^3 \right) + \frac{1}{12} \left(2 \rho^3 + 3 \rho\right) \sigma_X^3 \sigma_Y^3 \right)
\end{align*}

\textbf{Illustration}: We consider a normally distributed outcome with
expectation 1 and variance 0.5 (i.e standard deviation about 0.707). What is its expectation and variance
after exp-transformation?
\lstset{language=r,label= ,caption= ,captionpos=b,numbers=none}
\begin{lstlisting}
set.seed(10); n <- 1e4
mu <- 1; sigma2 <- 0.5

## first order method
mu.exp1 <- exp(mu)
var.exp1 <- exp(2*mu)*sigma2

## third order method
mu.exp2 <- exp(mu)*(1+sigma2/2)
var.exp2 <- exp(2*mu)*(sigma2 + (3/2)*sigma2^2 + (15/36)*sigma2^3)

## n order method
mu.exp3 <- exp(mu)*(1 + sigma2/2 + sigma2^2/8 + sigma2^3/48)
var.exp3 <- exp(2*mu)*(sigma2 + (3/2)*sigma2^2 + (7/6)*sigma2^3 + (11/24)*sigma2^4 + (21/320)*sigma2^10)

## empirical value
X.exp <- exp(rnorm(n, mean = mu, sd = sqrt(sigma2)))
mu.expGS <- mean(X.exp)
var.expGS <-  var(X.exp)
\end{lstlisting}

\clearpage

Comparison mean:
\lstset{language=r,label= ,caption= ,captionpos=b,numbers=none}
\begin{lstlisting}
rbind(value = c(first.order = mu.exp1, 
		second.order = mu.exp2, 
		third.order = mu.exp3, 
		truth = mu.expGS),
      bias = c(mu.exp1,mu.exp2,mu.exp3,mu.expGS)-mu.expGS,
      relative.bias = (c(mu.exp1,mu.exp2,mu.exp3,mu.expGS)-mu.expGS)/mu.expGS)
\end{lstlisting}

\begin{verbatim}
              first.order second.order  third.order    truth
value           2.7182818   3.39785229  3.489877452 3.505691
bias           -0.7874091  -0.10783859 -0.015813428 0.000000
relative.bias  -0.2246088  -0.03076101 -0.004510788 0.000000
\end{verbatim}

Comparison variance:
\lstset{language=r,label= ,caption= ,captionpos=b,numbers=none}
\begin{lstlisting}
rbind(value = c(first.order = var.exp1, 
		second.order = var.exp2, 
		third.order = var.exp3, 
		truth = var.expGS),
      bias = c(var.exp1,var.exp2,var.exp3,var.expGS)-var.expGS,
      relative.bias = (c(var.exp1,var.exp2,var.exp3,var.expGS)-var.expGS)/var.expGS)
\end{lstlisting}

\begin{verbatim}
              first.order second.order third.order    truth
value           3.6945280    6.8502708  7.75513398 8.224438
bias           -4.5299096   -1.3741669 -0.46930364 0.000000
relative.bias  -0.5507865   -0.1670834 -0.05706209 0.000000
\end{verbatim}

The second order estimate is much more accurate, especially for the
variance.

\bigskip

We now consider a bivariate normally distributed outcome with
expectation 0.1, variance 0.1, and correlation 0.5. What is the
correlation after exp-transformation?
\lstset{language=r,label= ,caption= ,captionpos=b,numbers=none}
\begin{lstlisting}
set.seed(10); n <- 1e4
mu <- c(0.1,0.1); sigma2 <- c(0.1,0.1); rho <- 0.5
Sigma <- matrix(c(sigma2[1], rho*sqrt(prod(sigma2)),
		  rho*sqrt(prod(sigma2)), sigma2[2]), 2,2)
XY <- mvtnorm::rmvnorm(n, mean = mu, sigma = Sigma)
X <- XY[,1] ; Y <- XY[,2]

cov(exp(X),exp(Y))
exp(mean(X)+2*mean(Y)) * (cor(X,Y)*sd(Y)*sd(X) + 0.5*cor(X,Y)^2*var(Y)*var(X))
\end{lstlisting}

\begin{verbatim}
[1] 0.06839007
[1] 0.06846545
\end{verbatim}


\clearpage

\subsection{Application:  log-transformation (\(f = \log\))}
\label{sec:orgcd3ff0e}

\begin{align*}
\Esp[\log(Y)] &\approx \log(\mu_Y) - \frac{\sigma_Y^2}{2\mu_Y^2} \\
\Var[\log(Y)] &\approx \frac{\sigma^2_Y}{\mu_Y^2} + \frac{5 \sigma^4_Y}{2\mu_Y^4} + \frac{5 \sigma^6_Y}{3\mu_Y^6} \\
\Cov[\log(X),\log(Y)] &\approx \frac{\rho \sigma_X \sigma_Y}{\mu_X\mu_Y} + \frac{\rho^2 \sigma^2_X \sigma^2_Y}{2\mu_X^2\mu_Y^2}
\end{align*}

Note: one can always go one order further to get a better approximation:
\begin{align*}
\Esp[\log(Y)] &\approx \log(\mu_Y) - \frac{\sigma_Y^2}{2\mu_Y^2} - \frac{3\sigma_Y^4}{4\mu_Y^4}  - \frac{5\sigma_Y^6}{2\mu_Y^6}  \\
\Var[\log(Y)] &\approx \frac{\sigma^2_Y}{\mu_Y^2} + \frac{5 \sigma^4_Y}{2\mu_Y^4} + \frac{67 \sigma^6_Y}{6\mu_Y^6} + \frac{20\sigma^8_Y}{6\mu_Y^8} + \frac{189\sigma^{10}_Y}{5\mu_Y^{10}} \\
\end{align*}

\textbf{Illustration}: We consider a normally distributed outcome with
expectation 7 and variance 2 (i.e standard deviation about
1.414). What is its expectation and variance after log-transformation?
\lstset{language=r,label= ,caption= ,captionpos=b,numbers=none}
\begin{lstlisting}
set.seed(10); n <- 1e4
mu <- 7; sigma2 <- 2

## first order method
mu.log1 <- log(mu)
var.log1 <- sigma2/mu^2

## third order method
mu.log2 <-  log(mu) - sigma2/(2*mu^2)
var.log2 <- sigma2/mu^2 + 5*sigma2^2/(2*mu^4) + 5*sigma2^3/(3*mu^6)

## n order method
mu.log3 <-  log(mu) - sigma2/(2*mu^2) - 3*sigma2^2/(4*mu^4) - 5*sigma2^6/(2*mu^6)
var.log3 <- sigma2/mu^2 + 5*sigma2^2/(2*mu^4) + 67*sigma2^3/(6*mu^6) + 20*sigma2^4/(6*mu^8) + 189*sigma2^5/(5*mu^10)

## empirical value
X.log <- log(rnorm(n, mean = mu, sd = sqrt(sigma2)))
mu.logGS <- mean(X.log)
var.logGS <-  var(X.log)
\end{lstlisting}

\clearpage

Comparison mean:
\lstset{language=r,label= ,caption= ,captionpos=b,numbers=none}
\begin{lstlisting}
rbind(value = c(first.order = mu.log1, 
		second.order = mu.log2, 
		third.order = mu.log3, 
		truth = mu.logGS),
      bias = c(mu.log1,mu.log2,mu.log3,mu.logGS)-mu.logGS,
      relative.bias = (c(mu.log1,mu.log2,mu.log3,mu.logGS)-mu.logGS)/mu.logGS)
\end{lstlisting}

\begin{verbatim}
              first.order second.order  third.order    truth
value          1.94591015 1.9255019858  1.922892529 1.924102
bias           0.02180784 0.0013996795 -0.001209777 0.000000
relative.bias  0.01133403 0.0007274455 -0.000628749 0.000000
\end{verbatim}

Comparison variance:
\lstset{language=r,label= ,caption= ,captionpos=b,numbers=none}
\begin{lstlisting}
rbind(value = c(first.order = var.log1, 
		second.order = var.log2, 
		third.order = var.log3, 
		truth = var.logGS),
      bias = c(var.log1,var.log2,var.log3,var.logGS)-var.logGS,
      relative.bias = (c(var.log1,var.log2,var.log3,var.logGS)-var.logGS)/var.logGS)
\end{lstlisting}

\begin{verbatim}
               first.order second.order   third.order      truth
value          0.040816327  0.045094589  0.0457541123 0.04632675
bias          -0.005510428 -0.001232166 -0.0005726425 0.00000000
relative.bias -0.118946995 -0.026597277 -0.0123609457 0.00000000
\end{verbatim}

The second order estimate is much more accurate, especially for the
variance.

\clearpage

\subsection{Log-normal distribution}
\label{sec:org35b76dc}

An alternative approach is to use a log-normal distribution. Random
variables with log normal distribution have their logarithm equal to a
specific value \(a\) and their standard deviation equal to a specific
value \(s\). So we want to get:
\begin{align*}
\alpha &= \exp(a_0 + \frac{1}{2} s_0^2) \\
\sigma^2 &= \exp(2*a_0 + s_0^2)*(\exp(s_0^2)-1) \\
\alpha (1+\gamma) &= \exp(a_1 + \frac{1}{2} s_1^2) \\
\sigma^2 &= \exp(2*a_1 + s_1^2)*(\exp(s_1^2)-1)
\end{align*}
So
\begin{align*}
s_0 &= \log\left(1+\frac{\sigma^2}{\alpha^2}\right)\\
a_0 &= \log(\alpha)-\frac{s_0^2}{2}\\
s_1 &= \log\left(1+\frac{\sigma^2}{\alpha*(1+\gamma)^2}\right)\\
a_1 &= \log(\alpha*(1+\gamma))-\frac{s_1^2}{2}
\end{align*}

\clearpage

\textbf{Illustration}: We consider a normally distributed outcome with
expectation 7 and variance 2 (i.e standard deviation about
1.414). What is its expectation and variance after log-transformation?
\lstset{language=r,label= ,caption= ,captionpos=b,numbers=none}
\begin{lstlisting}
set.seed(10); n <- 1e4
X <- rlnorm(1e4, mean=1, sd = 0.5)
## X <- exp(rnorm(1e4, mean=1, sd = sqrt(0.5)))

mu.exp <- mean(X)
sigma2.exp <- var(X)

## taylor expansion method
## mu.exp = exp(mu)*(1 + sigma2/2 + sigma2^2/8 + sigma2^3/48)
## sigma2.exp = exp(2*mu)*(sigma2 + (3/2)*sigma2^2 + (7/6)*sigma2^3 + (11/24)*sigma2^4 + (21/320)*sigma2^10)
getSigma2 <- function(sigma2){
    mu.exp^2/sigma2.exp - (1 + sigma2/2 + sigma2^2/8 + sigma2^3/48)^2/(sigma2 + (3/2)*sigma2^2 + (7/6)*sigma2^3 + (11/24)*sigma2^4 + (21/320)*sigma2^10)
}
var.taylor <- uniroot(f = getSigma2, lower = 1e-5, upper = sigma2.exp)$root
mu.taylor <- log(mu.exp/(1 + var.taylor/2 + var.taylor^2/8 + var.taylor^3/48))
## mu.taylor <-  log(mu) - sigma2/(2*mu^2) - 3*sigma2^2/(4*mu^4) - 5*sigma2^6/(2*mu^6)
## var.taylor <- sigma2/mu^2 + 5*sigma2^2/(2*mu^4) + 67*sigma2^3/(6*mu^6) + 20*sigma2^4/(6*mu^8) + 189*sigma2^5/(5*mu^10)

## log distribution method
var.logdist <- log(1+sigma2/mu^2)
mu.logdist <- log(mu) - var.logdist/2 

## empirical value
X.log <- log(X)
mu.logGS <- mean(X.log)
var.logGS <-  var(X.log)
\end{lstlisting}

Comparison mean:
\lstset{language=r,label= ,caption= ,captionpos=b,numbers=none}
\begin{lstlisting}
rbind(value = c(taylor = mu.taylor, 
		dist = mu.logdist, 
		truth = mu.logGS),
      bias = c(mu.taylor,mu.logdist,mu.logGS)-mu.logGS,
      relative.bias = (c(mu.taylor,mu.logdist,mu.logGS)-mu.logGS)/mu.logGS)
\end{lstlisting}

\begin{verbatim}
                    taylor         dist    truth
value          0.999612153  0.998213975 1.000669
bias          -0.001056824 -0.002455001 0.000000
relative.bias -0.001056117 -0.002453360 0.000000
\end{verbatim}

Comparison variance:
\lstset{language=r,label= ,caption= ,captionpos=b,numbers=none}
\begin{lstlisting}
rbind(value = c(taylor = var.taylor, 
		dist = var.logdist, 
		truth = var.logGS),
      bias = c(var.taylor,var.logdist,var.logGS)-var.logGS,
      relative.bias = (c(var.taylor,var.logdist,var.logGS)-var.logGS)/var.logGS)
\end{lstlisting}

\begin{verbatim}
                   taylor      dist     truth
value         0.255318149 0.5123473 0.2528091
bias          0.002509088 0.2595382 0.0000000
relative.bias 0.009924835 1.0266175 0.0000000
\end{verbatim}

\clearpage
\end{document}