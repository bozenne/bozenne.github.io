% Created 2023-02-09 to 13:47
% Intended LaTeX compiler: pdflatex
\documentclass[12pt]{article}

%%%% settings when exporting code %%%% 

\usepackage{listings}
\lstdefinestyle{code-small}{
backgroundcolor=\color{white}, % background color for the code block
basicstyle=\ttfamily\small, % font used to display the code
commentstyle=\color[rgb]{0.5,0,0.5}, % color used to display comments in the code
keywordstyle=\color{black}, % color used to highlight certain words in the code
numberstyle=\ttfamily\tiny\color{gray}, % color used to display the line numbers
rulecolor=\color{black}, % color of the frame
stringstyle=\color[rgb]{0,.5,0},  % color used to display strings in the code
breakatwhitespace=false, % sets if automatic breaks should only happen at whitespace
breaklines=true, % sets automatic line breaking
columns=fullflexible,
frame=single, % adds a frame around the code (non,leftline,topline,bottomline,lines,single,shadowbox)
keepspaces=true, % % keeps spaces in text, useful for keeping indentation of code
literate={~}{$\sim$}{1}, % symbol properly display via latex
numbers=none, % where to put the line-numbers; possible values are (none, left, right)
numbersep=10pt, % how far the line-numbers are from the code
showspaces=false,
showstringspaces=false,
stepnumber=1, % the step between two line-numbers. If it's 1, each line will be numbered
tabsize=1,
xleftmargin=0cm,
emph={anova,apply,class,coef,colnames,colNames,colSums,dim,dcast,for,ggplot,head,if,ifelse,is.na,lapply,list.files,library,logLik,melt,plot,require,rowSums,sapply,setcolorder,setkey,str,summary,tapply},
aboveskip = \medskipamount, % define the space above displayed listings.
belowskip = \medskipamount, % define the space above displayed listings.
lineskip = 0pt} % specifies additional space between lines in listings
\lstset{style=code-small}
%%%% packages %%%%%

\usepackage[utf8]{inputenc}
\usepackage[T1]{fontenc}
\usepackage{lmodern}
\usepackage{textcomp}
\usepackage{color}
\usepackage{graphicx}
\usepackage{grffile}
\usepackage{wrapfig}
\usepackage{rotating}
\usepackage{longtable}
\usepackage{multirow}
\usepackage{multicol}
\usepackage{changes}
\usepackage{pdflscape}
\usepackage{geometry}
\usepackage[normalem]{ulem}
\usepackage{amssymb}
\usepackage{amsmath}
\usepackage{amsfonts}
\usepackage{dsfont}
\usepackage{array}
\usepackage{ifthen}
\usepackage{hyperref}
\usepackage{natbib}
%
%%%% specifications %%%%
%
\usepackage{ifthen}
\usepackage{xifthen}
\usepackage{xargs}
\usepackage{xspace}
\newcommand\Rlogo{\textbf{\textsf{R}}\xspace} %
\RequirePackage{fancyvrb}
\DefineVerbatimEnvironment{verbatim}{Verbatim}{fontsize=\small,formatcom = {\color[rgb]{0.5,0,0}}}
\RequirePackage{colortbl} % arrayrulecolor to mix colors
\RequirePackage{setspace} % to modify the space between lines - incompatible with footnote in beamer
\renewcommand{\baselinestretch}{1.1}
\geometry{top=1cm}
\RequirePackage{colortbl} % arrayrulecolor to mix colors
\RequirePackage{pifont}
\RequirePackage{relsize}
\newcommand{\Cross}{{\raisebox{-0.5ex}%
{\relsize{1.5}\ding{56}}}\hspace{1pt} }
\newcommand{\Valid}{{\raisebox{-0.5ex}%
{\relsize{1.5}\ding{52}}}\hspace{1pt} }
\newcommand{\CrossR}{ \textcolor{red}{\Cross} }
\newcommand{\ValidV}{ \textcolor{green}{\Valid} }
\usepackage{stackengine}
\usepackage{scalerel}
\newcommand\Warning[1][3ex]{%
\renewcommand\stacktype{L}%
\scaleto{\stackon[1.3pt]{\color{red}$\triangle$}{\tiny\bfseries !}}{#1}%
\xspace
}
\hypersetup{
citecolor=[rgb]{0,0.5,0},
urlcolor=[rgb]{0,0,0.5},
linkcolor=[rgb]{0,0,0.5},
}
\RequirePackage{epstopdf} % to be able to convert .eps to .pdf image files
\RequirePackage{capt-of} %
\RequirePackage{caption} % newlines in graphics
\RequirePackage{enumitem} % to be able to convert .eps to .pdf image files
\definecolor{light}{rgb}{1, 1, 0.9}
\definecolor{lightred}{rgb}{1.0, 0.7, 0.7}
\definecolor{lightblue}{rgb}{0.0, 0.8, 0.8}
\newcommand{\darkblue}{blue!80!black}
\newcommand{\darkgreen}{green!50!black}
\newcommand{\darkred}{red!50!black}
\usepackage{mdframed}
\newcommand{\first}{1\textsuperscript{st} }
\newcommand{\second}{2\textsuperscript{nd} }
\newcommand{\third}{3\textsuperscript{rd} }
\RequirePackage{amsmath}
\RequirePackage{algorithm}
\RequirePackage[noend]{algpseudocode}
\RequirePackage{dsfont}
\RequirePackage{amsmath,stmaryrd,graphicx}
\RequirePackage{prodint} % product integral symbol (\PRODI)
\newcommand\defOperator[7]{%
\ifthenelse{\isempty{#2}}{
\ifthenelse{\isempty{#1}}{#7{#3}#4}{#7{#3}#4 \left#5 #1 \right#6}
}{
\ifthenelse{\isempty{#1}}{#7{#3}#4_{#2}}{#7{#3}#4_{#1}\left#5 #2 \right#6}
}
}
\newcommand\defUOperator[5]{%
\ifthenelse{\isempty{#1}}{
#5\left#3 #2 \right#4
}{
\ifthenelse{\isempty{#2}}{\underset{#1}{\operatornamewithlimits{#5}}}{
\underset{#1}{\operatornamewithlimits{#5}}\left#3 #2 \right#4}
}
}
\newcommand{\defBoldVar}[2]{
\ifthenelse{\equal{#2}{T}}{\boldsymbol{#1}}{\mathbf{#1}}
}
\newcommandx\Cov[2][1=,2=]{\defOperator{#1}{#2}{C}{ov}{\lbrack}{\rbrack}{\mathbb}}
\newcommandx\Esp[2][1=,2=]{\defOperator{#1}{#2}{E}{}{\lbrack}{\rbrack}{\mathbb}}
\newcommandx\Prob[2][1=,2=]{\defOperator{#1}{#2}{P}{}{\lbrack}{\rbrack}{\mathbb}}
\newcommandx\Qrob[2][1=,2=]{\defOperator{#1}{#2}{Q}{}{\lbrack}{\rbrack}{\mathbb}}
\newcommandx\Var[2][1=,2=]{\defOperator{#1}{#2}{V}{ar}{\lbrack}{\rbrack}{\mathbb}}
\newcommandx\Binom[2][1=,2=]{\defOperator{#1}{#2}{B}{}{(}{)}{\mathcal}}
\newcommandx\Gaus[2][1=,2=]{\defOperator{#1}{#2}{N}{}{(}{)}{\mathcal}}
\newcommandx\Wishart[2][1=,2=]{\defOperator{#1}{#2}{W}{ishart}{(}{)}{\mathcal}}
\newcommandx\Likelihood[2][1=,2=]{\defOperator{#1}{#2}{L}{}{(}{)}{\mathcal}}
\newcommandx\Information[2][1=,2=]{\defOperator{#1}{#2}{I}{}{(}{)}{\mathcal}}
\newcommandx\Score[2][1=,2=]{\defOperator{#1}{#2}{S}{}{(}{)}{\mathcal}}
\newcommandx\Vois[2][1=,2=]{\defOperator{#1}{#2}{V}{}{(}{)}{\mathcal}}
\newcommandx\IF[2][1=,2=]{\defOperator{#1}{#2}{IF}{}{(}{)}{\mathcal}}
\newcommandx\Ind[1][1=]{\defOperator{}{#1}{1}{}{(}{)}{\mathds}}
\newcommandx\Max[2][1=,2=]{\defUOperator{#1}{#2}{(}{)}{min}}
\newcommandx\Min[2][1=,2=]{\defUOperator{#1}{#2}{(}{)}{max}}
\newcommandx\argMax[2][1=,2=]{\defUOperator{#1}{#2}{(}{)}{argmax}}
\newcommandx\argMin[2][1=,2=]{\defUOperator{#1}{#2}{(}{)}{argmin}}
\newcommandx\cvD[2][1=D,2=n \rightarrow \infty]{\xrightarrow[#2]{#1}}
\newcommandx\Hypothesis[2][1=,2=]{
\ifthenelse{\isempty{#1}}{
\mathcal{H}
}{
\ifthenelse{\isempty{#2}}{
\mathcal{H}_{#1}
}{
\mathcal{H}^{(#2)}_{#1}
}
}
}
\newcommandx\dpartial[4][1=,2=,3=,4=\partial]{
\ifthenelse{\isempty{#3}}{
\frac{#4 #1}{#4 #2}
}{
\left.\frac{#4 #1}{#4 #2}\right\rvert_{#3}
}
}
\newcommandx\dTpartial[3][1=,2=,3=]{\dpartial[#1][#2][#3][d]}
\newcommandx\ddpartial[3][1=,2=,3=]{
\ifthenelse{\isempty{#3}}{
\frac{\partial^{2} #1}{\partial #2^2}
}{
\frac{\partial^2 #1}{\partial #2\partial #3}
}
}
\newcommand\Real{\mathbb{R}}
\newcommand\Rational{\mathbb{Q}}
\newcommand\Natural{\mathbb{N}}
\newcommand\trans[1]{{#1}^\intercal}%\newcommand\trans[1]{{\vphantom{#1}}^\top{#1}}
\newcommand{\independent}{\mathrel{\text{\scalebox{1.5}{$\perp\mkern-10mu\perp$}}}}
\newcommand\half{\frac{1}{2}}
\newcommand\normMax[1]{\left|\left|#1\right|\right|_{max}}
\newcommand\normTwo[1]{\left|\left|#1\right|\right|_{2}}
\date{\today}
\title{Mediation analysis}
\hypersetup{
 colorlinks=true,
 pdfauthor={},
 pdftitle={Mediation analysis},
 pdfkeywords={},
 pdfsubject={},
 pdfcreator={Emacs 27.2 (Org mode 9.5.2)},
 pdflang={English}
 }
\begin{document}

\maketitle
Denote by \(Y\) an oucome, \(M\) a mediator, \(E\) an exposure, and
\(C\) possible confounders (between \(Y\) and \(E\), or \(Y\) and
\(M\), or \(E\) and \(M\)).
\begin{itemize}
\item \(Y(e,m)=Y(e,m,c)\) will be the counterfactual outcome, i.e. the outcome
value had the exposure be set to \(e\), the mediator to \(m\), and the confounder set to \(c\).
\item \(M(e)=M(e,c)\) will be the counterfactual mediator, i.e. the mediator
value had the exposure be set to \(e\) and the confounder set to \(c\).
\end{itemize}

\bigskip

We define (all at a fixed confounder value \(c\)):
\begin{itemize}
\item the total effect as \(Y(e,M(e))-Y(e^*,M(e^*))\)
\item the natural direct effect as \(Y(e,M(e))-Y(e^*,M(e))\)
\item the indirect direct effect as \(Y(e^*,M(e))-Y(e^*,M(e^*))\)
\end{itemize}

\section{Binary outcome, continuous exposure}
\label{sec:org6ee127e}
\subsection{Theory}
\label{sec:org6eb5d4d}
We consider the following models:
\begin{align*}
\text{logit}\left(\Prob[Y=1|E,M,C]\right) &= \beta_0 + \beta_1 E + \beta_2 M + \beta_3 C \\
\Esp[M|E,C] &= \alpha_0 + \alpha_1 E + \alpha_3 C
\end{align*}

Therefore the counterfactual probability can be expressed as:
\begin{align*}
Y(e,M(e^*),c) &= \frac{1}{1+\exp^{-\beta_0 - \beta_1 e - \beta_2 M(e^*) + \beta_3 c}} \\
              &= \frac{1}{1+\exp^{-(\beta_0+\beta_2 \alpha_0) - (\beta_1 e + \beta_2 \alpha_1 e^*) + (\beta_3 + \beta_2 \alpha_3) c}} \\
\text{logit}\left(\Prob[Y(e,M(e^*),c)]\right) &= (\beta_0+\beta_2 \alpha_0) + (\beta_1 e + \beta_2 \alpha_1 e^*) + (\beta_3 + \beta_2 \alpha_3) c
\end{align*}

So the odd for the total effect is:
\[ OR^{TE} = \frac{\Prob[Y(e,M(e))]/\Prob[Y(e,M(e))]}{\Prob[Y(e^*,M(e^*))]/\Prob[Y(e^*,M(e^*))]} = \exp( (\beta_1 + \beta_2 \alpha_1)(e-e^*)) \]
for the natural direct effect:
\[ OR^{NDE} = \frac{\Prob[Y(e,M(e))]/\Prob[Y(e,M(e))]}{\Prob[Y(e^*,M(e))]/\Prob[Y(e^*,M(e))]} = \exp(\beta_1(e-e^*)) \]
for the natural indirect effect:
\[ OR^{NIE} = \frac{\Prob[Y(e^*,M(e))]/\Prob[Y(e^*,M(e))]}{\Prob[Y(e^*,M(e^*))]/\Prob[Y(e^*,M(e^*))]} = \exp(\beta_2\alpha_1(e-e^*)) \]
This matches formula in \cite{vanderweele2010odds}


\bigskip


Note that we can also express the total effect, natural direct effect,
and natural indirect effect on the probability scale, e.g.:
\begin{align*}
&\Prob[Y(e,M(e),c)=1] - \Prob[Y(e^*,M(e),c)=1] \\
&= \frac{1}{1+\exp^{-(\beta_0+\beta_2 \alpha_0) - (\beta_1 e + \beta_2 \alpha_1 e) + (\beta_3 + \beta_2 \alpha_3) c}}
- \frac{1}{1+\exp^{-(\beta_0+\beta_2 \alpha_0) - (\beta_1 e^* + \beta_2 \alpha_1 e) + (\beta_3 + \beta_2 \alpha_3) c}}
\end{align*}
which will be a function of the confounder value. To get a single
estimate we could average it out over the confounder distribution in
our population.

\subsection{Practice}
\label{sec:org0bb8102}

\lstset{language=r,label= ,caption= ,captionpos=b,numbers=none}
\begin{lstlisting}
library(medflex)
data(UPBdata)
head(UPBdata)
sum(is.na(UPBdata))
\end{lstlisting}

\begin{verbatim}
         att attbin attcat     negaff  initiator gender educ age UPB
1  1.0005617      1      M  0.8404610     myself      F    M  41   1
2 -0.7085889      0      L -1.2574650       both      M    M  42   0
3 -0.7085889      0      L -1.2022564       both      F    H  43   0
4  0.6061423      1      M -0.3741277 ex-partner      M    H  52   1
5  0.2117230      1      M  1.9446325 ex-partner      M    M  32   1
6  2.0523467      1      H -0.8157964 ex-partner      M    H  47   0
[1] 0
\end{verbatim}


Manually
\lstset{language=r,label= ,caption= ,captionpos=b,numbers=none}
\begin{lstlisting}
e.lm <- glm(negaff ~ factor(attbin) + gender + educ + age, data = UPBdata)
e.logit <- glm(UPB ~ attbin + negaff + gender + educ + age, data = UPBdata)
\end{lstlisting}

Using Medflex
\lstset{language=r,label= ,caption= ,captionpos=b,numbers=none}
\begin{lstlisting}
expData <- neWeight(negaff ~ factor(attbin) + gender + educ + age, family = gaussian, data = UPBdata)
neMod1 <- neModel(UPB ~ attbin0 + attbin1 + gender + educ + age, family = binomial("logit"), expData = expData)
\end{lstlisting}

\lstset{language=r,label= ,caption= ,captionpos=b,numbers=none}
\begin{lstlisting}
exp(cbind(estimate = coef(neMod1),confint(neMod1))[c("attbin01", "attbin11"), ])
\end{lstlisting}
\begin{verbatim}
         estimate  95% LCL  95% UCL
attbin01 1.485757 0.946311 2.294548
attbin11 1.421865 1.188970 1.678737
\end{verbatim}

\section{References}
\label{sec:org559036f}
\begingroup
\renewcommand{\section}[2]{}
\bibliographystyle{apalike}
\bibliography{bibliography}

\endgroup
\end{document}