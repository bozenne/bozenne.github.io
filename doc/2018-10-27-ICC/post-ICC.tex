% Created 2021-03-08 man 11:08
% Intended LaTeX compiler: pdflatex
\documentclass[12pt]{article}

%%%% settings when exporting code %%%% 

\usepackage{listings}
\lstdefinestyle{code-small}{
backgroundcolor=\color{white}, % background color for the code block
basicstyle=\ttfamily\small, % font used to display the code
commentstyle=\color[rgb]{0.5,0,0.5}, % color used to display comments in the code
keywordstyle=\color{black}, % color used to highlight certain words in the code
numberstyle=\ttfamily\tiny\color{gray}, % color used to display the line numbers
rulecolor=\color{black}, % color of the frame
stringstyle=\color[rgb]{0,.5,0},  % color used to display strings in the code
breakatwhitespace=false, % sets if automatic breaks should only happen at whitespace
breaklines=true, % sets automatic line breaking
columns=fullflexible,
frame=single, % adds a frame around the code (non,leftline,topline,bottomline,lines,single,shadowbox)
keepspaces=true, % % keeps spaces in text, useful for keeping indentation of code
literate={~}{$\sim$}{1}, % symbol properly display via latex
numbers=none, % where to put the line-numbers; possible values are (none, left, right)
numbersep=10pt, % how far the line-numbers are from the code
showspaces=false,
showstringspaces=false,
stepnumber=1, % the step between two line-numbers. If it's 1, each line will be numbered
tabsize=1,
xleftmargin=0cm,
emph={anova,apply,class,coef,colnames,colNames,colSums,dim,dcast,for,ggplot,head,if,ifelse,is.na,lapply,list.files,library,logLik,melt,plot,require,rowSums,sapply,setcolorder,setkey,str,summary,tapply},
aboveskip = \medskipamount, % define the space above displayed listings.
belowskip = \medskipamount, % define the space above displayed listings.
lineskip = 0pt} % specifies additional space between lines in listings
\lstset{style=code-small}
%%%% packages %%%%%

\usepackage[utf8]{inputenc}
\usepackage[T1]{fontenc}
\usepackage{lmodern}
\usepackage{textcomp}
\usepackage{color}
\usepackage{graphicx}
\usepackage{grffile}
\usepackage{wrapfig}
\usepackage{rotating}
\usepackage{longtable}
\usepackage{multirow}
\usepackage{multicol}
\usepackage{changes}
\usepackage{pdflscape}
\usepackage{geometry}
\usepackage[normalem]{ulem}
\usepackage{amssymb}
\usepackage{amsmath}
\usepackage{amsfonts}
\usepackage{dsfont}
\usepackage{array}
\usepackage{ifthen}
\usepackage{hyperref}
\usepackage{natbib}
\RequirePackage{setspace} % to modify the space between lines - incompatible with footnote in beamer
\renewcommand{\baselinestretch}{1.1}
\geometry{top=1cm}
\usepackage{titlesec}
\usepackage{etoolbox}
\makeatletter
\patchcmd{\ttlh@hang}{\parindent\z@}{\parindent\z@\leavevmode}{}{}
\patchcmd{\ttlh@hang}{\noindent}{}{}{}
\makeatother
\RequirePackage{colortbl} % arrayrulecolor to mix colors
\definecolor{myorange}{rgb}{1,0.2,0}
\definecolor{mypurple}{rgb}{0.7,0,8}
\definecolor{mycyan}{rgb}{0,0.6,0.6}
\newcommand{\lightblue}{blue!50!white}
\newcommand{\darkblue}{blue!80!black}
\newcommand{\darkgreen}{green!50!black}
\newcommand{\darkred}{red!50!black}
\definecolor{gray}{gray}{0.5}
\hypersetup{
citecolor=[rgb]{0,0.5,0},
urlcolor=[rgb]{0,0,0.5},
linkcolor=[rgb]{0,0,0.5},
}
\newenvironment{comment}{\small \color{gray}\fontfamily{lmtt}\selectfont}{\par}
\newenvironment{activity}{\color{orange}\fontfamily{qzc}\selectfont}{\par}
\RequirePackage{pifont}
\RequirePackage{relsize}
\newcommand{\Cross}{{\raisebox{-0.5ex}%
{\relsize{1.5}\ding{56}}}\hspace{1pt} }
\newcommand{\Valid}{{\raisebox{-0.5ex}%
{\relsize{1.5}\ding{52}}}\hspace{1pt} }
\newcommand{\CrossR}{ \textcolor{red}{\Cross} }
\newcommand{\ValidV}{ \textcolor{green}{\Valid} }
\usepackage{stackengine}
\usepackage{scalerel}
\newcommand\Warning[1][3ex]{%
\renewcommand\stacktype{L}%
\scaleto{\stackon[1.3pt]{\color{red}$\triangle$}{\tiny\bfseries !}}{#1}%
\xspace
}
\newcommand\Rlogo{\textbf{\textsf{R}}\xspace} %
\RequirePackage{fancyvrb}
\DefineVerbatimEnvironment{verbatim}{Verbatim}{fontsize=\small,formatcom = {\color[rgb]{0.5,0,0}}}
\lstdefinestyle{code-tiny}{basicstyle=\ttfamily\footnotesize}
\RequirePackage{enumitem} % better than enumerate
\RequirePackage{epstopdf} % to be able to convert .eps to .pdf image files
\RequirePackage{capt-of} %
\RequirePackage{caption} % newlines in graphics
\RequirePackage{tikz-cd} % graph
\RequirePackage{booktabs} % for nice lines in table (e.g. toprule, bottomrule, midrule, cmidrule)
\RequirePackage{amsmath}
\RequirePackage{algorithm}
\RequirePackage[noend]{algpseudocode}
\RequirePackage{dsfont}
\RequirePackage{amsmath,stmaryrd,graphicx}
\RequirePackage{prodint} % product integral symbol (\PRODI)
\usepackage{ifthen}
\usepackage{xifthen}
\usepackage{xargs}
\usepackage{xspace}
\newcommand\defOperator[7]{%
\ifthenelse{\isempty{#2}}{
\ifthenelse{\isempty{#1}}{#7{#3}#4}{#7{#3}#4 \left#5 #1 \right#6}
}{
\ifthenelse{\isempty{#1}}{#7{#3}#4_{#2}}{#7{#3}#4_{#1}\left#5 #2 \right#6}
}
}
\newcommand\defUOperator[5]{%
\ifthenelse{\isempty{#1}}{
#5\left#3 #2 \right#4
}{
\ifthenelse{\isempty{#2}}{\underset{#1}{\operatornamewithlimits{#5}}}{
\underset{#1}{\operatornamewithlimits{#5}}\left#3 #2 \right#4}
}
}
\newcommand{\defBoldVar}[2]{
\ifthenelse{\equal{#2}{T}}{\boldsymbol{#1}}{\mathbf{#1}}
}
\newcommandx\Esp[2][1=,2=]{\defOperator{#1}{#2}{E}{}{\lbrack}{\rbrack}{\mathbb}}
\newcommandx\Prob[2][1=,2=]{\defOperator{#1}{#2}{P}{}{\lbrack}{\rbrack}{\mathbb}}
\newcommandx\Qrob[2][1=,2=]{\defOperator{#1}{#2}{Q}{}{\lbrack}{\rbrack}{\mathbb}}
\newcommandx\Var[2][1=,2=]{\defOperator{#1}{#2}{V}{ar}{\lbrack}{\rbrack}{\mathbb}}
\newcommandx\Cov[2][1=,2=]{\defOperator{#1}{#2}{C}{ov}{\lbrack}{\rbrack}{\mathbb}}
\newcommandx\Binom[2][1=,2=]{\defOperator{#1}{#2}{B}{}{(}{)}{\mathcal}}
\newcommandx\Gaus[2][1=,2=]{\defOperator{#1}{#2}{N}{}{(}{)}{\mathcal}}
\newcommandx\Wishart[2][1=,2=]{\defOperator{#1}{#2}{W}{ishart}{(}{)}{\mathcal}}
\newcommandx\Likelihood[2][1=,2=]{\defOperator{#1}{#2}{L}{}{(}{)}{\mathcal}}
\newcommandx\logLikelihood[2][1=,2=]{\defOperator{#1}{#2}{\ell}{}{(}{)}{}}
\newcommandx\Information[2][1=,2=]{\defOperator{#1}{#2}{I}{}{(}{)}{\mathcal}}
\newcommandx\Score[2][1=,2=]{\defOperator{#1}{#2}{S}{}{(}{)}{\mathcal}}
\newcommandx\Vois[2][1=,2=]{\defOperator{#1}{#2}{V}{}{(}{)}{\mathcal}}
\newcommandx\IF[2][1=,2=]{\defOperator{#1}{#2}{IF}{}{(}{)}{\mathcal}}
\newcommandx\Ind[1][1=]{\defOperator{}{#1}{1}{}{(}{)}{\mathds}}
\newcommandx\Max[2][1=,2=]{\defUOperator{#1}{#2}{(}{)}{min}}
\newcommandx\Min[2][1=,2=]{\defUOperator{#1}{#2}{(}{)}{max}}
\newcommandx\argMax[2][1=,2=]{\defUOperator{#1}{#2}{(}{)}{argmax}}
\newcommandx\argMin[2][1=,2=]{\defUOperator{#1}{#2}{(}{)}{argmin}}
\newcommandx\cvD[2][1=D,2=n \rightarrow \infty]{\xrightarrow[#2]{#1}}
\newcommandx\Hypothesis[2][1=,2=]{
\ifthenelse{\isempty{#1}}{
\mathcal{H}
}{
\ifthenelse{\isempty{#2}}{
\mathcal{H}_{#1}
}{
\mathcal{H}^{(#2)}_{#1}
}
}
}
\newcommandx\dpartial[4][1=,2=,3=,4=\partial]{
\ifthenelse{\isempty{#3}}{
\frac{#4 #1}{#4 #2}
}{
\left.\frac{#4 #1}{#4 #2}\right\rvert_{#3}
}
}
\newcommandx\dTpartial[3][1=,2=,3=]{\dpartial[#1][#2][#3][d]}
\newcommandx\ddpartial[3][1=,2=,3=]{
\ifthenelse{\isempty{#3}}{
\frac{\partial^{2} #1}{\partial #2^2}
}{
\frac{\partial^2 #1}{\partial #2\partial #3}
}
}
\newcommand\Real{\mathbb{R}}
\newcommand\Rational{\mathbb{Q}}
\newcommand\Natural{\mathbb{N}}
\newcommand\trans[1]{{#1}^\intercal}%\newcommand\trans[1]{{\vphantom{#1}}^\top{#1}}
\newcommand{\independent}{\mathrel{\text{\scalebox{1.5}{$\perp\mkern-10mu\perp$}}}}
\newcommand\half{\frac{1}{2}}
\newcommand\normMax[1]{\left|\left|#1\right|\right|_{max}}
\newcommand\normTwo[1]{\left|\left|#1\right|\right|_{2}}
\newcommand\Veta{\boldsymbol{\eta}}
\newcommand\VX{\mathbf{X}}
\author{Brice Ozenne}
\date{\today}
\title{Different types of ICC in R}
\hypersetup{
 colorlinks=true,
 pdfauthor={Brice Ozenne},
 pdftitle={Different types of ICC in R},
 pdfkeywords={},
 pdfsubject={},
 pdfcreator={Emacs 27.0.50 (Org mode 9.0.4)},
 pdflang={English}
 }
\begin{document}

\maketitle
In this document, we will consider measurements done at several
sessions over several individuals. We assume no missing data, i.e.,
all individuals have measurements for all sessions. The question we
would like to answer is what is the reliability of the measurements?

\begingroup
\renewcommand{\section}[2]{}
\tableofcontents
\endgroup

\clearpage

Recommanded references: 
\begin{itemize}
\item Kenneth O. McGraw and S. P. Wong \textbf{Forming Inferences About Some Intraclass Correlation Coefficients}, Psychological Methods, 1996, Vol. l,No. 1,30-46
\item Lawrence Lin, A.S. Hedayat, Wenting Wu, \textbf{Statistical Tools for Measuring Agreement}, 2012, Springer.
\end{itemize}

\clearpage

\section{Preparation}
\label{sec:orgba9e85f}

\lstset{style=code-tiny}

\subsection{Load R packages}
\label{sec:org1b70b00}
\lstset{language=r,label= ,caption= ,captionpos=b,numbers=none}
\begin{lstlisting}
library(data.table)
library(lava)
library(lme4)
library(nlme)
library(irr)
library(psych)
\end{lstlisting}



\subsection{Data generation}
\label{sec:org1494f2b}
\lstset{language=r,label= ,caption= ,captionpos=b,numbers=none}
\begin{lstlisting}
n.id <- 10
n.scan <- 2
n.rater <- 5

set.seed(10)
m <- lvm(c(Ytest,Yretest) ~ 2*eta)
latent(m) <- ~eta
distribution(m, ~Ytest) <- lava::gaussian.lvm(mean = 1, sd = 1)
distribution(m, ~Yretest) <- lava::gaussian.lvm(mean = 1, sd = 1)

dtW.icc1 <- as.data.table(lava::sim(m, n.id))
dtW.icc1[, eta := NULL]
dtW.icc1[, Id := as.character(1:.N)]
dtL.icc1 <- melt(dtW.icc1, id.vars = "Id", value.name = "Y", variable.name = "session")

dfW.icc1 <- as.data.frame(dtW.icc1)
dfL.icc1 <- as.data.frame(dtL.icc1)

mm <- lvm(c(Y1,Y2,Y3,Y4,Y5) ~ 2*eta)
latent(mm) <- ~eta
distribution(mm, ~Y1) <- lava::gaussian.lvm(mean = 1, sd = 1)
distribution(mm, ~Y2) <- lava::gaussian.lvm(mean = 1, sd = 1)
distribution(mm, ~Y3) <- lava::gaussian.lvm(mean = 1, sd = 1)
distribution(mm, ~Y4) <- lava::gaussian.lvm(mean = 1, sd = 1)
distribution(mm, ~Y5) <- lava::gaussian.lvm(mean = 1, sd = 1)

dtW.icc2 <- as.data.table(lava::sim(mm, n.id))
dtW.icc2[, eta := NULL]
dtW.icc2[, Id := as.character(1:.N)]
dtL.icc2 <- melt(dtW.icc2, id.vars = "Id", value.name = "Y", variable.name = "session")

dfW.icc2 <- as.data.frame(dtW.icc2)
dfL.icc2 <- as.data.frame(dtL.icc2)
\end{lstlisting}


\clearpage

\section{ICC1: fixed number of sessions assuming no session effect}
\label{sec:orgf598e5e}

Using ICC1, we assess the absolute agreement among measurements
repeated over one random factor (here patient identity) assuming no
other effect. The number of sessions is fixed meaning that it is
precisely those sessions that are interesting. 

\bigskip

\emph{Example:} we want to plan a study to assess treatment efficacy with
two sessions: baseline and post-treatment. EEG is used to measure
treatment efficacy. Prior to this study, another study is performed to
assessed the stability of the EEG measurement between two sessions. We
don't expect systematic difference in EEG signal between the sessions.

\bigskip

ICC1 is equivalent to the following mixed model:

\begin{equation}
 Y_{i,t} = \alpha + u_{i} + \varepsilon_{i,t} 
\end{equation}

\begin{tabular}{ll}
where & \(u_{i} \sim \Gaus[0,\tau^2] \)  \\
& \(\varepsilon_{i,t} \sim \Gaus[0,\sigma^2] \) \\
\end{tabular}

\begin{align}
 ICC^{single}_{1,agreement} &= \frac{\tau^2}{\tau^2 + \sigma^2} \\
 ICC^{average}_{1,agreement} &= \frac{\tau^2}{\tau^2 + \frac{\sigma^2}{n}}
\end{align}

\subsection{using mixed models}
\label{sec:org2225593}
\lstset{language=r,label= ,caption= ,captionpos=b,numbers=none}
\begin{lstlisting}
lme_ICC1 <- lmer(Y ~ 1 + (1|Id), data = dtL.icc1)

sigma_id <- attr(VarCorr(lme_ICC1)[[1]],"stddev")
sigma_error <- attr(VarCorr(lme_ICC1),"sc")

ICC1.lme <- c(
    single = as.double(sigma_id^2/(sigma_error^2+sigma_id^2)),
    average = as.double(sigma_id^2/(sigma_error^2/n.scan+sigma_id^2))
)

ICC1.lme
\end{lstlisting}

\begin{verbatim}
   single   average 
0.5191614 0.6834842
\end{verbatim}

\subsection{using generalized least squares}
\label{sec:orgce679be}
\lstset{language=r,label= ,caption= ,captionpos=b,numbers=none}
\begin{lstlisting}
gls_ICC1 <- gls(Y ~ 1, correlation = corCompSymm(form = ~ 1|Id), 
		data = dtL.icc1)

## correlation matrix
unclass(cov2cor(getVarCov(gls_ICC1)))
\end{lstlisting}

\begin{verbatim}
          [,1]      [,2]
[1,] 1.0000000 0.5191607
[2,] 0.5191607 1.0000000
\end{verbatim}

\subsection{using anova}
\label{sec:org3a76f51}
\lstset{language=r,label= ,caption= ,captionpos=b,numbers=none}
\begin{lstlisting}
aov_ICC1 <- aov(Y ~ Error(Id), data = dtL.icc1) 

MSB <- summary(aov_ICC1)[["Error: Id"]][[1]]["Mean Sq"]
MSW <- summary(aov_ICC1)[["Error: Within"]][[1]]["Mean Sq"]

ICC1.aov <-  c(
    single = as.double((MSB - MSW)/(MSB + (n.scan-1)*MSW)),
    average = as.double((MSB - MSW)/MSB)
)
ICC1.aov
\end{lstlisting}

\begin{verbatim}
   single   average 
0.5191614 0.6834842
\end{verbatim}

\subsection{using existing packages}
\label{sec:orge0de3a2}
\lstset{language=r,label= ,caption= ,captionpos=b,numbers=none}
\begin{lstlisting}
tt.single <- c(irr = irr::icc(dfW.icc1[,c("Ytest", "Yretest")],
			      model="oneway", type="consistency")$value,
	       psych = psych::ICC(dfW.icc1[,c("Ytest", "Yretest")])[[1]]$ICC[1])

tt.average <- c(irr = irr::icc(dfW.icc1[,c("Ytest", "Yretest")],
			       model="oneway", type="consistency", unit = "average")$value,
		psych = psych::ICC(dfW.icc1[,c("Ytest", "Yretest")])[[1]]$ICC[4])

rbind(single = tt.single, average = tt.average)
\end{lstlisting}

\begin{verbatim}
              irr     psych
single  0.5191614 0.5191609
average 0.6834842 0.6834838
\end{verbatim}

\clearpage

\section{ICC3: fixed number of sessions accounting for a session effect}
\label{sec:org836ea13}

Using ICC3, we assess the absolute agreement among measurements
repeated over one random factor (here patient identity) accounting for
a possible session effect. The number of sessions is fixed meaning
that it is precisely those sessions that are interesting.

\bigskip

\emph{Example:} we want to plan a study (called main study) to assess
treatment efficacy with two sessions: baseline and post-treatment. EEG
is used to measure treatment efficacy. Prior to this study, another
study is performed to assessed the stability of the EEG measurement
between two sessions. Unfortunately the two measurements were not
performed using the same EEG machine so we suspect a systematic
difference in EEG signal between the sessions. The investigator will
make sure that this won't happen in the main study. Therefore the
systematic difference is not inherant to the EEG technic (i.e. won't
be observed in the main study).

\bigskip

\emph{Note:}  In test
re-test analysis this should not be the case since we replicate a
measurement under the same conditions.

\bigskip
\bigskip

Using ICC3, we assess the absolute agreement among measurements
repeated over one random factor account for a possible session
effect. Compared to ICC1, this will lead to:
\begin{itemize}
\item a lower \(\sigma^2\) (some of the residual variance is explained)
\item a higher \(\tau\) (individual measurements are better correlated since
there is less arbitrary variation between them)
\item a higher ICC
\end{itemize}

\bigskip

ICC3 is equivalent to the following mixed model:

\begin{equation}
 Y_{i,t} = \alpha + \beta_t t + u_{i} + \varepsilon_{i,t} 
\end{equation}

\begin{tabular}{ll}
where & \(u_{i} \sim \Gaus[0,\tau^2] \)  \\
& \(t\) is treated as a categorical variable \\
& \(\varepsilon_{i,t} \sim \Gaus[0,\sigma^2] \) \\
\end{tabular}

\begin{align}
 ICC^{single}_{1,agreement} &= \frac{\tau^2}{\tau^2 + \sigma^2} \\
 ICC^{average}_{1,agreement} &= \frac{\tau^2}{\tau^2 + \frac{\sigma^2}{n}}
\end{align}

\clearpage

\subsection{using mixed models}
\label{sec:org20de13e}
\lstset{language=r,label= ,caption= ,captionpos=b,numbers=none}
\begin{lstlisting}
lme_ICC2 <- lmer(Y ~ session + (1|Id), data = dtL.icc1)

sigma_id <- attr(VarCorr(lme_ICC2)[[1]],"stddev")
sigma_error <- attr(VarCorr(lme_ICC2),"sc")

ICC2.lme <- c(
    single = as.double(sigma_id^2/(sigma_error^2+sigma_id^2)),
    average = as.double(sigma_id^2/(sigma_error^2/n.scan+sigma_id^2))
)

ICC2.lme
\end{lstlisting}

\begin{verbatim}
   single   average 
0.7495312 0.8568366
\end{verbatim}


\subsection{using generalized least squares}
\label{sec:org18b2737}
\lstset{language=r,label= ,caption= ,captionpos=b,numbers=none}
\begin{lstlisting}
gls_ICC2 <- gls(Y ~ session, correlation = corCompSymm(form = ~ 1|Id), data = dtL.icc1)

# correlation matrix
unclass(cov2cor(getVarCov(gls_ICC2)))
\end{lstlisting}

\begin{verbatim}
          [,1]      [,2]
[1,] 1.0000000 0.7495308
[2,] 0.7495308 1.0000000
\end{verbatim}


\subsection{using anova}
\label{sec:org7fc92bc}
\lstset{language=r,label= ,caption= ,captionpos=b,numbers=none}
\begin{lstlisting}
aov_ICC2 <- aov(Y ~ session + Error(Id), data = dtL.icc1) 

MSB <- summary(aov_ICC2)[["Error: Id"]][[1]]["Mean Sq"]
MSW <- summary(aov_ICC2)[["Error: Within"]][[1]]["Residuals","Mean Sq"]

ICC2.aov <-  c(
    single = as.double((MSB - MSW)/(MSB + (n.scan-1)*MSW)),
    average = as.double((MSB - MSW)/MSB)
)

ICC2.aov
\end{lstlisting}

\begin{verbatim}
   single   average 
0.7495312 0.8568366
\end{verbatim}


\subsection{using existing packages}
\label{sec:org3c1fbf4}
\lstset{language=r,label= ,caption= ,captionpos=b,numbers=none}
\begin{lstlisting}
# single
cat("single \n")
psych::ICC(dfW.icc1[,c("Ytest", "Yretest")])[[1]]$ICC[3]

# average
cat("average \n")
psych::ICC(dfW.icc1[,c("Ytest", "Yretest")])[[1]]$ICC[6]
\end{lstlisting}

\begin{verbatim}
single
[1] 0.7495307
average
[1] 0.8568363
\end{verbatim}

\section{Heterogenous variance between sessions}
\label{sec:orgac12914}

The previous ICC are not corrected for a possible difference in
variance between sessions:
\lstset{language=r,label= ,caption= ,captionpos=b,numbers=none}
\begin{lstlisting}
gls_heteroV <- gls(Y ~ session, 
		   weights = varIdent(form = ~ 1|session),
		   correlation = corCompSymm(form = ~ 1|Id), 
		   data = dtL.icc1)

# correlation matrix
unclass(cov2cor(getVarCov(gls_heteroV)))
\end{lstlisting}

\begin{verbatim}
         [,1]     [,2]
[1,] 1.000000 0.763223
[2,] 0.763223 1.000000
\end{verbatim}

In test re-test analysis this should not be the case since we
replicate a measurement under the same conditions.

\clearpage


\section{ICC2: random number of sessions}
\label{sec:org1b8d2d6}

Using ICC2, we assess the absolute agreement among measurements
repeated over two random factors, e.g. patient identity and
sessions. We don't account for a possible session effect. The number
of sessions is random meaning that we are not precisely interested in
those sessions. A more realistic example would be to consider raters
instead of sessions.

\bigskip

\emph{Example:} we want to study the reliability of the grades given by
teachers. We include 10 teachers that will grade 20 students. We are
not interested in those specific teachers since further studies may be
performed in other schools with other teachers.

\bigskip

ICC2 is equivalent to the following mixed model:
\begin{equation}
Y_{i,j,t} = \alpha + u_{i} + v_{j} + \varepsilon_{i,j,t}
\end{equation}

\begin{tabular}{ll}
where & \(u_{i} \sim \Gaus[0,\tau^2] \) \\
 &  \(v_{j} \sim \Gaus[0,\delta^2] \) \\
 & \(\varepsilon_{i,t} \sim \Gaus[0,\sigma^2] \) \\
\end{tabular}

\begin{align}
 ICC_{2,agreement} &= \frac{\tau^2}{\tau^2 + \sigma^2} \text{ or } ICC_{2,consistency} = \frac{\tau^2}{\tau^2 + \delta^2 + \sigma^2} \\
 ICC^{average}_{2,agreement} &= \frac{\tau^2}{\tau^2 + \frac{\sigma^2}{n}} \text{ or } ICC^{average}_{2,consistency} = \frac{\tau^2}{\tau^2 + \frac{1}{n}(\delta^2 + \sigma^2)}
\end{align}


\subsection{using mixed models}
\label{sec:org5be25e1}
\lstset{language=r,label= ,caption= ,captionpos=b,numbers=none}
\begin{lstlisting}
lme_ICC2r <- lmer(Y ~ 1 + (1|session)+ (1|Id), data = dtL.icc2)

sigma_id <- attr(VarCorr(lme_ICC2r)[[1]],"stddev")
sigma_scan <- attr(VarCorr(lme_ICC2r)[[2]],"stddev")
sigma_error <- attr(VarCorr(lme_ICC2r),"sc")

ICC2r.lme <- c(
    consistency.single = as.double(sigma_id^2/(sigma_error^2+sigma_id^2)),
    agreement.single = as.double(sigma_id^2/(sigma_error^2+sigma_scan^2+sigma_id^2)),
    consistency.average = as.double(sigma_id^2/(sigma_error^2/n.rater+sigma_id^2)),
    agreement.average = as.double(sigma_id^2/(sigma_error^2/n.rater+sigma_scan^2/n.rater+sigma_id^2))
)
ICC2r.lme
\end{lstlisting}

\begin{verbatim}
consistency.single    agreement.single consistency.average   agreement.average 
         0.8057243           0.8001868           0.9539947           0.9524339
\end{verbatim}

\clearpage

\subsection{using generalized least squares}
\label{sec:org18a1437}
Don't know how to specify non-nested random effects

\subsection{using anova}
\label{sec:orgf330dea}
\lstset{language=r,label= ,caption= ,captionpos=b,numbers=none}
\begin{lstlisting}
aov_ICC2r <- aov(Y ~ Error(session + Id), data = dtL.icc2) 

MSB <- summary(aov_ICC2r)[["Error: Id"]][[1]]["Mean Sq"]
MSscan <- summary(aov_ICC2r)[["Error: session"]][[1]]["Residuals","Mean Sq"]
MSW <- summary(aov_ICC2r)[["Error: Within"]][[1]]["Residuals","Mean Sq"]

ICC2r.aov <-  c(
    consistency.single = as.double((MSB - MSW)/(MSB + (n.rater-1)*MSW)),
    agreement.single = as.double((MSB - MSW)/(MSB + (n.rater-1)*MSW + n.rater * (MSscan-MSW)/n.id)),
    consistency.average = as.double((MSB - MSW)/MSB),
    agreement.average = as.double((MSB - MSW)/(MSB + (MSscan-MSW)/n.id))
)
ICC2r.aov
\end{lstlisting}

\begin{verbatim}
consistency.single    agreement.single consistency.average   agreement.average 
         0.8057232           0.8001849           0.9539944           0.9524334
\end{verbatim}

\subsection{using existing packages}
\label{sec:orgae24fdf}
\lstset{language=r,label= ,caption= ,captionpos=b,numbers=none}
\begin{lstlisting}
tt.single <- c(
    iccC = irr::icc(dfW.icc2[,c("Y1", "Y2", "Y3", "Y4", "Y5")],
		    model="twoway", type="consistency")$value,
    irrA = irr::icc(dfW.icc2[,c("Y1", "Y2", "Y3", "Y4", "Y5")],
		    model="twoway", type="agreement")$value,
    psych = psych::ICC(dfW.icc2[,c("Y1", "Y2", "Y3", "Y4", "Y5")])[[1]]$ICC[3:2]
)

tt.average <- c(
    iccC = irr::icc(dfW.icc2[,c("Y1", "Y2", "Y3", "Y4", "Y5")],
		    model="twoway", type="consistency", unit = "average")$value,
    iccA = irr::icc(dfW.icc2[,c("Y1", "Y2", "Y3", "Y4", "Y5")],
		    model="twoway", type="agreement", unit = "average")$value,
    psych = psych::ICC(dfW.icc2[,c("Y1", "Y2", "Y3", "Y4", "Y5")])[[1]]$ICC[6:5]
)
rbind(single = tt.single, average = tt.average)
\end{lstlisting}

\begin{verbatim}
             iccC      irrA    psych1    psych2
single  0.8057232 0.8001849 0.8057243 0.8001868
average 0.9539944 0.9524334 0.9539947 0.9524339
\end{verbatim}

\clearpage

\section{Different correlation between sessions}
\label{sec:orgd8b384c}

When considering more than 2 sessions, the previous ICC assume the
same correlation between sessions (or raters). This can be relaxed
using an unstructured covariance matrix:

\lstset{language=r,label= ,caption= ,captionpos=b,numbers=none,otherkeywords={}, deletekeywords={}}
\begin{lstlisting}
gls_heteroC <- gls(Y ~ session,
		   correlation = corSymm(form = ~ 1|Id),
		   data = dtL.icc2)

# correlation matrix
M.icc <- unclass(cov2cor(getVarCov(gls_heteroC)))
print(M.icc)
\end{lstlisting}

\begin{verbatim}
          [,1]      [,2]      [,3]      [,4]      [,5]
[1,] 1.0000000 0.8257161 0.6932362 0.8742870 0.7906416
[2,] 0.8257161 1.0000000 0.7640218 0.7507233 0.8421197
[3,] 0.6932362 0.7640218 1.0000000 0.5166736 0.6906414
[4,] 0.8742870 0.7507233 0.5166736 1.0000000 0.5647855
[5,] 0.7906416 0.8421197 0.6906414 0.5647855 1.0000000
\end{verbatim}

Note that the mean ICC:
\lstset{language=r,label= ,caption= ,captionpos=b,numbers=none,otherkeywords={}, deletekeywords={}}
\begin{lstlisting}
c(mean = mean(setdiff(unique(M.icc),1)), 
  indiv = apply(M.icc, 1, function(x){mean(setdiff(x,1))}))
\end{lstlisting}

\begin{verbatim}
     mean    indiv1    indiv2    indiv3    indiv4    indiv5 
0.7312846 0.7959702 0.7956452 0.6661432 0.6766173 0.7220470
\end{verbatim}

does not match the "normal" ICC:
\lstset{language=r,label= ,caption= ,captionpos=b,numbers=none,otherkeywords={}, deletekeywords={}}
\begin{lstlisting}
gls_homoC <- gls(Y ~ session,
		 correlation = corCompSymm(form = ~ 1|Id),
		 data = dtL.icc2)

# correlation matrix
unclass(cov2cor(getVarCov(gls_homoC)))
\end{lstlisting}

\begin{verbatim}
          [,1]      [,2]      [,3]      [,4]      [,5]
[1,] 1.0000000 0.8057232 0.8057232 0.8057232 0.8057232
[2,] 0.8057232 1.0000000 0.8057232 0.8057232 0.8057232
[3,] 0.8057232 0.8057232 1.0000000 0.8057232 0.8057232
[4,] 0.8057232 0.8057232 0.8057232 1.0000000 0.8057232
[5,] 0.8057232 0.8057232 0.8057232 0.8057232 1.0000000
\end{verbatim}

\clearpage

\emph{Note:} small difference between the function \texttt{icc} and
\texttt{gls}. Due to non optimal optimization in \texttt{gls}?

\lstset{language=r,label= ,caption= ,captionpos=b,numbers=none}
\begin{lstlisting}
irr::icc(dfW.icc2[,c("Y1", "Y2", "Y3", "Y4", "Y5")],
	 model="oneway", type="consistency")$value
\end{lstlisting}

\begin{verbatim}
[1] 0.7999099
\end{verbatim}
\end{document}