% Created 2023-02-19 Sun 16:54
% Intended LaTeX compiler: pdflatex
\documentclass[12pt]{article}

%%%% settings when exporting code %%%% 

\usepackage{listings}
\lstdefinestyle{code-small}{
backgroundcolor=\color{white}, % background color for the code block
basicstyle=\ttfamily\small, % font used to display the code
commentstyle=\color[rgb]{0.5,0,0.5}, % color used to display comments in the code
keywordstyle=\color{black}, % color used to highlight certain words in the code
numberstyle=\ttfamily\tiny\color{gray}, % color used to display the line numbers
rulecolor=\color{black}, % color of the frame
stringstyle=\color[rgb]{0,.5,0},  % color used to display strings in the code
breakatwhitespace=false, % sets if automatic breaks should only happen at whitespace
breaklines=true, % sets automatic line breaking
columns=fullflexible,
frame=single, % adds a frame around the code (non,leftline,topline,bottomline,lines,single,shadowbox)
keepspaces=true, % % keeps spaces in text, useful for keeping indentation of code
literate={~}{$\sim$}{1}, % symbol properly display via latex
numbers=none, % where to put the line-numbers; possible values are (none, left, right)
numbersep=10pt, % how far the line-numbers are from the code
showspaces=false,
showstringspaces=false,
stepnumber=1, % the step between two line-numbers. If it's 1, each line will be numbered
tabsize=1,
xleftmargin=0cm,
emph={anova,apply,class,coef,colnames,colNames,colSums,dim,dcast,for,ggplot,head,if,ifelse,is.na,lapply,list.files,library,logLik,melt,plot,require,rowSums,sapply,setcolorder,setkey,str,summary,tapply},
aboveskip = \medskipamount, % define the space above displayed listings.
belowskip = \medskipamount, % define the space above displayed listings.
lineskip = 0pt} % specifies additional space between lines in listings
\lstset{style=code-small}
%%%% packages %%%%%

\usepackage[utf8]{inputenc}
\usepackage[T1]{fontenc}
\usepackage{lmodern}
\usepackage{textcomp}
\usepackage{color}
\usepackage{graphicx}
\usepackage{grffile}
\usepackage{wrapfig}
\usepackage{rotating}
\usepackage{longtable}
\usepackage{multirow}
\usepackage{multicol}
\usepackage{changes}
\usepackage{pdflscape}
\usepackage{geometry}
\usepackage[normalem]{ulem}
\usepackage{amssymb}
\usepackage{amsmath}
\usepackage{amsfonts}
\usepackage{dsfont}
\usepackage{array}
\usepackage{ifthen}
\usepackage{hyperref}
\usepackage{natbib}
%
%%%% specifications %%%%
%
\usepackage{ifthen}
\usepackage{xifthen}
\usepackage{xargs}
\usepackage{xspace}
\newcommand{\first}{1\textsuperscript{st} }
\newcommand{\second}{2\textsuperscript{nd} }
\newcommand{\third}{3\textsuperscript{rd} }
\RequirePackage{fancyvrb}
\DefineVerbatimEnvironment{verbatim}{Verbatim}{fontsize=\small,formatcom = {\color[rgb]{0.5,0,0}}}
\RequirePackage{colortbl} % arrayrulecolor to mix colors
\RequirePackage{setspace} % to modify the space between lines - incompatible with footnote in beamer
\renewcommand{\baselinestretch}{1.1}
\geometry{top=1cm}
\RequirePackage{colortbl} % arrayrulecolor to mix colors
\RequirePackage{pifont}
\RequirePackage{relsize}
\newcommand{\Cross}{{\raisebox{-0.5ex}%
{\relsize{1.5}\ding{56}}}\hspace{1pt} }
\newcommand{\Valid}{{\raisebox{-0.5ex}%
{\relsize{1.5}\ding{52}}}\hspace{1pt} }
\newcommand{\CrossR}{ \textcolor{red}{\Cross} }
\newcommand{\ValidV}{ \textcolor{green}{\Valid} }
\usepackage{stackengine}
\usepackage{scalerel}
\newcommand\Warning[1][3ex]{%
\renewcommand\stacktype{L}%
\scaleto{\stackon[1.3pt]{\color{red}$\triangle$}{\tiny\bfseries !}}{#1}%
\xspace
}
\hypersetup{
citecolor=[rgb]{0,0.5,0},
urlcolor=[rgb]{0,0,0.5},
linkcolor=[rgb]{0,0,0.5},
}
\RequirePackage{epstopdf} % to be able to convert .eps to .pdf image files
\RequirePackage{capt-of} %
\RequirePackage{caption} % newlines in graphics
\RequirePackage{enumitem} % to be able to convert .eps to .pdf image files
\definecolor{light}{rgb}{1, 1, 0.9}
\definecolor{lightred}{rgb}{1.0, 0.7, 0.7}
\definecolor{lightblue}{rgb}{0.0, 0.8, 0.8}
\newcommand{\darkblue}{blue!80!black}
\newcommand{\darkgreen}{green!50!black}
\newcommand{\darkred}{red!50!black}
\usepackage{mdframed}
\definecolor{grayR}{HTML}{8A8990}
\definecolor{grayL}{HTML}{C4C7C9}
\definecolor{blueM}{HTML}{1F63B5}
\newcommand{\Rlogo}[1][0.07]{
\begin{tikzpicture}[scale=#1]
\shade [right color=grayR,left color=grayL,shading angle=60]
(-3.55,0.3) .. controls (-3.55,1.75)
and (-1.9,2.7) .. (0,2.7) .. controls (2.05,2.7)
and (3.5,1.6) .. (3.5,0.3) .. controls (3.5,-1.2)
and (1.55,-2) .. (0,-2) .. controls (-2.3,-2)
and (-3.55,-0.75) .. cycle;

\fill[white]
(-2.15,0.2) .. controls (-2.15,1.2)
and (-0.7,1.8) .. (0.5,1.8) .. controls (2.2,1.8)
and (3.1,1.2) .. (3.1,0.2) .. controls (3.1,-0.75)
and (2.4,-1.45) .. (0.5,-1.45) .. controls (-1.1,-1.45)
and (-2.15,-0.7) .. cycle;

\fill[blueM]
(1.75,1.25) -- (-0.65,1.25) -- (-0.65,-2.75) -- (0.55,-2.75) -- (0.55,-1.15) --
(0.95,-1.15)  .. controls (1.15,-1.15)
and (1.5,-1.9) .. (1.9,-2.75) -- (3.25,-2.75)  .. controls (2.2,-1)
and (2.5,-1.2) .. (1.8,-0.95) .. controls (2.6,-0.9)
and (2.85,-0.35) .. (2.85,0.2) .. controls (2.85,0.7)
and (2.5,1.2) .. cycle;

\fill[white]  (1.4,0.4) -- (0.55,0.4) -- (0.55,-0.3) -- (1.4,-0.3).. controls (1.75,-0.3)
and (1.75,0.4) .. cycle;

\end{tikzpicture}
}
\date{}
\title{Teaching Portfolio}
\hypersetup{
 colorlinks=true,
 pdfauthor={},
 pdftitle={Teaching Portfolio},
 pdfkeywords={},
 pdfsubject={},
 pdfcreator={Emacs 26.3 (Org mode 9.4.6)},
 pdflang={English}
 }
\begin{document}

\maketitle

\section{Teaching responsabilities (per year)}
\label{sec:org8128d3d}

Current teaching activity at the University of Copenhagen (KU) for Phd
students in medical sciences:

\smallskip

\begin{tabular}{l@{ }l}
2015 - 2022 : & \href{https://absalon.ku.dk/courses/47665}{Statistical analysis of repeated measurements} (course director Julie Forman). \\
              & 3 lectures of 3 hours and 6 practicals of 3 hours\\
              & Development of a dedicated \Rlogo package for the course (\href{https://cran.r-project.org/web/packages/LMMstar/index.html}{LMMstar}) \\
2021 - 2023 : & \href{https://absalon.ku.dk/courses/58764}{Epidemiological methods in medical research} as course director. \\
              & 3.5 lectures of 3 hours, 7 practical of 3 hours, 1/2 day student presentations \\
2021 - 2023 : & \href{http://paulblanche.com/files/BasicStat2022.html}{Basic statistics}  (course director Paul Blanche). \\
              & 2 lecture of 3 hours, 2 practical of 3 hours, 2 day student presentations
\end{tabular}

\bigskip

\noindent Past teaching activity at KU for students in Master of
statistics:

\begin{tabular}{l@{ }l}
2016 - 2017 : & Structural Equation Models (2h, lecture). \\
\end{tabular}

\bigskip

\noindent Past teaching activity at the University of Lyon 1 (France) for students in Master of biostatistics

\smallskip

\begin{tabular}{l@{ }l}
2014 - 2015 : & \href{http://mastersantepublique.univ-lyon1.fr/webapp/website/website.html?id=3124911&pageId=215839}{Survival Analysis} for Master students in public health (18h, practicals).\\
2013 - 2015 : & \href{http://mastersantepublique.univ-lyon1.fr/webapp/website/website.html?id=3124911&pageId=215839}{Bayesian statistics} for Master students in public health (6h, practicals).\\
\end{tabular}




\section{Formal pedagogic training}
\label{sec:org34e13a4}
\begin{tabular}{r@{ }l}
April 2022 : & \href{https://absalon.ku.dk/courses/58829}{Introduction to University Pedagogy} \\
 2022-2023 : & \href{https://absalon.ku.dk/courses/58114}{University Pedagogy} (Universitetspædagogikum) \\
\end{tabular}
Key learning points:
\begin{itemize}
\item Didactic triangle: a learner acquires a knowledge or a skill with
the help of a teacher by being exposed to notions and interacting
with them.
\item planning and organizing teaching around intended learning outcomes
(ILOs) and phases from the theory of didactical situations (TDS):
\item Involve students using an inductive (i.e. problem-based) approach
\item constructively receive feedback on their own teaching.
\item 
\end{itemize}

Didactic triangle

Plan, conduct and evaluate own teaching with student activities
Select and apply the introduced pedagogical themes in own teaching (onsite as well as online)
Identify pedagogical themes and provide feedback on the teaching of others,
Discuss the possibilities and limitations of teaching, based on the pedagogical themes: 


\section{Course development}
\label{sec:org24b73be}
\section{Pedagogical supervison}
\label{sec:orgf2c03f9}
\section{Evaluation}
\label{sec:org1a72725}
\section{Personal description}
\label{sec:orga33c582}
\subsection{Theoretical anchoring}
\label{sec:org5d741a7}
\subsection{Reflection on own teaching}
\label{sec:org76a7728}
\end{document}